\documentclass[10pt,fleqn]{article} % Default font size and left-justified equations
\usepackage[%
    pdftitle={Modélisation dynamique},
    pdfauthor={Xavier Pessoles}]{hyperref}

\input{style/new_style}
\input{style/macros_SII}
\usepackage{bm}
\fichetrue
\fichefalse

\proftrue
%\proffalse

%\tdtrue
\tdfalse

\courstrue
%\coursfalse



% -------------------------------------
% Déclaration des titres
% -------------------------------------

\def\discipline{Sciences \\Industrielles de \\ l'Ingénieur}
\def\xxtete{Sciences Industrielles de l'Ingénieur}

\def\classe{\textsf{PSI$\star$ -- MP}}
\def\xxnumpartie{Cycle 04}
\def\xxpartie{Modéliser le comportement des systèmes mécaniques dans le but d'établir une loi de comportement ou de déterminer des actions mécaniques en utilisant le PFD}

\def\xxnumchapitre{Chapitre 2 \vspace{.2cm}}
\def\xxchapitre{\hspace{.12cm} Caractéristation inertielle des solides}

\def\xxposongletx{2}
\def\xxposonglettext{1.45}
\def\xxposonglety{19}%16

\def\xxonglet{Cycle 04}

\def\xxactivite{Cours}
\def\xxauteur{\textsl{Xavier Pessoles}}

\def\xxcompetences{%
\textsl{%
\textbf{Savoirs et compétences :}\\
\begin{itemize}[label=\ding{112},font=\color{ocre}] 
\item \textit{Mod2.C13} : centre d'inertie
\item \textit{Mod2.C14} : opérateur d'inertie
\item \textit{Mod2.C15} : matrice d'inertie
\end{itemize}
}}
		
\def\xxfigures{
\includegraphics[width=4cm]{images/fig_01}\\
\textit{Toupie}

\includegraphics[width=4cm]{images/fig_02}\\
\textit{Volants d'inertie d'un vilebrequin}

}%figues de la page de garde

\def\xxpied{%
Cycle 04 -- Modéliser le comportement des systèmes mécaniques\\% afin de valider leurs performances.\\
Chapitre 2 -- \xxactivite%
}

\setcounter{secnumdepth}{5}
%---------------------------------------------------------------------------


\begin{document}
\chapterimage{png/Fond_CIN}
\input{style/new_pagegarde}
\setlength{\columnseprule}{.1pt}

\vspace{2cm}
\pagestyle{fancy}
\thispagestyle{plain}


\section{Masse et centre de masse (centre d'inertie)}
\subsection{Masse d'un solide indéformable}
\begin{defi}
On peut définir la masse totale d'un système $E$ par : $M=\int\limits_{P\in E} \,\dd m$. Si de plus l'ensemble est fait d'un matériau homogène de masse volumique $\mu$, on a $M=\mu \int\limits_{P\in E} \dd V$. \end{defi}

\subsection{Centre d'inertie d'un solide}

\begin{defi}
La position du centre d'inertie $G$ d'un ensemble matériel $E$ est définie par $\int\limits_{P\in E} \vect{GP} \dd m = \vect{0}$.
\end{defi}

Pour déterminer la position du centre d'inertie d'un ensemble $E$, on passe généralement par l'origine du repère associé à $E$. On a alors 
$\int\limits_{P\in E} \vect{GP} \, \dd m=\int\limits_{P\in E} \left(\vect{GO}+\vect{OP}\right) \dd m = \vect{0} 
\Leftrightarrow \int\limits_{P\in E} \vect{OG} \,\dd m =\int\limits_{P\in E} \vect{OP} \,\dd m
\Leftrightarrow  M\vect{OG} =\int\limits_{P\in E} \vect{OP} \,\dd m$.

\begin{methode}
Pour déterminer les coordonnées $\left(x_G,y_G,z_G\right)$ du centre d'inertie $G$ du solide $E$ dans la base $\base{O}{x}{y}{z}$, on a donc :
$$
\left\{
\begin{array}{l}
M x_G =\mu \int\limits_{P\in E} x_P \,\dd V \\
M y_G =\mu \int\limits_{P\in E} y_P \,\dd V \\
M z_G =\mu \int\limits_{P\in E} z_P \,\dd V \\
\end{array}
\right. \quad \text{avec }\dd V \text{ volume élémentaire du solide $E$.}
$$ 

Pour simplifier les calculs, on peut noter que le centre d'inertie appartient au(x) éventuel(s) plan(s) de symétrie du solide.
\end{methode}

\begin{rem}
Centre d'inertie et centre de gravité sont confondus lorsque le champ de pesanteur est considéré comme uniforme en tout point de l'espace. 
\end{rem}

\subsection{Centre d'inertie d'un ensemble de solides encastrés entre eux}
\begin{methode}
Soit un solide composé de $n$ solides élémentaires dont la position des centres d'inertie $G_i$ et les masses $M_i$ sont connues. On note $M=\sum\limits_{i=1}^{n}M_i$.  La position du centre d'inertie $G$ de l'ensemble $E$ est donné par :
$$\vect{OG}=\dfrac{1}{M}\sum\limits_{i=1}^{n}M_i \vect{OG_i} .$$

\end{methode}




\section{Matrice d'inertie d'un solide}
\subsection{Opérateur et matrice d'inertie}

\subsection[Déplacement d'une matrice d'inertie]{Déplacement d'une matrice d'inertie -- Théorème de Huygens}

\subsection{Détermination de la matrice d'inertie d'un solide}

\subsection{Compléments}

\subsection{Matrice d'inertie de soldies usuels}


\begin{thebibliography}{2}
   \bibitem[1]{ref1} Emilien Durif, {\it Introduction à la dynamique des solides, Lycée La Martinière Monplaisir, Lyon.}
      \bibitem[2]{ref2} Florestan Mathurin, {\it Correction des SLCI, Lycée Bellevue, Toulouse, \url{http://florestan.mathurin.free.fr/}.}



\end{thebibliography}

\end{document}




