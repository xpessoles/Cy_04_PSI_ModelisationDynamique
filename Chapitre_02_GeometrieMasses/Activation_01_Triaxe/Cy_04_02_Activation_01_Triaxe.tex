\documentclass[10pt,fleqn]{article} % Default font size and left-justified equations
\usepackage[%
    pdftitle={Modélisation dynamique : cinétique},
    pdfauthor={Xavier Pessoles}]{hyperref}

    
\input{style/new_style}
\input{style/macros_SII}

\fichetrue
%\fichefalse

\proftrue
%\proffalse

\tdtrue
%\tdfalse

\courstrue
\coursfalse


\def\discipline{Sciences \\Industrielles de \\ l'Ingénieur}
\def\xxtete{Sciences Industrielles de l'Ingénieur}

\def\classe{\textsf{PSI$\star$ -- MP}}
\def\xxnumpartie{Cycle 04}
\def\xxpartie{Modéliser le comportement des systèmes mécaniques dans le but d'établir une loi de comportement ou de déterminer des actions mécaniques en utilisant le PFD}

\def\xxnumchapitre{Chapitre 2 \vspace{.2cm}}
\def\xxchapitre{\hspace{.12cm} Caractérisation inertielle des solides}




\def\xxtitreexo{Activation 1}%Motorisation du moteur Haibike}
\def\xxsourceexo{ \hspace{.2cm} \footnotesize{X. Pessoles}}


\def\xxposongletx{2}
\def\xxposonglettext{1.45}
\def\xxposonglety{20}
%\def\xxonglet{Part. 1 -- Ch. 3}
\def\xxonglet{Cycle 04}

\def\xxactivite{Activation 1}
\def\xxauteur{\textsl{X. Pessoles}}

\def\xxcompetences{%
\textsl{%
\textbf{Savoirs et compétences :}\\
\begin{itemize}[label=\ding{112},font=\color{ocre}] 
\item \textit{Mod2.C13} : centre d'inertie
\item \textit{Mod2.C14} : opérateur d'inertie
\item \textit{Mod2.C15} : matrice d'inertie
\end{itemize}
}}
\def\xxfigures{
%\includegraphics[width=.7\linewidth]{images/fig_00}
}%figues de la page de garde


\def\xxpied{%
Cycle 04 -- Modélisation mécanique -- Cinétique\\% afin de valider leurs performances.\\
Chapitre 2 -- \xxactivite%
}

\setcounter{secnumdepth}{5}
%---------------------------------------------------------------------------

\usepackage{pgfplots}
\begin{document}
\def\pathfig{images}
%\chapterimage{png/Fond_Cin}
\input{style/new_pagegarde}
\vspace{5cm}
\pagestyle{fancy}
\thispagestyle{plain}

\def\columnseprulecolor{\color{ocre}}
\setlength{\columnseprule}{0.4pt} 

\def\pathfig{images}

\ifprof
\else
\begin{multicols}{2}
\fi
%\section*{Encore un vilebrequin}
%On donne le plan d'un vilebrequin. 
%
%\begin{center}
%\includegraphics[width=\linewidth]{images/plan.png}
%\end{center}
%
%\subparagraph{}
%\textit{Indiquer la méthode pour déterminer le tenseur d'inertie en $O$ sur la base $\base{X}{Y}{Z}$.}
%\ifprof
%\begin{corrige}
%\end{corrige}
%\else
%\fi
%
%On donne $\mu=\SI{7,8}{kg.dm^{-3}}$.
%\subparagraph{}
%\textit{Déterminer la matrice de tenseur d'inertie.}
%\ifprof
%\begin{corrige}
%\end{corrige}
%\else
%\fi
%
%Pour cette pièce, SolidWorks nous donne les informations suivantes. 
%\begin{center}
%\includegraphics[width=\linewidth]{images/SW_03.png}
%\end{center}
%
%%
%%Propriétés de masse de Vilebrequin
%%     Configuration: Défaut
%%     Système de coordonnées: (X,Y,Z)
%%
%%Densité = 0.0078 grammes par millimètre cube
%%
%%Masse = 836 grammes
%%
%%Volume = 1.07e+005 millimètres cubes
%%
%%Superficie = 1.67e+004  millimètres carrés
%%
%%Centre de gravité: ( millimètres )
%%	X = -2.62
%%	Y = 5.89
%%	Z = 0
%%
%%Principaux axes et moments d'inertie: ( grammes *  millimètres carrés )
%%Pris au centre de gravité.
%%	 Ix = ( 0.96,  0.26,  0)   	Px = 1.85e+005
%%	 Iy = (-0.26,  0.96,  0)   	Py = 6.67e+005
%%	 Iz = ( 0,  0,  1)   	Pz = 6.74e+005
%%
%%Moments d'inertie: ( grammes *  millimètres carrés )
%%Pris au centre de gravité et aligné avec le système de coordonnées de sortie.
%%	Lxx = 2.18e+005	Lxy = 1.23e+005	Lxz = 0
%%	Lyx = 1.23e+005	Lyy = 6.34e+005	Lyz = 0
%%	Lzx = 0	Lzy = 0	Lzz = 6.74e+005
%%
%%Moments d'inertie: ( grammes *  millimètres carrés )
%%Pris au système de coordonnées de sortie.
%%	Ixx = 2.47e+005	Ixy = 1.1e+005	Ixz = 0
%%	Iyx = 1.1e+005	Iyy = 6.4e+005	Iyz = 0
%%	Izx = 0	Izy = 0	Izz = 7.09e+005
%%	
%\subparagraph{}
%\textit{Détailler ce que cela signifie.}
%\ifprof
%\begin{corrige}
%\end{corrige}
%\else

\section*{Triaxe}
\setcounter{exo}{0}
On donne le plan d'un triaxe constitué des 3 axes $A_1$, $A_2$, $A_3$ et du moyeu central noté $M$. On note  $T$ l'ensemble.

\begin{center}
\includegraphics[width=\linewidth]{images/triaxe.png}
\end{center}
On note $\vect{z}$ l'axe perpendiculaire au plan de la feuille. On se place ci-dessus dans le plan de symétrie $\left(O,\vect{x},\vect{y}\right)$.


\textbf{TOUS LES CALCULS SE FERONT DE MANIÈRE LITTEREALE !}
\begin{itemize}
\item $D_1=\SI{18}{mm}$ et $H_1=\SI{25}{mm}$.
\item $D=\SI{46}{mm}$, $D'=\SI{30}{mm}$ et $H=\SI{48}{mm}$.
\item $\alpha_1=\angl{x}{x_1}=90\degres$, $\alpha_2=\angl{x}{x_2}=-150\degres$ et 
$\alpha_3=\angl{x}{x_3}=-30\degres$.
\end{itemize}

\subparagraph{}
\textit{Déterminer (sans calcul) la position du centre de gravité du triaxe. }
\ifprof
\begin{corrige} ~\\

Le plan $\left(O,\vect{x},\vect{y}\right)$ est plan de symétrie du triaxe; donc $\vect{OG}\cdot\vect{z} =0$

Le plan $\left(O,\vect{y},\vect{z}\right)$ est plan de symétrie du triaxe; donc $\vect{OG}\cdot\vect{x} =0$

Reste la coordonnée selon $\vect{y}$. 

Les plans $\left(O,\vect{z},\vect{x_2}\right)$ et $\left(O,\vect{z},\vect{x_3}\right)$ étant plans de symétrie, on a 
 $\vect{OG}\cdot\vect{y_2} =0$ et  $\vect{OG}\cdot\vect{y_3} =0$. 
 Or $\vect{OG}= y_g \vect{y} =  y_g \cos \alpha_2 \vect{y_2} - y_g \sin \alpha_2 \vect{x_2}$. Il en résulte que   $y_g \cos \alpha_2 = 0$ et donc nécessairement $y_g=0$ car $\alpha_2\neq 0$. 
\end{corrige}
\else
\fi



\subparagraph{}
\textit{Déterminer analytiquement la position du centre de gravité $G_1$ du solide $A_1$ dans le repère $\rep{1}\repere{O_1}{x_1}{y_1}{z_1}$.}
\ifprof
\begin{corrige}
On pourrait répondre directement en disant que le solide à 3 plans de symétrie orthogonaux entre eux. 
En utilisant la définition on a :
\begin{itemize}
\item $M_1 =\mu H_1\pi \dfrac{D_1^2}{4}$;
\item en coordonnées cylindriques, $\vect{O_1P}=x\vect{x_1}+ \rho \cos\theta \vect{y_1}+\rho \sin\theta \vect{z_1}$ et $\dd V = \rho\dd \rho \dd \theta \dd x$ avec $x\in[0,H_1]$, $\theta\in [0,2\pi]$, $\rho \in \left[0,D_1/2\right]$;
\item $m_1x_{G_1}=\mu \iiint x_P dV = \mu \iiint x  \rho\dd \rho \dd \theta \dd x= \mu \dfrac{H_1^2}{2} 2\pi \dfrac{D_1^2}{8}$;
\item $m_1y_{G_1}=\mu \iiint y_P dV = \mu \iiint \rho \cos\theta   \rho\dd \rho \dd \theta \dd x=0$;
\item $m_1z_{G_1}=\mu \iiint z_P dV = \mu \iiint \rho \sin\theta  \rho\dd \rho \dd \theta \dd x=0$. 
\end{itemize}
Au final,  $\mu H_1\pi  \dfrac{D_1^2}{4}x_{G_1}=\mu \dfrac{H_1^2}{2}  2\pi  \dfrac{D_1^2}{8} \Leftrightarrow   x_{G_1}= \dfrac{H_1}{2}$.
\end{corrige}
\else
\fi


\subparagraph{}
\textit{Déterminer (sans calcul) la \textbf{forme} de la matrice d'inertie du triaxe.}
\ifprof
\begin{corrige}
Me plan $\left(O,\vect{x},\vect{y}\right)$ est plan de symétrie du triaxe; donc $E = \iiint xz \dd m = 0 $ et $D = \iiint yz \dd m = 0 $.

Le plan $\left(O,\vect{y},\vect{z}\right)$ est plan de symétrie du triaxe; donc $E = \iiint xz \dd m = 0 $ et $E = \iiint xy \dd m = 0 $.

La matrice est donc diagonale et de la forme $\matinertie{A_1}{B_1}{C_1}{0}{0}{0}{\rep{}}$.

\end{corrige}
\else
\fi

\subparagraph{}
\textit{Déterminer analytiquement la matrice d'inertie du solide $A_1$ en $G_1$ dans $\rep{1}$. On la note $\inertie{G_1}{A_1}=\matinertie{A_1}{B_1}{C_1}{-D_1}{-E_1}{-F_1}{\rep{1}}$ où les constantes seront à déterminer.}
\ifprof
\begin{corrige}
Au vu de la forme du solide, on a : $D_1=E_1=F_1 = 0$ et $C=B$. D'où  $\inertie{G_1}{A_1}=\matinertie{A_1}{B_1}{B_1}{0}{0}{0}{\rep{1}}$.

Calculons $A_1 = \iiint \left(y^2+z^2\right)  \dd m=\mu \iiint \left(\rho^2\cos^2\theta +\rho^2\sin^2\theta\right) \rho \dd \rho \dd \theta \dd x  $

$ = \mu \iiint \rho^3  \dd \rho \dd \theta \dd z  = \mu \left[ \dfrac{\rho^4}{4}\right]_{0}^{D_1/2}2\pi H_1 = \mu \dfrac{D_1^4}{16\cdot 4}2\pi H_1 = M_1 \dfrac{D_1^2}{8}$.


Calculons $B_1 = \iiint \left(x^2+z^2\right)  \dd m=\mu \iiint \left(x^2 +\rho^2\sin^2\theta\right) \rho \dd \rho \dd \theta \dd x$ 

$B_x=\mu \iiint x^2  \rho \dd \rho \dd \theta \dd x+\mu \iiint \rho^2\sin^2\theta \rho \dd \rho \dd \theta \dd x  $
$=\mu \iiint x^2  \rho \dd \rho \dd \theta \dd x = \mu \dfrac{H_1^3}{4\cdot 3}\dfrac{D_1^2}{8} 2\pi  = M  \dfrac{H_1^2}{12}   $

$B_z =\mu \iiint \rho^2\sin^2\theta \rho \dd \rho \dd \theta \dd x =\mu \iiint \rho^3\dfrac{1-\cos 2x}{2}\theta  \dd \rho \dd \theta \dd x=\mu \iiint \dfrac{\rho^3}{2}\theta  \dd \rho \dd \theta \dd x =\mu \dfrac{D_i^4}{2\cdot 16\cdot 4}2 \pi H_1 =M  \dfrac{D_i^2}{16}$.

Au final, $A =M_1 \dfrac{D_1^2}{8}$ et  $B =M\left(\dfrac{H_1^2}{12} + \dfrac{D_1^2}{16} \right)$.

%$\mu  \iiint y^2 \dd V +\mu  \iiint z^2 \dd V $.

%$ \mu  \iiint y^2 \dd V =  \mu  \iiint  \rho^2\cos^2\theta \dd V $ 
%$ = \mu  \iiint  \rho^2 \dd V \;  \iiint  \cos^2\theta \dd V  $

%$ = \mu  \iiint  \rho^3   \dfrac{1+\cos2\theta }{2} \dd \rho \dd \theta \dd z   $
%$ = \dfrac{\mu}{2}  \iiint     \rho^3+\rho^3\cos2\theta \dd \rho \dd \theta \dd z   $

%$ = \dfrac{\mu}{2} \left(\dfrac{D_1^4}{16\cdot 4} 2\pi H_1  +  \dfrac{D_1^4}{16\cdot 4} \left[ \dfrac{\sin 2 \theta}{2}\right]_{0}^{2\pi} H_1\right) = \dfrac{\mu}{2} \dfrac{D_1^4}{16\cdot 4} 2\pi H_1  =M_1 \dfrac{D_1^2}{16}  $

\end{corrige}
\else
\fi

\subparagraph{}
\textit{Déterminer $\inertie{G_1}{A_1}$ dans la base $\mathcal{B}\base{x}{y}{z}$ puis 
$\inertie{O}{A_1}$ dans la base $\mathcal{B}\base{x}{y}{z}$.}
\ifprof
\begin{corrige}
 On a $\vect{x_1}=\cos\alpha\vect{x}+\sin\alpha\vect{y}$, $\vect{y_1}=\cos\alpha\vect{y}-\sin\alpha\vect{x}$. 
 En conséquences, on a : 
$P_{10}=\begin{pmatrix} \cos \alpha & - \sin\alpha & 0 \\ \sin \alpha & \cos \alpha & 0 \\ 0 & 0 & 1 \\ \end{pmatrix}$. 
On a donc 
$\inertie{G_1}{A_1}_{\rep{}}=P_{10}^{-1}\inertie{G_1}{A_1}_{\rep{1}}P_{10}$. 

$\inertie{G_1}{A_1}_{\rep{}}=
\begin{pmatrix} \cos \alpha & \sin\alpha & 0 \\-\sin \alpha & \cos \alpha & 0 \\ 0 & 0 & 1 \\ \end{pmatrix}
\matinertie{A_1}{B_1}{B_1}{0}{0}{0}{\rep{1}}
\begin{pmatrix} \cos \alpha & - \sin\alpha & 0 \\ \sin \alpha & \cos \alpha & 0 \\ 0 & 0 & 1 \\ \end{pmatrix}
$

$=
\begin{pmatrix} \cos \alpha & \sin\alpha & 0 \\-\sin \alpha & \cos \alpha & 0 \\ 0 & 0 & 1 \\ \end{pmatrix}
\begin{pmatrix}A_1 \cos \alpha & -A_1 \sin\alpha & 0 \\ B_1\sin \alpha & B_1\cos \alpha & 0 \\ 0 & 0 & B_1 \\ \end{pmatrix}
$
$=
\begin{pmatrix}
A_1 \cos^2\alpha +B_1 \sin^2\alpha& -A_1 \sin\alpha\cos\alpha+B_1\cos\alpha\sin\alpha & 0 \\ 
-A_1\sin \alpha\cos\alpha+B_1\cos\alpha\sin\alpha  &A_1\sin^2\alpha+ B_1\cos^2 \alpha & 0 \\ 0 & 0 & B_1 \\ \end{pmatrix}
$
%\textbf{Attention, pour aller de $\vect{x_1}$ vers $\vect{x}$, $\alpha=-\alpha_1$.}
Avec $\alpha=\pi/2$, 
on a : $\inertie{G_1}{A_1}_{\rep{}}=\begin{pmatrix}
B_1 &0 & 0 \\ 
0 & A_1 & 0 \\
 0 & 0 & B_1 \\\end{pmatrix}_{\rep{}}
$.


Par ailleurs, $\vect{OG_1}=\dfrac{H+D}{2}\vect{y}$; donc :
  $\inertie{O}{A_1}_{\rep{}}=\matinertie{B_1}{A_1}{B_1}{0}{0}{0}{\rep{}}+M_1\begin{pmatrix}
\left( \dfrac{H+D}{2}\right)^2 &0 & 0 \\ 
0 & 0& 0 \\
 0 & 0 & \left( \dfrac{H+D}{2}\right)^2 \\\end{pmatrix}_{\rep{}}$.
 
Au final, 
$\inertie{O}{A_1}_{\rep{}}=\matinertie{B_1+M_1\left( \dfrac{H+D}{2}\right)^2}{A_1}{B_1+M_1\left( \dfrac{H+D}{2}\right)^2}{0}{0}{0}{\rep{}}$.
 
\end{corrige}
\else
\fi

\subparagraph{}
\textit{Déterminer $\inertie{O}{A_2}$  et $\inertie{O}{A_3}$ dans la base $\mathcal{B}\base{x}{y}{z}$.}
\ifprof

\begin{corrige}  ~\\


 $\inertie{G_2}{A_2}_{\rep{}}=
\begin{pmatrix}
A_1 \cos^2\alpha +B_1 \sin^2\alpha& -A_1 \sin\alpha\cos\alpha+B_1\cos\alpha\sin\alpha & 0 \\ 
-A_1\sin \alpha\cos\alpha+B_1\cos\alpha\sin\alpha  &A_1\sin^2\alpha+ B_1\cos^2 \alpha & 0 \\ 0 & 0 & B_1 \\ \end{pmatrix}.$



Avec $\alpha=-\pi/6$, 
on a : 
 $\inertie{G_2}{A_2}_{\rep{}}=
\begin{pmatrix}
 \dfrac{3A_1 + B_1}{4} & \left(A_1-B_1\right) \dfrac{\sqrt{3}}{4} & 0 \\ 
 \left(A_1-B_1\right) \dfrac{\sqrt{3}}{4}  &\dfrac{A_1+3B_1}{4}  & 0 \\ 0 & 0 & B_1 \\ \end{pmatrix}.$


Par ailleurs, $\vect{OG_2}=\dfrac{H+D}{2}\cos\alpha\vect{x}+\dfrac{H+D}{2}\sin\alpha\vect{y}$; 


donc :
  $\inertie{O}{A_1}_{\rep{}}=\matinertie{B_1}{A_1}{B_1}{0}{0}{0}{\rep{}}+M_1\begin{pmatrix}
\left( \dfrac{H+D}{2}\right)^2\dfrac{1}{4}  & -\dfrac{(H+D)^2}{4}\dfrac{\sqrt{3}}{4}  & 0 \\ 
-\dfrac{(H+D)^2}{4}\dfrac{\sqrt{3}}{4}& \left( \dfrac{H+D}{2}\right)^2\dfrac{3}{4}& 0 \\
 0 & 0 & \left( \dfrac{H+D}{2}\right)^2 \\\end{pmatrix}_{\rep{}}$.
 

%
%Par ailleurs, $\vect{OG_2}=\dfrac{H+D}{2}\cos\alpha\vect{x}+\dfrac{H+D}{2}\sin\alpha\vect{y}$; 
%
%
%donc :
%  $\inertie{O}{A_1}_{\rep{}}=\matinertie{B_1}{A_1}{B_1}{0}{0}{0}{\rep{}}+M_1\begin{pmatrix}
%\left( \dfrac{H+D}{2}\right)^2 \sin^2\alpha  & -\dfrac{(H+D)^2}{4}\cos\alpha\sin\alpha  & 0 \\ 
%-\dfrac{(H+D)^2}{4}\cos\alpha\sin\alpha & \left( \dfrac{H+D}{2}\right)^2 \cos^2\alpha& 0 \\
% 0 & 0 & \left( \dfrac{H+D}{2}\right)^2 \\\end{pmatrix}_{\rep{}}$.
% 
%Au final, 
%$\inertie{O}{A_1}_{\rep{}}=\matinertie{B_1+M_1\left( \dfrac{H+D}{2}\right)^2}{A_1}{B_1+M_1\left( \dfrac{H+D}{2}\right)^2}{0}{0}{0}{\rep{}}$.
 
\end{corrige}
\else
\fi

\subparagraph{}
\textit{Déterminer $\inertie{O}{M}$  la matrice d'inertie du moyeu $M$.}
\ifprof
\begin{corrige}
 
\end{corrige}
\else
\fi


\subparagraph{}
\textit{Déterminer $\inertie{O}{T}$  la matrice d'inertie du triaxe $T$.}
\ifprof
\begin{corrige}
 
\end{corrige}
\else
\fi
%
%
%On donne $\mu=\SI{7,8}{kg.dm^{-3}}$.
%\subparagraph{}
%\textit{Déterminer ce moment d'inertie.}
%\ifprof
%\begin{corrige}
%\end{corrige}
%\else
%\fi
%
%\subparagraph{}
%\textit{Indiquer la méthode pour déterminer le tenseur d'inertie en $O$ dans la base $\base{x}{y}{z}$ ($O$ étant situé au centre de la pièce).}
%\ifprof
%\begin{corrige}
%\end{corrige}
%\else
%%\fi
%%
%On donne $\mu=\SI{7,8}{kg.dm^{-3}}$.
%\subparagraph{}
%\textit{Déterminer ce moment d'inertie.}
%\ifprof
%\begin{corrige}
%\end{corrige}
%\else
%\fi
%Pour cette pièce, SolidWorks nous donne les informations suivantes. 
%\begin{center}
%\includegraphics[width=.9\linewidth]{images/SW_04.png}
%\end{center}

%Propriétés de masse de TriAxe
%     Configuration: Défaut
%     Système de coordonnées: -- par défaut --
%
%Densité = 0.0078 grammes par millimètre cube
%
%Masse = 357 grammes
%
%Volume = 4.58e+004 millimètres cubes
%
%Superficie = 1.44e+004  millimètres carrés
%
%Centre de gravité: ( millimètres )
%	X = 0
%	Y = 0
%	Z = 0
%
%Principaux axes et moments d'inertie: ( grammes *  millimètres carrés )
%Pris au centre de gravité.
%	 Ix = ( 0,  0,  1)   	Px = 1.55e+005
%	 Iy = ( 1,  0,  0)   	Py = 1.55e+005
%	 Iz = ( 0,  1,  0)   	Pz = 2.77e+005
%
%Moments d'inertie: ( grammes *  millimètres carrés )
%Pris au centre de gravité et aligné avec le système de coordonnées de sortie.
%	Lxx = 1.55e+005	Lxy = 0	Lxz = 0
%	Lyx = 0	Lyy = 2.77e+005	Lyz = 0
%	Lzx = 0	Lzy = 0	Lzz = 1.55e+005
%
%Moments d'inertie: ( grammes *  millimètres carrés )
%Pris au système de coordonnées de sortie.
%	Ixx = 1.55e+005	Ixy = 0	Ixz = 0
%	Iyx = 0	Iyy = 2.77e+005	Iyz = 0
%	Izx = 0	Izy = 0	Izz = 1.55e+005
%\subparagraph{}
%\textit{Détailler ce que cela signifie.}
%\ifprof
%\begin{corrige}
%\end{corrige}
%\else
%
%
%




\ifprof
\else
\end{multicols}
\fi

\end{document}
\begin{center}
\includegraphics[width=\linewidth]{images/}
\end{center}

\subparagraph{}
\textit{}
\ifprof
\begin{corrige}
\end{corrige}
\else
\fi

