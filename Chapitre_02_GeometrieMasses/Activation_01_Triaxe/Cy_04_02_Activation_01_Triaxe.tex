\documentclass[10pt,fleqn]{article} % Default font size and left-justified equations
\usepackage[%
    pdftitle={Modélisation dynamique : cinétique},
    pdfauthor={Xavier Pessoles}]{hyperref}

    
\input{style/new_style}
\input{style/macros_SII}

\fichetrue
%\fichefalse

\proftrue
\proffalse

\tdtrue
%\tdfalse

\courstrue
\coursfalse


\def\discipline{Sciences \\Industrielles de \\ l'Ingénieur}
\def\xxtete{Sciences Industrielles de l'Ingénieur}

\def\classe{\textsf{PSI$\star$ -- MP}}
\def\xxnumpartie{Cycle 04}
\def\xxpartie{Modéliser le comportement des systèmes mécaniques dans le but d'établir une loi de comportement ou de déterminer des actions mécaniques en utilisant le PFD}

\def\xxnumchapitre{Chapitre 2 \vspace{.2cm}}
\def\xxchapitre{\hspace{.12cm} Caractérisation inertielle des solides}




\def\xxtitreexo{Activation 1}%Motorisation du moteur Haibike}
\def\xxsourceexo{ \hspace{.2cm} \footnotesize{X. Pessoles}}


\def\xxposongletx{2}
\def\xxposonglettext{1.45}
\def\xxposonglety{20}
%\def\xxonglet{Part. 1 -- Ch. 3}
\def\xxonglet{Cycle 04}

\def\xxactivite{Activation 1}
\def\xxauteur{\textsl{X. Pessoles}}

\def\xxcompetences{%
\textsl{%
\textbf{Savoirs et compétences :}\\
\begin{itemize}[label=\ding{112},font=\color{ocre}] 
\item \textit{Mod2.C13} : centre d'inertie
\item \textit{Mod2.C14} : opérateur d'inertie
\item \textit{Mod2.C15} : matrice d'inertie
\end{itemize}
}}
\def\xxfigures{
%\includegraphics[width=.7\linewidth]{images/fig_00}
}%figues de la page de garde


\def\xxpied{%
Cycle 04 -- Modélisation mécanique -- Cinétique\\% afin de valider leurs performances.\\
Chapitre 2 -- \xxactivite%
}

\setcounter{secnumdepth}{5}
%---------------------------------------------------------------------------

\usepackage{pgfplots}
\begin{document}
\def\pathfig{images}
%\chapterimage{png/Fond_Cin}
\input{style/new_pagegarde}
\vspace{5cm}
\pagestyle{fancy}
\thispagestyle{plain}

\def\columnseprulecolor{\color{ocre}}
\setlength{\columnseprule}{0.4pt} 

\def\pathfig{images}

\ifprof
\else
\begin{multicols}{2}
\fi
%\section*{Encore un vilebrequin}
%On donne le plan d'un vilebrequin. 
%
%\begin{center}
%\includegraphics[width=\linewidth]{images/plan.png}
%\end{center}
%
%\subparagraph{}
%\textit{Indiquer la méthode pour déterminer le tenseur d'inertie en $O$ sur la base $\base{X}{Y}{Z}$.}
%\ifprof
%\begin{corrige}
%\end{corrige}
%\else
%\fi
%
%On donne $\mu=\SI{7,8}{kg.dm^{-3}}$.
%\subparagraph{}
%\textit{Déterminer la matrice de tenseur d'inertie.}
%\ifprof
%\begin{corrige}
%\end{corrige}
%\else
%\fi
%
%Pour cette pièce, SolidWorks nous donne les informations suivantes. 
%\begin{center}
%\includegraphics[width=\linewidth]{images/SW_03.png}
%\end{center}
%
%%
%%Propriétés de masse de Vilebrequin
%%     Configuration: Défaut
%%     Système de coordonnées: (X,Y,Z)
%%
%%Densité = 0.0078 grammes par millimètre cube
%%
%%Masse = 836 grammes
%%
%%Volume = 1.07e+005 millimètres cubes
%%
%%Superficie = 1.67e+004  millimètres carrés
%%
%%Centre de gravité: ( millimètres )
%%	X = -2.62
%%	Y = 5.89
%%	Z = 0
%%
%%Principaux axes et moments d'inertie: ( grammes *  millimètres carrés )
%%Pris au centre de gravité.
%%	 Ix = ( 0.96,  0.26,  0)   	Px = 1.85e+005
%%	 Iy = (-0.26,  0.96,  0)   	Py = 6.67e+005
%%	 Iz = ( 0,  0,  1)   	Pz = 6.74e+005
%%
%%Moments d'inertie: ( grammes *  millimètres carrés )
%%Pris au centre de gravité et aligné avec le système de coordonnées de sortie.
%%	Lxx = 2.18e+005	Lxy = 1.23e+005	Lxz = 0
%%	Lyx = 1.23e+005	Lyy = 6.34e+005	Lyz = 0
%%	Lzx = 0	Lzy = 0	Lzz = 6.74e+005
%%
%%Moments d'inertie: ( grammes *  millimètres carrés )
%%Pris au système de coordonnées de sortie.
%%	Ixx = 2.47e+005	Ixy = 1.1e+005	Ixz = 0
%%	Iyx = 1.1e+005	Iyy = 6.4e+005	Iyz = 0
%%	Izx = 0	Izy = 0	Izz = 7.09e+005
%%	
%\subparagraph{}
%\textit{Détailler ce que cela signifie.}
%\ifprof
%\begin{corrige}
%\end{corrige}
%\else

\section*{Triaxe}
\setcounter{exo}{0}
On donne le plan d'un triaxe constitué des 3 axes $A_1$, $A_2$, $A_3$ et du moyeu central noté $M$. On note  $T$ l'ensemble.

\begin{center}
\includegraphics[width=\linewidth]{images/triaxe.png}
\end{center}
On note $\vect{z}$ l'axe perpendiculaire au plan de la feuille. On se place ci-dessus dans le plan de symétrie $\left(O,\vect{x},\vect{y}\right)$.


\textbf{TOUS LES CALCULS SE FERONT DE MANIÈRE LITTEREALE !}
\begin{itemize}
\item $D_1=\SI{18}{mm}$ et $H_1=\SI{25}{mm}$.
\item $D=\SI{46}{mm}$, $D'=\SI{30}{mm}$ et $H=\SI{48}{mm}$.
\item $\alpha=\angl{x}{x_2}=-150$ et 
$\beta=\angl{x}{x_3}=-30\degres$.
\end{itemize}

\subparagraph{}
\textit{Déterminer (sans calcul) la position du centre de gravité du triaxe. }
\ifprof
\begin{corrige}
\end{corrige}
\else
\fi



\subparagraph{}
\textit{Déterminer analytiquement la position du centre de gravité $G_1$ du solide $A_1$ dans le repère $\rep{1}\repere{O_1}{x_1}{y_1}{z_1}$.}
\ifprof
\begin{corrige}
\end{corrige}
\else
\fi


\subparagraph{}
\textit{Déterminer (sans calcul) la \textbf{forme} de la matrice d'inertie du triaxe.}
\ifprof
\begin{corrige}
\end{corrige}
\else
\fi

\subparagraph{}
\textit{Déterminer analytiquement la matrice d'inertie du solide $A_1$ en $G_1$ dans $\rep{1}$. On la note $\inertie{G_1}{A_1}=\matinertie{A_1}{B_1}{C_1}{-D_1}{-E_1}{-F_1}{\rep{1}}$ où les constantes seront à déterminer.}
\ifprof
\begin{corrige}
\end{corrige}
\else
\fi

\subparagraph{}
\textit{Déterminer $\inertie{G_1}{A_1}$ dans la base $\mathcal{B}\base{x}{y}{z}$ puis 
$\inertie{O}{A_1}$ dans la base $\mathcal{B}\base{x}{y}{z}$.}
\ifprof
\begin{corrige}
 
\end{corrige}
\else
\fi

\subparagraph{}
\textit{Déterminer $\inertie{O}{A_2}$  et $\inertie{O}{A_3}$ dans la base $\mathcal{B}\base{x}{y}{z}$.}
\ifprof
\begin{corrige}
 
\end{corrige}
\else
\fi

\subparagraph{}
\textit{Déterminer $\inertie{O}{M}$  la matrice d'inertie du moyeu $M$.}
\ifprof
\begin{corrige}
 
\end{corrige}
\else
\fi


\subparagraph{}
\textit{Déterminer $\inertie{O}{T}$  la matrice d'inertie du triaxe $T$.}
\ifprof
\begin{corrige}
 
\end{corrige}
\else
\fi
%
%
%On donne $\mu=\SI{7,8}{kg.dm^{-3}}$.
%\subparagraph{}
%\textit{Déterminer ce moment d'inertie.}
%\ifprof
%\begin{corrige}
%\end{corrige}
%\else
%\fi
%
%\subparagraph{}
%\textit{Indiquer la méthode pour déterminer le tenseur d'inertie en $O$ dans la base $\base{x}{y}{z}$ ($O$ étant situé au centre de la pièce).}
%\ifprof
%\begin{corrige}
%\end{corrige}
%\else
%%\fi
%%
%On donne $\mu=\SI{7,8}{kg.dm^{-3}}$.
%\subparagraph{}
%\textit{Déterminer ce moment d'inertie.}
%\ifprof
%\begin{corrige}
%\end{corrige}
%\else
%\fi
%Pour cette pièce, SolidWorks nous donne les informations suivantes. 
%\begin{center}
%\includegraphics[width=.9\linewidth]{images/SW_04.png}
%\end{center}

%Propriétés de masse de TriAxe
%     Configuration: Défaut
%     Système de coordonnées: -- par défaut --
%
%Densité = 0.0078 grammes par millimètre cube
%
%Masse = 357 grammes
%
%Volume = 4.58e+004 millimètres cubes
%
%Superficie = 1.44e+004  millimètres carrés
%
%Centre de gravité: ( millimètres )
%	X = 0
%	Y = 0
%	Z = 0
%
%Principaux axes et moments d'inertie: ( grammes *  millimètres carrés )
%Pris au centre de gravité.
%	 Ix = ( 0,  0,  1)   	Px = 1.55e+005
%	 Iy = ( 1,  0,  0)   	Py = 1.55e+005
%	 Iz = ( 0,  1,  0)   	Pz = 2.77e+005
%
%Moments d'inertie: ( grammes *  millimètres carrés )
%Pris au centre de gravité et aligné avec le système de coordonnées de sortie.
%	Lxx = 1.55e+005	Lxy = 0	Lxz = 0
%	Lyx = 0	Lyy = 2.77e+005	Lyz = 0
%	Lzx = 0	Lzy = 0	Lzz = 1.55e+005
%
%Moments d'inertie: ( grammes *  millimètres carrés )
%Pris au système de coordonnées de sortie.
%	Ixx = 1.55e+005	Ixy = 0	Ixz = 0
%	Iyx = 0	Iyy = 2.77e+005	Iyz = 0
%	Izx = 0	Izy = 0	Izz = 1.55e+005
%\subparagraph{}
%\textit{Détailler ce que cela signifie.}
%\ifprof
%\begin{corrige}
%\end{corrige}
%\else
%
%
%




\ifprof
\else
\end{multicols}
\fi

\end{document}
\begin{center}
\includegraphics[width=\linewidth]{images/}
\end{center}

\subparagraph{}
\textit{}
\ifprof
\begin{corrige}
\end{corrige}
\else
\fi

