\documentclass[10pt,fleqn]{article} % Default font size and left-justified equations
\usepackage[%
    pdftitle={Modélisation dynamique : cinétique},
    pdfauthor={Xavier Pessoles}]{hyperref}

    
\input{style/new_style}
\input{style/macros_SII}

\fichetrue
%\fichefalse

\proftrue
\proffalse

\tdtrue
%\tdfalse

\courstrue
\coursfalse


\def\discipline{Sciences \\Industrielles de \\ l'Ingénieur}
\def\xxtete{Sciences Industrielles de l'Ingénieur}

\def\classe{\textsf{PSI$\star$ -- MP}}
\def\xxnumpartie{Cycle 04}
\def\xxpartie{Modéliser le comportement des systèmes mécaniques dans le but d'établir une loi de comportement ou de déterminer des actions mécaniques en utilisant le PFD}

\def\xxnumchapitre{Chapitre 2 \vspace{.2cm}}
\def\xxchapitre{\hspace{.12cm} Caractéristation inertielle des solides}




\def\xxtitreexo{Application -- Vilebrequin de moteur}%Motorisation du moteur Haibike}
\def\xxsourceexo{\hspace{.2cm} \footnotesize{C. Gamelon \& P. Dubois}}


\def\xxposongletx{2}
\def\xxposonglettext{1.45}
\def\xxposonglety{20}
%\def\xxonglet{Part. 1 -- Ch. 3}
\def\xxonglet{Cycle 04}

\def\xxactivite{Application}
\def\xxauteur{\textsl{C. Gamelon \& P. Dubois}}

\def\xxcompetences{%
\textsl{%
\textbf{Savoirs et compétences :}\\
\begin{itemize}[label=\ding{112},font=\color{ocre}] 
\item \textit{Mod2.C13} : centre d'inertie
\item \textit{Mod2.C14} : opérateur d'inertie
\item \textit{Mod2.C15} : matrice d'inertie
\end{itemize}
}}
\def\xxfigures{
\includegraphics[width=.7\linewidth]{images/fig_00}
}%figues de la page de garde


\def\xxpied{%
Cycle 04 -- Modélisation mécanique -- Cinétique\\% afin de valider leurs performances.\\
Chapitre 2 -- \xxactivite%
}

\setcounter{secnumdepth}{5}
%---------------------------------------------------------------------------

\usepackage{pgfplots}
\begin{document}
\def\pathfig{images}
%\chapterimage{png/Fond_Cin}
\input{style/new_pagegarde}
\vspace{5cm}
\pagestyle{fancy}
\thispagestyle{plain}

\def\columnseprulecolor{\color{ocre}}
\setlength{\columnseprule}{0.4pt} 

\def\pathfig{images}

\ifprof
\else
\begin{multicols}{2}
\fi
Un vilebrequin est réalisé en mécanosoudage pour faire fonctionner un prototype de moteur. Les géométries sont par conséquent simples : assemblage de tôles ou cylindres en acier.

\begin{center}
\includegraphics[width=\linewidth]{images/fig_01}
\end{center}

On donne :
\begin{multicols}{2}
\begin{itemize}
\item $a = \SI{20}{mm}$;
\item $b = \SI{30}{mm}$;
\item $e = \SI{5}{mm}$;
\item $l = \SI{20}{mm}$;
\item $r = \SI{5}{mm}$;
\item $L = \SI{50}{mm}$;
\item $r_4 = \SI{7.5}{mm}$;
\item $h = \SI{20}{mm}$.
\end{itemize}
\end{multicols}
\subparagraph{}\textit{Calculer les masses des différentes pièces: $m_1$, $m_1$, $m_3$ et $m_4$.}
\subparagraph{}\textit{Déterminer le centre d’inertie de chaque pièce.}
\subparagraph{}\textit{Déterminer la valeur de $R$ afin que le centre d’inertie du vilebrequin soit sur son axe de rotation. Faire l’application numérique.}

\subparagraph{}\textit{Calculer les matrices d’inertie de chaque pièce au point où elles s’expriment de manière la plus simple et dans la base $\base{x}{y}{z}$.}
\subparagraph{}\textit{Calculer la matrice d’inertie du vilebrequin en $O$ dans la base $\base{x}{y}{z}$.}

Le carter moteur peut être basculé pour l’entretien. Cette opération ne doit normalement pas être effectuée lorsque le moteur fonctionne. Afin de calculer les effets dynamiques engendrés par cette manipulation, il est nécessaire de calculer l’inertie en rotation du vilebrequin par rapport à cet axe de rotation.
\subparagraph{}\textit{Calculer l’inertie en rotation par rapport à l’axe $\vect{OA}$.}


%\section*{Présentation du support du cours du cours}

\ifprof
\else
\end{multicols}
\fi
%
%\newpage
%Question 2 : Position des centres de gravité:
%$\vect{OG_1}=h/2.\vect{y}+\dfrac{l+e}{2}\vect{z}$,  $\vect{OG_2}=b/2\vect{y}$ $\vect{OG_4} =(L+e)/2.\vect{z}$; 
%Avec Guldin : $(4.\pi.R^3)/3=2.\pi.Z_{G_3}.(\pi.R^2)/2$;  $Z_{G_3}=(4.R)/(3.\pi)$. Ainsi 
%$\vect{OG_3}=-(4.R)/(3.\pi).\vect{y}$.
%
%Question 3 : Calcul de R pour que le centre de gravité de l'ensemble soit sur l'axe de rotation.
%
%$m_1 \vect{OG_1}+m_2 \vect{OG_2} +m_3 \vect{OG_3 }+m_4 \vect{OG_4} \vect{z} =0 $
%
%$\rho.(V_1 \vect{OG_1}+V_2 \vect{OG_2}+V_3 \vect{OG_3}+V_4 \vect{OG_4} \vect{z}=0$
%
%$V_1.b_1+V_2.b_2+V_3.b_3+0=0$
%
%$\pi r^2.l.h+a.b^2/2.e-(\pi R^2)/2.(4.R)/3\pi .e=0$
%
%$R=(3/(2.e).( \pi r^2.l.h+a.b^2/2.e))^{1/3}=\SI{28,4}{mm}$


\end{document}