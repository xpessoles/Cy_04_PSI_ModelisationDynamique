\documentclass[10pt,fleqn]{article} % Default font size and left-justified equations
\usepackage[%
    pdftitle={Résolutions de problèmes de statique : PFS 2D},
    pdfauthor={Xavier Pessoles}]{hyperref}

\input{style/new_style}
\input{style/macros_SII}
\usepackage{multicol}
\usepackage{siunitx}
%\usepackage{picins}
\fichetrue
%\fichefalse

\proftrue
\proffalse

\tdtrue
%\tdfalse

\courstrue
\coursfalse

% -------------------------------------
% Déclaration des titres
% -------------------------------------

\def\discipline{Sciences \\Industrielles de \\ l'Ingénieur}
\def\xxtete{Sciences Industrielles de l'Ingénieur}


\def\classe{\textsf{PSI$\star$ -- MP}}
\def\xxnumpartie{Rév -- Stat}
\def\xxpartie{Modéliser le comportement statique des systèmes mécaniques}

\def\xxnumchapitre{Révision 1 \vspace{.2cm}}
\def\xxchapitre{\hspace{.12cm} Résolution des problèmes de statique -- Statique plane}

\def\xxposongletx{2}
\def\xxposonglettext{1.45}
\def\xxposonglety{19}%16

\def\xxonglet{\textsf{Rév -- Cin}}

\def\xxactivite{Application 01}
\def\xxauteur{\textsl{Équipe PTSI La Martinière Monplaisir}}


\def\xxtitreexo{}
\def\xxsourceexo{\hspace{.2cm} \footnotesize{Équipe PTSI La Martinière Monplaisir}}

\def\xxcompetences{%
\textsl{%
\textbf{Savoirs et compétences :}\\
\begin{itemize}
\item Appliquer le PFS à un solide ou un système de solides;
\item Réaliser l’inventaire des actions mécaniques agissant sur un solide ou un système de solides;
%\item Identifier les paramètres cinématiques d’entrée et de sortie d’une chaîne cinématique de transformation de mouvement;
\item Identifier les puissances extérieures à un solide ou à un système de solides.
\end{itemize}
}}

\def\xxfigures{
%\includegraphics[width=.8\textwidth]{images/fig_01}
}%figues de la page de garde

\def\xxpied{%
Révision statique -- Résolution des problèmes de statique plane\\
Fiche 1 -- \xxactivite%
}

\setcounter{secnumdepth}{5}
%---------------------------------------------------------------------------


\begin{document}
%\chapterimage{png/Fond_Cin}
\input{style/new_pagegarde}
\vspace{6cm}
\pagestyle{fancy}
\thispagestyle{plain}


\def\columnseprulecolor{\color{ocre}}
\setlength{\columnseprule}{0.4pt} 

\ifprof
\else
\begin{multicols}{2}
\fi
\section*{Étude d'un mécanisme de levage}
Le mécanisme représenté schématiquement ci-dessus est destiné à assurer le levage d’une charge liée au coulisseau \textbf{(3)} au moyen d’un levier à excentrique \textbf{(1)} et d’un balancier \textbf{(2)}.

\begin{center}
\includegraphics[width=\linewidth]{images/fig_01}
%\textit{}
\end{center}
\begin{obj}
Objectif : Dans cette étude, on va mettre en évidence l’influence du frottement sur l’équilibre d’un système.
\end{obj}

On note  $\vect{P_3}$ le poids de la charge appliquée sur le coulisseau et $\vect{F_m}$ l’effort appliqué en $E$ par l’opérateur sur le levier à excentrique \textbf{(1)}.

\subsection*{Paramétrage géométrique}
$\vect{AB}=L_0\vect{x_0}$; 
$\vect{AE}=-L_1\vect{x_1}$; 
$\vect{BI}=d_0\vect{x_0}$; 
$\vect{AC}=e_1\vect{x_1}$; 
$\vect{HC}=R_1\vect{y_2}$; 
$\vect{BJ}=\lambda_{32}\vect{x_2}$; 
$\vect{ID}=\lambda_{30}\vect{y_0}$; 
$\vect{JD}=R_3\vect{y_2}$; 
$\left(\vect{x_0},\vect{x_1} \right)=\theta_{\left(1/0\right)}$; 
$\left(\vect{x_0},\vect{x_2} \right)=\theta_{\left(2/0\right)}$.
 	 	 	 	 

\subsection*{On suppose dans un premier temps que toutes les liaisons sont sans frottement.}
\subparagraph{}\textit{Justifier que le système est statiquement plan.}
\subparagraph{}\textit{En écrivant les équations associées à l’équilibre de chacune des pièces, établir la relation liant $F_m$ et $P_3$ à l’équilibre.   \textit{On cherchera à écrire le minimum d'équations.}}
%\subparagraph{}\textit{Indiquer quelles sont les équations qui auraient pu permettre de trouver cette relation sans écrire tout le système d’équation.}
\subparagraph{}\textit{Pour quelle(s) valeur(s) particulières de $\theta_{1/0}$ l’équilibre est-il possible avec un effort $F_m$ nul ?}
\subparagraph{}\textit{Établir les équations permettant de relier la translation $\lambda_{3/0}$ du coulisseau, la position angulaire $\theta_{(1/0)}$ et les constantes géométriques du mécanisme.}
\subparagraph{}\textit{En établissant un bilan de puissance, vérifier les relations obtenues.}

\subsection*{On suppose que les contacts en $H$ et $J$ s’effectuent avec frottement de même coefficient $f$}

\subparagraph{}\textit{On suppose que les contacts en $H$ et $J$ s’effectuent avec frottement de même coefficient $f$. Reprendre la question 2 dans le cadre de cette hypothèse. On se place dans la situation de descente de la charge.}

Distinguer deux situations, selon que $J$ est situé au-dessus ou en dessous de l’axe $\left(B,\vect{x_0}\right)$.
\subparagraph{}\textit{Définir le domaine de valeurs de $\theta_{(1/0)}$ pour lequel l’équilibre du système est possible sans exercer d’effort sur le levier \textbf{(1)} ($F_m = 0$)}.


\ifprof
\else
\end{multicols}
\fi

%\begin{center}
%\includegraphics[width=\linewidth]{images/fig_04}
%%\textit{}
%\end{center}

\end{document}

\subparagraph{}\textit{}


\begin{center}
\includegraphics[width=\linewidth]{images/fig_06}
%\textit{}
\end{center}
\begin{center}
\includegraphics[width=\linewidth]{images/img_04}
%\textit{}
\end{center}

