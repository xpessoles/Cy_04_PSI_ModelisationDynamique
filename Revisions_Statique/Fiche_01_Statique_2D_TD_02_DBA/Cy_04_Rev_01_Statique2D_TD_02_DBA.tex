\documentclass[10pt,fleqn]{article} % Default font size and left-justified equations
\usepackage[%
    pdftitle={Résolutions de problèmes de statique : PFS 2D},
    pdfauthor={Xavier Pessoles}]{hyperref}

\input{style/new_style}
\input{style/macros_SII}
\usepackage{multicol}
\usepackage{siunitx}
%\usepackage{picins}
\fichetrue
%\fichefalse

\proftrue
\proffalse

\tdtrue
%\tdfalse

\courstrue
\coursfalse

% -------------------------------------
% Déclaration des titres
% -------------------------------------

\def\discipline{Sciences \\Industrielles de \\ l'Ingénieur}
\def\xxtete{Sciences Industrielles de l'Ingénieur}


\def\classe{\textsf{PSI$\star$ -- MP}}
\def\xxnumpartie{Rév -- Stat}
\def\xxpartie{Modéliser le comportement statique des systèmes mécaniques}

\def\xxnumchapitre{Révision 1 \vspace{.2cm}}
\def\xxchapitre{\hspace{.12cm} Résolution des problèmes de statique -- Statique 2D}

\def\xxposongletx{2}
\def\xxposonglettext{1.45}
\def\xxposonglety{19}%16

\def\xxonglet{\textsf{Rév -- Stat}}

\def\xxactivite{TD 02}
\def\xxauteur{\textsl{Xavier Pessoles}}


\def\xxtitreexo{Dépose de bagage automatique dans les aéroports (DBA)}
\def\xxsourceexo{\hspace{.2cm} \footnotesize{Concours Centrale Supelec TSI 2018}}

\def\xxcompetences{%
\textsl{%
\textbf{Savoirs et compétences :}\\
}}

\def\xxfigures{
\includegraphics[width=.65\textwidth]{images/fig_00}
}%figues de la page de garde

\def\xxpied{%
Révision statique -- Résolution des problèmes de statique plane\\
Fiche 1 -- \xxactivite%
}

\setcounter{secnumdepth}{5}
%---------------------------------------------------------------------------

\usepackage{bm}
\begin{document}
%\chapterimage{png/Fond_Cin}
\input{style/new_pagegarde}
\vspace{4.5cm}
\pagestyle{fancy}
\thispagestyle{plain}


\def\columnseprulecolor{\color{ocre}}
\setlength{\columnseprule}{0.4pt} 

\ifprof
\begin{multicols}{2}
\else
\begin{multicols}{2}
\fi
\section*{Mise en situation}
\ifprof
\else
\fi

Le processus d’enregistrement des passagers dans les aéroports est en train de
vivre une mutation en évoluant de la « banque d’enregistrement » classique vers une idée de « dépose bagages »
automatisée. Cette évolution a été justifiée pour fluidifier le trafic passager notamment sur les destinations avec
des fréquences très importantes, par exemple certains vols Paris-Province.


Le système DBA est constitué par un basculeur actionné par un dispositif bielle-manivelle et une machine
asynchrone.


\begin{center}
\includegraphics[width=\linewidth]{images/fig_01}
%\textit{}
\end{center}

\subsection*{Recherche de la vitesse de rotation maximale}
\begin{obj} 
Le bagage et le chariot sont animés par un dispositif bielle-manivelle et une machine asynchrone
triphasée avec un réducteur entraînant la manivelle. L’objectif est de déterminer la vitesse de rotation
maximale de la machine asynchrone triphasée actionnant le basculeur en accord avec l’exigence 1.4
(le basculement du bagage doit se faire en \SI{8}{s}).
\end{obj}

Pour dimensionner correctement la machine asynchrone, la première étape est le calcul de la vitesse maximale
de l’arbre moteur.
On choisit comme loi de mouvement de rotation du moteur une loi en trapèze. On donne ainsi le profil de vitesse de rotation $\omega_r$ de l’arbre de sortie du réducteur par rapport au bâti.

\begin{center}
\includegraphics[width=\linewidth]{images/fig_02}
%\textit{}
\end{center}

Le rapport de réduction entre l’arbre moteur de vitesse de rotation et l’arbre de sortie de réducteur est noté $k=\dfrac{\omega_r}{\omega_{\text{mot}}} = \dfrac{1}{107,7}$.
Compte tenu du temps de basculement du bagage de \SI{8}{s}, les valeurs des temps sont les suivantes : $t_1=\SI{0,5}{s}$, $t_2=\SI{2,5}{s}$, $t_3=\SI{3}{s}$, $t_4=\SI{5}{s}$, $t_5=\SI{5,5}{s}$, $t_6=\SI{7,5}{s}$, $t_7=\SI{8}{s}$. L’arbre de sortie du motoréducteur doit faire un demi-tour entre 0 et $t_3$, puis un demi-tour entre $t_4$ et $t_7$.

\subparagraph{}\textit{Déterminer $\omega_{\text{max}}$ en fonction des différents 
$t_i$. Faire l’application numérique.}\textit{}
\ifprof
\begin{corrige}
\end{corrige}
\else
\fi
\subparagraph{}\textit{En déduire la vitesse de rotation de l’arbre moteur maximale $\omega_{\text{mot max}}$. Faire l’application numérique et donner le résultat en $\text{tr}\cdot\text{min}^{-1}$.}

\ifprof
\begin{corrige}
\end{corrige}
\else
\fi
\begin{center}

\subsection*{Recherche du couple moteur maximal en vue du dimensionnement de la machine asynchrone}

\begin{obj}
La seconde étape du dimensionnement consiste à rechercher le couple moteur maximal en accord avec
l’exigence 1.2 (la masse du bagage pouvant être manœuvré par le système est de \SI{50}{kg}).
\end{obj}

%\includegraphics[width=.8\linewidth]{images/fig_01_bis}
%\includegraphics[width=.8\linewidth]{images/hayon_parametrage}
%\textit{}
\end{center}


\vspace{-.5cm}

\subsection*{Caractéristiques géométriques du vérin}
\ifprof
\else

\footnotesize
%\vspace{.5cm}

\begin{multicols}{2}
\fbox{\noindent Éléments de corrigé}
%\begin{enumerate}
%\item Angle d'ouverture: $67,5\degres$.
%\item $L^2 =\left(-a + c\cos\theta \right)^2 + \left(b + c\sin\theta \right)^2  $.
%\item Course de \SI{13,2}{cm}.
%\item $   F_v = \dfrac{\lambda Mg\cos \theta}{c\sin \left( \alpha - \theta\right)} $ ($F_v/2$).
%\item $k=\SI{1667}{N.m^{-1}}$, écrasement de $\SI{300}{mm}$.
%\item .
%\item $\Delta F = \pm \SI{443}{N}$.
%\item $I_{\text{max}} = \SI{3,95}{A}$.
%\end{enumerate}
\end{multicols}
\fi

\normalsize


\ifprof
\end{multicols}
\else
\end{multicols}
\fi


\end{document}

\subparagraph{}\textit{}
\ifprof
\begin{corrige}
\end{corrige}
\else
\fi

\begin{center}
\includegraphics[width=\linewidth]{images/fig_04}
%\textit{}
\end{center}

