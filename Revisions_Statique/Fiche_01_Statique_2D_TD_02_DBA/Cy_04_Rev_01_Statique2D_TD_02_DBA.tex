\documentclass[10pt,fleqn]{article} % Default font size and left-justified equations
\usepackage[%
    pdftitle={Résolutions de problèmes de statique : PFS 2D},
    pdfauthor={Xavier Pessoles}]{hyperref}

\input{style/new_style}
\input{style/macros_SII}
\usepackage{multicol}
\usepackage{siunitx}
%\usepackage{picins}
\fichetrue
%\fichefalse

\proftrue
\proffalse

\tdtrue
%\tdfalse

\courstrue
\coursfalse

% -------------------------------------
% Déclaration des titres
% -------------------------------------

\def\discipline{Sciences \\Industrielles de \\ l'Ingénieur}
\def\xxtete{Sciences Industrielles de l'Ingénieur}


\def\classe{\textsf{PSI$\star$ -- MP}}
\def\xxnumpartie{Rév -- Stat}
\def\xxpartie{Modéliser le comportement statique des systèmes mécaniques}

\def\xxnumchapitre{Révision 1 \vspace{.2cm}}
\def\xxchapitre{\hspace{.12cm} Résolution des problèmes de statique -- Statique 2D}

\def\xxposongletx{2}
\def\xxposonglettext{1.45}
\def\xxposonglety{19}%16

\def\xxonglet{\textsf{Rév -- Stat}}

\def\xxactivite{TD 02}
\def\xxauteur{\textsl{Xavier Pessoles}}


\def\xxtitreexo{Dépose de bagage automatique dans les aéroports (DBA)}
\def\xxsourceexo{\hspace{.2cm} \footnotesize{Concours Centrale Supelec TSI 2018}}

\def\xxcompetences{%
\textsl{%
\textbf{Savoirs et compétences :}\\
}}

\def\xxfigures{
\includegraphics[width=.65\textwidth]{images/fig_00}
}%figues de la page de garde

\def\xxpied{%
Révision statique -- Résolution des problèmes de statique plane\\
Fiche 1 -- \xxactivite%
}

\setcounter{secnumdepth}{5}
%---------------------------------------------------------------------------

\usepackage{bm}
\begin{document}
%\chapterimage{png/Fond_Cin}
\input{style/new_pagegarde}
\vspace{4.5cm}
\pagestyle{fancy}
\thispagestyle{plain}


\def\columnseprulecolor{\color{ocre}}
\setlength{\columnseprule}{0.4pt} 

\ifprof
\begin{multicols}{2}
\else
\begin{multicols}{2}
\fi
\section*{Mise en situation}
\ifprof
\else
\fi

Le processus d’enregistrement des passagers dans les aéroports est en train de
vivre une mutation en évoluant de la « banque d’enregistrement » classique vers une idée de « dépose bagages »
automatisée. Cette évolution a été justifiée pour fluidifier le trafic passager notamment sur les destinations avec
des fréquences très importantes, par exemple certains vols Paris-Province.


Le système DBA est constitué par un basculeur actionné par un dispositif bielle-manivelle et une machine
asynchrone (figures 2 et 3).

\begin{obj}~\\
\vspace{-.3cm}
\begin{itemize}
\item ****
\end{itemize}
\end{obj}



\vspace{-.5cm}

\begin{center}
\includegraphics[width=.8\linewidth]{images/fig_01_bis}
%\includegraphics[width=.8\linewidth]{images/hayon_parametrage}
%\textit{}
\end{center}


\vspace{-.5cm}

\subsection*{Caractéristiques géométriques du vérin}
\ifprof
\else

\footnotesize
%\vspace{.5cm}

\begin{multicols}{2}
\fbox{
\noindent Éléments de corrigé}
%\begin{enumerate}
%\item Angle d'ouverture: $67,5\degres$.
%\item $L^2 =\left(-a + c\cos\theta \right)^2 + \left(b + c\sin\theta \right)^2  $.
%\item Course de \SI{13,2}{cm}.
%\item $   F_v = \dfrac{\lambda Mg\cos \theta}{c\sin \left( \alpha - \theta\right)} $ ($F_v/2$).
%\item $k=\SI{1667}{N.m^{-1}}$, écrasement de $\SI{300}{mm}$.
%\item .
%\item $\Delta F = \pm \SI{443}{N}$.
%\item $I_{\text{max}} = \SI{3,95}{A}$.
%\end{enumerate}
\end{multicols}
\fi

\normalsize


\ifprof
\end{multicols}
\else
\end{multicols}
\fi


\end{document}

\subparagraph{}\textit{}

\begin{center}
\includegraphics[width=\linewidth]{images/img_04}
%\textit{}
\end{center}

