%%%% Paramétrage du TD %%%%
\def\xxactivite{Révisions \ifprof -- Corrigé \else \fi} % \normalsize \vspace{-.4cm}
\def\xxauteur{\textsl{Xavier Pessoles}}
\def\xxnumchapitre{ Révisions 2\vspace{.2cm}}
\def\xxchapitre{\hspace{.12cm} Modélisation du frottement}
\def\xxonglet{\textsf{Rév -- Stat}}
\def\xxactivite{\ifcolle Colle \else TD 01\fi }
\def\xxauteur{\textsl{Xavier Pessoles}}

\def\xxpied{%
Révision statique -- Frottement sec\\
Fiche 1 -- \xxactivite%
}

\def\xxtitreexo{Modèle du frottement exponentiel -- Poulie-courroie  \ifnormal $\star$ \else \fi \ifdifficile $\star\star$ \else \fi \iftdifficile $\star\star\star$ \else \fi}
\def\xxsourceexo{\hspace{.2cm} \footnotesize{Lycée Mistral Avignon}}

\def\xxcompetences{%
\textsl{%
\textbf{Savoirs et compétences :}\\} \vspace{-.5cm}
\begin{itemize}
\item \textit{Res2.C18} : principe fondamental de la statique:
\item \textit{Res2.C19} : équilibre d’un solide, d’un ensemble de solides;
\item \textit{Res2.C20} : théorème des actions réciproques.
\end{itemize}
}

\def\xxauteur{\textsl{Xavier Pessoles}}


\def\xxfigures{
%\includegraphics[width=.5\textwidth]{fig_00}
}%figues de la page de garde


\iflivret
\input{../../style/new_pagegarde}
\else
\input{../../style/new_pagegarde}
\fi
\setlength{\columnseprule}{.1pt}

\pagestyle{fancy}
\thispagestyle{plain}

\vspace{5cm}

\def\columnseprulecolor{\color{ocre}}
\setlength{\columnseprule}{0.4pt} 

\setcounter{exo}{0}

%%%%%%%%%%%%%%%%%%%%%%%%%%%%%%%%%%%%%%%%%%%%%%%%%%

\ifprof
\begin{multicols}{2}
\else
\begin{multicols}{2}
\fi
\section*{Mise en situation}
\ifprof
\else
\fi

Le problème du frottement d'une corde, d'une sangle ou d'une courroie sur une poulie ou
un tambour est un problème classique. 
\begin{obj}
Modéliser l'évolution de la tension dans un câble en fonction de l'angle d'enroulement sur une poulie. 
\end{obj}

On note $f$ le coefficient de frottement entre le câble et la poulie. 
\begin{center}
\includegraphics[width=\linewidth]{fig_01}
%\textit{}
\end{center}

On considère que le câble est enroulé d'un angle $\alpha$ autour de la poulie. Le câble est à la limite du glissement sous l'action des deux brins $\vect{T_1}$ et $\vect{T_2}$. Soit $M(\theta)$ un point de l'enroulement. 

\subparagraph{}\textit{Après avoir isolé une tranche élémentaire de câble en $M(\theta)$ de largeur $\dd \theta$, réaliser un bilan des actions mécaniques extérieures. }
\ifprof
\begin{corrige}
BAME :
\begin{itemize}
\item action de tension du câble 1 $\vect{F}\left(\theta+\dfrac{\dd \theta}{2}\right)$;
\item action de tension du câble 2 $\vect{F}\left(\theta-\dfrac{\dd \theta}{2}\right)$;
\item action de la poulie sur le câble : $N\vect{u_r}+T\vect{u_{\theta}}$ avec $T=\pm fN$.
\end{itemize}

\end{corrige}
\else
\fi


\subparagraph{}\textit{Appliquer le théorème en résultante statique en projection dans la base $\left( \vect{u_r}, \vect{u_{\theta}}\right)$.}
\ifprof
\begin{corrige}
L'application du TRS à la tranche de câble, on a $\vect{F}\left(\theta+\dfrac{\dd \theta}{2}\right) + \vect{F}\left(\theta-\dfrac{\dd \theta}{2}\right) + N\vect{u_r}+T\vect{u_{\theta}} = \vect{0}$.

En projetant dans $\left( \vect{u_r}, \vect{u_{\theta}}\right)$ on a :
 
$\left\{ 
\begin{array}{l}
-F\left(\theta+\dfrac{\dd \theta}{2}\right)  \sin\left(\dfrac{\dd \theta}{2}\right)  -F\left(\theta-\dfrac{\dd \theta}{2}\right) \sin\left(\dfrac{\dd \theta}{2}\right) + N= 0 \\
F\left(\theta+\dfrac{\dd \theta}{2}\right) \cos\left(\dfrac{\dd \theta}{2}\right) -F\left(\theta-\dfrac{\dd \theta}{2}\right) \cos\left(\dfrac{\dd \theta}{2}\right) + T = 0
\end{array}
\right.$.

\end{corrige}
\else
\fi


\subparagraph{}\textit{En considérant que l'angle $\theta$ est petit, établir l'équation différentielle liant $f$ et $F(\theta)$ et $\theta$.}
\ifprof
\begin{corrige} ~\\
En utilisant $\cos \dd \theta/2 \simeq 1$ et $\sin \dd \theta/2 \simeq \dd \theta/2$ :
$\left\{ 
\begin{array}{l}
-F\left(\theta+\dfrac{\dd \theta}{2}\right)\dfrac{\dd \theta}{2} -F\left(\theta-\dfrac{\dd \theta}{2}\right) \dfrac{\dd \theta}{2} + N= 0 \\
F\left(\theta+\dfrac{\dd \theta}{2}\right) -F\left(\theta-\dfrac{\dd \theta}{2}\right) + T = 0
\end{array}
\right.$.


$
\Leftrightarrow \left\{ 
\begin{array}{l}
-\left(F\left(\theta+\dfrac{\dd \theta}{2}\right) +F\left(\theta-\dfrac{\dd \theta}{2}\right)\right) \dfrac{\dd \theta}{2} + N= 0 \\
F\left(\theta+\dfrac{\dd \theta}{2}\right) -F\left(\theta-\dfrac{\dd \theta}{2}\right) + T = 0
\end{array}
\right.$.

De plus, en faisant un DL à l'ordre 2, $F\left(\theta+\dfrac{\dd \theta}{2}\right)\simeq  F(\theta)+\dfrac{\dd \theta}{2} \dfrac{\dd F(\theta)}{\dd \theta}$.
On a donc : 
$
\left\{ 
\begin{array}{l}
-2 F\left(\theta\right) \dfrac{\dd \theta}{2} + N= 0 \\
 \dd F(\theta)+ T = 0
\end{array}
\right.$.

En utilisant le modèle de Coulomb, $T=\pm fN$

 $ \dd F(\theta)\pm f \left( 2 F\left(\theta\right) \dfrac{\dd \theta}{2}\right) = 0$
 $  \Leftrightarrow \dd F(\theta)\pm f  F\left(\theta\right) \dd \theta = 0$

%
%En utilisant le modèle de Coulomb, on a $T=fN$ et 
%
%$F\left(\theta+\dfrac{\dd \theta}{2}\right) -F\left(\theta-\dfrac{\dd \theta}{2}\right)  = -f \left(F\left(\theta+\dfrac{\dd \theta}{2}\right)\dfrac{\dd \theta}{2} +F\left(\theta-\dfrac{\dd \theta}{2}\right) \dfrac{\dd \theta}{2} \right)$.
%
%
%$ \Leftrightarrow F\left(\theta+\dfrac{\dd \theta}{2}\right) -F\left(\theta-\dfrac{\dd \theta}{2}\right)  + f \left(F\left(\theta+\dfrac{\dd \theta}{2}\right)\dfrac{\dd \theta}{2} +F\left(\theta-\dfrac{\dd \theta}{2}\right) \dfrac{\dd \theta}{2} \right) = 0$.
%
%$ \Leftrightarrow F\left(\theta+\dfrac{\dd \theta}{2}\right) \left(1 +f\dfrac{\dd \theta}{2} \right) -F\left(\theta-\dfrac{\dd \theta}{2}\right) \left(1+ f\dfrac{\dd \theta}{2} \right) = 0$.
%
%
%$ \Leftrightarrow  \left(1 +f\dfrac{\dd \theta}{2} \right)\left(F\left(\theta+\dfrac{\dd \theta}{2}\right) - F\left(\theta-\dfrac{\dd \theta}{2}\right) \right)  = 0$.
%
%En réutilisant l'hypothèse des petits angles, on a : 
%$ \Leftrightarrow  \left(1 +f\dfrac{\dd \theta}{2} \right)\left(F\left(\theta\right) - F\left(\theta\right) \right)  = 0$.

\end{corrige}
\else
\fi

\subparagraph{}\textit{Résoudre l'équation différentielle pour établir la relation entre $T_1$, $T_2$, $f$ et $\alpha$.}
\ifprof
\begin{corrige}
On a :
 $ \dd F(\theta)= \pm  F\left(\theta\right) \dd \theta $
  $ \Leftrightarrow \dfrac{ \dd F(\theta)}{ F\left(\theta\right)}= \pm \dd \theta $
  
  En intégrant l'équation précédente, on a :
  $\left[ \ln F \right] _{T_1}^{T_2} =\pm f \left[ \theta \right] _{0}^{\alpha}$
  Soit $\ln T_2 - \ln T_1 = -f\alpha$ $\Leftrightarrow \ln \dfrac{T_2}{T_1} = \pm f\alpha$ et $T_2 = T_1 \text{e}^{\pm f\alpha}$.
  
(Le signe dépend du sens de glissement. )
\end{corrige}
\else
\fi

\ifprof
\end{multicols}
\else
\end{multicols}
\fi
