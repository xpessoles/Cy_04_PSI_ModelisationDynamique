\documentclass[10pt,fleqn]{article} % Default font size and left-justified equations
\usepackage[%
    pdftitle={Résolutions de problèmes de statique : PFS 2D},
    pdfauthor={Xavier Pessoles}]{hyperref}

\input{style/new_style}
\input{style/macros_SII}
\usepackage{multicol}
\usepackage{siunitx}
%\usepackage{picins}
\fichetrue
%\fichefalse

\proftrue
%\proffalse

\tdtrue
%\tdfalse

\courstrue
\coursfalse

% -------------------------------------
% Déclaration des titres
% -------------------------------------

\def\discipline{Sciences \\Industrielles de \\ l'Ingénieur}
\def\xxtete{Sciences Industrielles de l'Ingénieur}


\def\classe{\textsf{PSI$\star$ -- MP}}
\def\xxnumpartie{Rév -- Stat}
\def\xxpartie{Modéliser le comportement statique des systèmes mécaniques}

\def\xxnumchapitre{Révision 1 \vspace{.2cm}}
\def\xxchapitre{\hspace{.12cm} Résolution des problèmes de statique -- Statique 2D}

\def\xxposongletx{2}
\def\xxposonglettext{1.45}
\def\xxposonglety{19}%16

\def\xxonglet{\textsf{Rév -- Stat}}

\def\xxactivite{TD 01}
\def\xxauteur{\textsl{Xavier Pessoles}}


\def\xxtitreexo{Modélisation d'un hayon de coffre électrique}
\def\xxsourceexo{\hspace{.2cm} \footnotesize{Concours Centrale Supelec TSI 2013}}

\def\xxcompetences{%
\textsl{%
\textbf{Savoirs et compétences :}\\
}}

\def\xxfigures{
\includegraphics[width=.55\textwidth]{images/fig_00}
}%figues de la page de garde

\def\xxpied{%
Révision statique -- Résolution des problèmes de statique plane\\
Fiche 1 -- \xxactivite%
}

\setcounter{secnumdepth}{5}
%---------------------------------------------------------------------------


\begin{document}
%\chapterimage{png/Fond_Cin}
\input{style/new_pagegarde}
\vspace{4.5cm}
\pagestyle{fancy}
\thispagestyle{plain}


\def\columnseprulecolor{\color{ocre}}
\setlength{\columnseprule}{0.4pt} 

\ifprof
\begin{multicols}{2}
\else
\begin{multicols}{2}
\fi
\section*{Mise en situation}
Le PCS (Power Closure System), conçu par Valéo, est un système d’ouverture et de fermeture automatique de
hayon de coffre automobile.
Le système étant symétrique, les deux vérins sont ramenées dans 
le plan d’évolution de la porte de coffre et leur action mécanique s’exerçant sur la porte de coffre est supposée identique.

Le repère $\repere{B}{x_t}{y_t}{z_0}$ est lié à la Terre. $\vect{y_t}$ est un vecteur unitaire vertical tel que l’accélération de la pesanteur s’écrit $\vect{g}=-g\vect{y_t}$ avec $g=\SI{9,81}{m.s^{-2}}$. $\vect{x_t}$ est un vecteur unitaire horizontal. La liaison pivot entre
la structure du véhicule et la porte de coffre est d’axe $\axe{B}{z_0}$.

Le repère $\repere{B}{x_p}{y_p}{z_0}$ est lié à la porte de coffre de masse $M=\SI{30}{kg}$. Le repère $\repere{B}{x_v}{y_v}{z_0}$ est lié au corps du vérin. La sortie
de tige par rapport au corps du vérin se fait dans la direction du vecteur $\vect{x_v}$.
Les liaisons entre la tige du vérin et le bâti d’une part et entre le corps du vérin et la porte de coffre d’autre part, sont des liaisons rotules de centres respectifs $A$ et $C$.
Le point $D$ représente l’extrémité de la porte du coffre. La hauteur du point $D$ par rapport au sol suivant la
verticale est de \SI{0,7}{m} en position coffre fermé et de \SI{1,8}{m} en position coffre ouvert.

\begin{obj}
Déterminer les caractéristiques du vérin à choisir pour répondre au cahier des charges : longueur
du vérin en position coffre ouvert et coffre fermé, course du vérin, raideur du ressort équipant le
vérin, couple moteur maximal nécessaire pour le maintien en position du hayon.
\end{obj}

\begin{center}
\includegraphics[width=\linewidth]{images/fig_01}
%\textit{}
\end{center}

\subsection*{Caractéristiques géométriques du vérin}
Le centre d’inertie du coffre est situé en $G$ tel que $\vect{BG}=\lambda\vect{x_p}$ avec $\lambda=\SI{0,6}{m}$.

$\vect{AB}=-a\vect{x_0}+b\vect{y_0}$, $\vect{AC}=L\vect{x_v}$, $\vect{BC}=c\vect{x_p}$, $\vect{BD}=d\vect{x_p}$ avec $a=\SI{0,55}{m}$, $b=\SI{0,14}{m}$, $c=\SI{0,14}{m}$  et $d=\SI{1}{m}$. L’angle formé entre $\vect{x_0}$ et l’horizontale $\vect{x_t}$ est $\theta_0 = 42\degres$.

\subparagraph{}
\textit{Déterminer l’angle d’ouverture maximal.}
\ifprof
\begin{corrige}~\\
\begin{center}
\includegraphics[width=\linewidth]{images/cor_01}
%\textit{}
\end{center}
D'une part, $x = d\sin 42 \simeq \SI{0,67}{m}$. D'autre part, $\sin\alpha = \dfrac{1,8-0,7-x}{d}=0,43$. Au final $\alpha = 25,5\degres$. 

L'angle d'ouverture est donc de $67,5\degres$.

\end{corrige}
\else
\fi

%Par l’écriture de la fermeture géométrique dans le triangle 𝐴𝐵𝐶, déterminer la longueur du vérin 𝐿 en fonction de l’angle d’ouverture du coffre 𝜃.

\subparagraph{}
\textit{Déterminer la longueur du vérin $L$ en fonction de l’angle d’ouverture du coffre $\theta$.}
\ifprof
\begin{corrige}~\\
La longueur du vérin est donnée par la valeur de $L$. En réalisant la fermeture géométrique, on a $\vect{AB}+\vect{BC}+\vect{CA}=\vect{0} \Leftrightarrow 
-a\vect{x_0}+b\vect{y_0} + c\vect{x_p} -L\vect{x_v} =\vect{0}$.

En projetant l'équation vectorielle dans $\rep{0}$, on a : 
$$
\left\{ 
\begin{array}{l}
-a + c\cos\theta -L\cos\alpha ={0} \\
b + c\sin\theta -L\sin\alpha ={0}
\end{array}
\right.
$$
On a donc $L^2 =\left(-a + c\cos\theta \right)^2 + \left(b + c\sin\theta \right)^2  $.
%= a^2 +c^2 \cos ^2 \theta - 2 ac \cos \theta + b^2 + c^2\sin^2\theta + 2b c\sin\theta
%= a^2 +c^2 - 2 ac \cos \theta + b^2 +  2b c\sin\theta$ 

\end{corrige}
\else
\fi

On donne la courbe donnant l'évolution de la course du vérin en fonction de l'ouverture du hayon. 
\begin{center}
\includegraphics[width=\linewidth]{images/cor_02}
%\textit{}
\end{center}


\subparagraph{}
\textit{Déterminer les valeurs extrêmes de $L$, ainsi que la course du vérin.}
\ifprof
\begin{corrige}~\\
La longueur du vérin varie de \SI{43,3}{cm} à \SI{56,5}{cm} soit une course de \SI{13,2}{cm}. 
\end{corrige}
\else
\fi


\subsection*{Vérification de critère du cahier des charges de la fonction ***FC1 et détermination de la
raideur du ressort correspondante}

Les vérins utilisés sont constitués d’un moteur à courant continu, d’un réducteur à engrenage, d’une vis à billes et d’un ressort.

\begin{center}
\includegraphics[width=\linewidth]{images/fig_02}
%\textit{}
\end{center}

Le cahier des charges impose que la porte soit en équilibre sur une large plage d’ouverture en cas de panne
moteur. C’est pourquoi il faut déterminer la raideur et la longueur à vide des ressorts qui assurent cette fonction
dans les vérins électriques.

On suppose dans un premier temps que le coffre est à l’équilibre.

\subparagraph{}
\textit{Déterminer l’effort $F$ exercé par chacun des vérins sur la porte de coffre en fonction de $\theta$, $\alpha$ et des constantes du problème.}
\ifprof
\begin{corrige}~\\
On isole le corps et le piston du vérin. L'ensemble est soumis à deux actions mécaniques (liaisons sphériques en $A$ et $C$). D'après le PFS, cette action mécanique est donc suivant Ces deux actions mécaniques sont donc de même direction (le vecteur $\vect{x_v}$), de même norme et de sens opposé. 

On isole le hayon $h$. 

On réalise le BAME : 
\begin{itemize}
\item action mécanique du vérin $v$: $\torseurstat{T}{v}{h}=\torseurl{F_v\vect{x_v}}{\vect{0}}{C}$;
\item action de la pesanteur : $\torseurstat{T}{\text{pes}}{h}=\torseurl{-Mg\vect{y_t}}{\vect{0}}{G}$;
\item action de la pivot en $B$ : $\torseurstat{T}{\text{0}}{h}$.
\end{itemize}

On cherche à connaître l'action du vérin en fonction des actions de pesanteur. On réalise donc le théorème du moment statique en $B$ en projection sur $\vect{z_0}$ : 

$ \left( 
\vect{0} + \vect{BC}\wedge F_v\vect{x_v}
+ \vect{0} + \vect{BG}\wedge -Mg\vect{y_t}
\right) \cdot \vect {z_0} = \vect{0} $
$\Rrightarrow \left( c\vect{x_p}\wedge F_v\vect{x_v} + \lambda \vect{x_p}\wedge -Mg\vect{y_t}
\right) \cdot \vect {z_0} = \vect{0} $

\begin{center}
\includegraphics[width=.5\linewidth]{images/cor_03}
%\textit{}

$\Leftrightarrow  c F_v\sin \left( \alpha - \theta\right) - \lambda Mg\cos \theta  = {0} $

$   F_v = \dfrac{\lambda Mg\cos \theta}{c\sin \left( \alpha - \theta\right)} $.

Dans le cas où on considère les deux vérins, on aura $F_1=F_2=F_v/2$.

\end{center}


\end{corrige}


\else
\fi

En exploitant les équations obtenues à partir de l’écriture de la fermeture géométrique obtenue précédemment, on montre que la relation entre $\theta$ et $\alpha$ s’écrit : 
$ \tan \alpha = \dfrac{b+c\sin\theta}{-a+c\cos\theta}$.

On déduit de la question précédente le tracé de l’évolution de l’effort $F$ nécessaire au maintien en équilibre du coffre en fonction de la longueur $L$ du vérin.


On choisit d’utiliser un ressort précontraint au sein du vérin (voir figure suivante) de manière à assister l’ouverture du coffre et à assurer l’équilibre du coffre sur une plage de fonctionnement maximale. On estime que les forces de frottement maximales au sein du vérin (essentiellement dues à la friction dans la vis) sont de l’ordre de $F_{\text{frot}}=\SI{100}{N}$. 



La figure suivante représente la force que doit exercer le vérin sur la porte de coffre pour assurer
l’équilibre de cette dernière en fonction de la longueur du vérin. Les courbes en pointillés représentent la force du vérin $\pm\SI{100}{N}$.


\begin{center}
\includegraphics[width=\linewidth]{images/fig_03}
%\textit{}
\end{center}



\subparagraph{}
\textit{Déterminer la raideur $k$ du ressort et sa longueur à vide $L_0$ de manière à obtenir une situation d’équilibre
sur la plus grande plage de fonctionnement. Préciser votre démarche.}
\ifprof
\begin{corrige}~\\

\end{corrige}
\else
\fi

La figure suivante représente l’évolution du couple moteur dans un vérin lors des phases d’ouverture et de fermeture
du coffre.


\begin{center}
\includegraphics[width=\linewidth]{images/fig_04}
%\textit{}
\end{center}

\subparagraph{}
\textit{Déterminer le couple moteur maximal en phase d’ouverture puis en phase de fermeture.}
\ifprof
\begin{corrige}~\\

\end{corrige}
\else
\fi

\ifprof
\end{multicols}
\else
\end{multicols}
\fi

\end{document}

\subparagraph{}\textit{}

\begin{center}
\includegraphics[width=\linewidth]{images/img_04}
%\textit{}
\end{center}

