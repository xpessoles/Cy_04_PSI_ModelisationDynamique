%%%% Paramétrage du TD %%%%
\def\xxactivite{Révisions \ifprof -- Corrigé \else \fi} % \normalsize \vspace{-.4cm}
\def\xxauteur{\textsl{Xavier Pessoles}}


\def\xxnumchapitre{Révision 1 \vspace{.2cm}}
\def\xxchapitre{\hspace{.12cm} Résolution des problèmes de statique -- Statique 2D}
\def\xxonglet{\textsf{Rév -- Stat}}
\def\xxactivite{TD 03}
\def\xxauteur{\textsl{Xavier Pessoles}}

\def\xxpied{%
Révision statique -- Résolution des problèmes de statique plane\\
Fiche 1 -- \xxactivite%
}

\def\xxtitreexo{Interface maître et esclave d'un robot  \ifnormal $\star$ \else \fi \ifdifficile $\star\star$ \else \fi \iftdifficile $\star\star\star$ \else \fi}


\def\xxsourceexo{\hspace{.2cm} \footnotesize{CCP PSI 2015}}


\def\xxcompetences{%
\textsl{%
\textbf{Savoirs et compétences :}\\} \vspace{-.5cm}
\begin{itemize}
\item \textit{Res2.C18} : principe fondamental de la statique;
\item \textit{Res2.C19} : équilibre d’un solide, d’un ensemble de solides;
\item \textit{Res2.C20} : théorème des actions réciproques.
\end{itemize}
}


\def\xxfigures{
\includegraphics[width=.5\textwidth]{fig_00}
}%figues de la page de garde




\iflivret
\input{../../style/new_pagegarde}
\else
\input{../../style/new_pagegarde}
\fi
\setlength{\columnseprule}{.1pt}

\pagestyle{fancy}
\thispagestyle{plain}

\ifprof
\vspace{5cm}
\else
\vspace{5cm}
\fi

\def\columnseprulecolor{\color{ocre}}
\setlength{\columnseprule}{0.4pt} 

%%%%%%%%%%%%%%%%%%%%%%%

\setcounter{exo}{0}



%\ifprof
%\else
\begin{multicols}{2}
%\fi


\section*{Mise en situation}
\ifprof
\else
La téléopération consiste à mettre en relation deux manipulateurs appelés communément
maître et esclave. Le manipulateur maître permet au chirurgien de donner sa consigne de
déplacement à l’aide d’un levier de commande tandis que l’esclave l’exécute au contact de
l’environnement (l’organe à opérer). Les deux sous-systèmes échangent des informations de
déplacement et d’effort au travers d’un ou plusieurs canaux de communication. Un retour
visuel est également mis en place en parallèle à ce dispositif.

\begin{center}
\includegraphics[width=\linewidth]{fig_00a}
%\textit{}
\end{center}
\fi

\section*{Modélisation de l’interface maître}
\ifprof
\else

Ce mécanisme est constitué de 4 barres reliées par des liaisons pivots.

\begin{center}
\includegraphics[width=\linewidth]{fig_01}
%\textit{}
\end{center}

\begin{obj}
Vérifier que l’exigence « Linéarité couple/effort » (id 1.3.2.2) peut être satisfaite
par le mécanisme de HOEKEN.
\end{obj}


\begin{center}
\includegraphics[width=\linewidth]{fig_02}
%\textit{}
\end{center}

\begin{itemize}
\item Solide $S_0$, repère $\rep{0}\repere{A}{x_0}{y_0}{z_0}$, $\vect{AB}=L_0\vect{x_0}$ avec $L_0 = \SI{50}{mm}$.
\item Solide $S_1$, repère $\rep{1}\repere{B}{x_1}{y_1}{z_0}$, $\vect{BC}=L_1\vect{x_1}$ avec $L_1 = \SI{25}{mm}$, $\theta_1=\angl{x_0}{x_1}=\angl{y_0}{y_1}$.
\item Solide $S_2$, repère $\rep{2}\repere{A}{x_2}{y_2}{z_0}$, $\vect{AD}=L_2\vect{x_2}$ avec $L_2 = \SI{62,5}{mm}$, $\theta_2=\angl{x_0}{x_2}=\angl{y_0}{y_2}$.
\item Solide $S_3$, repère $\rep{3}\repere{C}{x_3}{y_3}{z_0}$, $\vect{ED}=\vect{DC}=L_2\vect{x_3}$ avec  $\theta_3=\angl{x_0}{x_3}=\angl{y_0}{y_3}$.
\end{itemize}


\begin{itemize}
\item On notera $\torseurstat{T}{S_i}{S_j}=\torseurcol{X_{ij}}{Y_{ij}}{Z_{ij}}{L_{ij}}{M_{ij}}{N_{ij}}{P,\mathcal{B}_0}$ l'expression l’expression au point $P$, en projection dans la
base $\mathcal{B}_0$, du torseur de l’action mécanique exercée par le solide $S_i$ sur le solide $S_j$ ; toutes
les inconnues seront exprimées dans la base $\mathcal{B}_0$.
\item L’action mécanique exercée par le moteur sur $S_1$ sera modélisée par un couple $C_m(t) \vect{z_0}$.
\item L’action mécanique exercée par l’opérateur sur $S_3$ sera modélisée par une force $F(t) \vect{x_0}$
appliquée au point $E$.
\item L’accélération de la pesanteur sera représentée par le vecteur $\vect{g}=-g\vect{z_0}$.
\item Les inerties des solides en mouvement et les frottements dans les guidages seront négligés.
\end{itemize}
\fi

\subparagraph{}
\textit{Réaliser le graphe d'analyse du mécanisme (liaisons et efforts).}
\ifprof
\begin{corrige}~\\
\end{corrige}
\else
\fi


\ifnormal 

\subparagraph{}\textbf{\#CCINP}
\textit{Déterminer les équations algébriques issues du développement des 4 relations suivantes :
\begin{itemize}
\item théorème du moment statique en $B$ appliqué à l’équilibre de $S_1$, en projection sur $\vect{z_0}$;
\item théorème du moment statique en $A$ appliqué à l’équilibre de $S_2$, en projection sur $\vect{z_0}$;
\item théorème du moment statique en $D$ appliqué à l’équilibre de $S_3$, en projection sur $\vect{z_0}$;
\item théorème de la résultante statique appliqué à l’équilibre de $S_3$, en projection sur $\vect{y_2}$.
\end{itemize}
Montrer que 
$$C_m = \dfrac{L_1 F}{\sin \left(\theta_2 - \theta_3\right)}\left( \sin \theta_1 \sin\left( \theta_2+\theta_3 \right)  -2\cos \theta_1 \sin\theta_2\sin\theta_3\right).$$}
\ifprof
\begin{corrige}~\\
\end{corrige}
\else
\fi
\else
\fi


\ifdifficile


\subparagraph{}\textbf{\#CCMP}
\textit{Proposer une démarche permettant d'exprimer le couple moteur en fonction de l'effort de l'opérateur et des parmètres géométriques.}
\ifprof
\begin{corrige}~\\
\end{corrige}
\else
\fi
\else
\fi


\ifdifficile
\subparagraph{}\textbf{\#CCMP}
\textit{Mettre en \oe{}uvre cette démarche et montrer que
$$C_m = \dfrac{L_1 F}{\sin \left(\theta_2 - \theta_3\right)}\left( \sin \theta_1 \sin\left( \theta_2+\theta_3 \right)  -2\cos \theta_1 \sin\theta_2\sin\theta_3\right).$$}
\ifprof
\begin{corrige}~\\
\end{corrige}
\else
\fi
\else
\fi

Cette relation n’étant pas linéaire, on propose d’analyser les résultats d’une
simulation numérique en traçant le couple moteur/effort opérateur en fonction de l’abscisse du point $E$
Q6.

\begin{center}
\includegraphics[width=.8\linewidth]{fig_03}
%\textit{}
\end{center}

\begin{center}
\includegraphics[width=.7\linewidth]{fig_04}
%\textit{}
\end{center}


\subparagraph{}
\textit{Retrouver ces graphes en utilsant Python. J'ai pas essayé, mais si eux ont réussi, pourquoi pas vous ? Il faut peut-être utiliser le prmier devoir de vacances.}
\ifprof
\begin{corrige}~\\
\end{corrige}
\else
\fi


\subparagraph{}
\textit{Déterminer, à partir de la figure précédente, sur quel intervalle de l’abscisse $X_E$ l’exigence « Linéarité
couple/effort » (id 1.3.2.2) est satisfaite. Indiquer si cet intervalle est compatible avec les
exigences précédemment vérifiées.}
\ifprof
\begin{corrige}~\\
\end{corrige}
\else
\fi

\end{multicols}
