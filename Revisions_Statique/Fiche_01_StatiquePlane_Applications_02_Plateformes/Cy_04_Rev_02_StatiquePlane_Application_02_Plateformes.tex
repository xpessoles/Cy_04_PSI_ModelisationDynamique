\documentclass[10pt,fleqn]{article} % Default font size and left-justified equations
\usepackage[%
    pdftitle={Résolutions de problèmes de statique : PFS 2D},
    pdfauthor={Xavier Pessoles}]{hyperref}

\input{style/new_style}
\input{style/macros_SII}
\usepackage{multicol}
\usepackage{siunitx}
%\usepackage{picins}
\fichetrue
%\fichefalse

\proftrue
\proffalse

\tdtrue
%\tdfalse

\courstrue
\coursfalse

% -------------------------------------
% Déclaration des titres
% -------------------------------------

\def\discipline{Sciences \\Industrielles de \\ l'Ingénieur}
\def\xxtete{Sciences Industrielles de l'Ingénieur}


\def\classe{\textsf{PSI$\star$ -- MP}}
\def\xxnumpartie{Rév -- Stat}
\def\xxpartie{Modéliser le comportement statique des systèmes mécaniques}

\def\xxnumchapitre{Révision 1 \vspace{.2cm}}
\def\xxchapitre{\hspace{.12cm} Résolution des problèmes de statique -- Statique plane}

\def\xxposongletx{2}
\def\xxposonglettext{1.45}
\def\xxposonglety{19}%16

\def\xxonglet{\textsf{Rév -- Cin}}

\def\xxactivite{Application 01}
\def\xxauteur{\textsl{Équipe PT -- PT$\star$ La Martinière Monplaisir}}


\def\xxtitreexo{Toitures}
\def\xxsourceexo{\hspace{.2cm} \footnotesize{Équipe PT -- PT$\star$ La Martinière Monplaisir}}

\def\xxcompetences{%
\textsl{%
\textbf{Savoirs et compétences :}\\
}}

\def\xxfigures{
%\includegraphics[width=.8\textwidth]{images/fig_01}
}%figues de la page de garde

\def\xxpied{%
Révision statique -- Résolution des problèmes de statique plane\\
Fiche 1 -- \xxactivite%
}

\setcounter{secnumdepth}{5}
%---------------------------------------------------------------------------


\begin{document}
%\chapterimage{png/Fond_Cin}
\input{style/new_pagegarde}
\vspace{4cm}
\pagestyle{fancy}
\thispagestyle{plain}


\def\columnseprulecolor{\color{ocre}}
\setlength{\columnseprule}{0.4pt} 

\ifprof
\else
\begin{multicols}{2}
\fi
\section*{Toit de stade}
Un toit de stade représenté sur la figure suivante est sollicité par son propre poids  avec $q=\SI{-5}{kN/m}$.  On donne $a=\SI{4}{m}$, $b=\SI{5}{m}$. Les liaisons sont supposées être des liaisons pivot parfaites.
\begin{center}
\includegraphics[width=.9\linewidth]{images/fig_01}
\end{center}

\begin{center}
\includegraphics[width=.9\linewidth]{images/fig_02}
\end{center}

\subparagraph{}\textit{Tracer le graphe de structure. Définir le nombre d'inconnues statiques.}

\subparagraph{}\textit{Donner la stratégie permettant de déterminer toutes les inconnues de liaisons.}

\subparagraph{}\textit{Déterminer les actions mécaniques dans toutes les liaisons.}


\section*{Toit en béton armé}
\setcounter{exo}{0}
\begin{center}
\includegraphics[width=.9\linewidth]{images/fig_03}
\end{center}

Les entrées et sorties d’un garage souterrain sont couvertes par un avant toit en béton armé $[ABC]$. Ce toit est en appui en $A$ sur un massif et en $B$ sur un mur incliné $[BD]$. Le toit est soumis à une charge uniformément répartie $p$. On donne :
$a=\SI{5}{m}$; $b=\SI{2}{m}$; $c=\SI{3}{m}$ et $p=\SI{-3}{kN/m}$.


\subparagraph{}\textit{Tracer le graphe de structure. Définir le nombre d'inconnues statiques.}

\subparagraph{}\textit{Donner la stratégie permettant de déterminer toutes les inconnues de liaisons.}

\subparagraph{}\textit{Déterminer les actions mécaniques dans toutes les liaisons.}



\section*{Paroi rigide amovible}
\setcounter{exo}{0}

La figure ci-dessous représente de profil une véranda équipée d'une paroi amovible. Cette paroi repérée \textbf{(1)} peut être totalement escamotée sous le toit au moyen d'un guidage dans un rail fixé sur la poutre supérieure, par l'intermédiaire d'un galet de centre $D$ non représenté sur le schéma. Le levier (2) permet de soulever la partie inférieure de la paroi \textbf{(1)}. Un vérin électrique \textbf{(3)}-\textbf{(4)} ; articulé en $B$ et $E$ assure la mise en mouvement de l'ensemble.

\begin{center}
\includegraphics[width=.9\linewidth]{images/fig_04}
\end{center}

On a : 
\begin{itemize}
\item $\vect{EA}=a_0\vect{x_0} + b_0\vect{y_0}$;
\item $\vect{BA}=d_2\vect{x_2}$;
\item $\vect{CA}=L_2\vect{x_0}$;
\item $\vect{EB}=d_4\vect{y_0}$;
\item $\vect{CD}=L_1\vect{x_1}$;
\item $\vect{HD}=H_1\vect{x_1}$;
\item $\left(\vect{x_0},\vect{x_2}\right)=\theta_{10}$ et $\left(\vect{x_0},\vect{u_0}\right)=\alpha_{0}$.
\end{itemize}
L'action du corps de vérin \textbf{(4)} sur la tige \textbf{(3)} est notée $\vect{F_{43}}\vect{y_0}=F_m$.

L'action de pesanteur sur la paroi \textbf{(1)} est modélisée par une action répartie $p_0$. Le poids des autres éléments est supposé négligeable.

\subparagraph{}\textit{Tracer le graphe de structure. Définir le nombre d'inconnues statiques.}

\subparagraph{}\textit{Donner la stratégie permettant de déterminer la valeur de la poussée $F_m$ en fonction de $p_0$.}

\subparagraph{}\textit{Déterminer les actions mécaniques dans toutes les liaisons.}

\ifprof
\else
\end{multicols}
\fi

%\begin{center}
%\includegraphics[width=\linewidth]{images/fig_04}
%%\textit{}
%\end{center}

\end{document}

\subparagraph{}\textit{}


\begin{center}
\includegraphics[width=\linewidth]{images/fig_06}
%\textit{}
\end{center}
\begin{center}
\includegraphics[width=\linewidth]{images/img_04}
%\textit{}
\end{center}

