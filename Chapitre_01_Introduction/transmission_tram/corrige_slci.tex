\documentclass[]{article}
\usepackage{lmodern}
\usepackage{amssymb,amsmath}
\usepackage{ifxetex,ifluatex}
\usepackage{fixltx2e} % provides \textsubscript
\ifnum 0\ifxetex 1\fi\ifluatex 1\fi=0 % if pdftex
  \usepackage[T1]{fontenc}
  \usepackage[utf8]{inputenc}
\else % if luatex or xelatex
  \ifxetex
    \usepackage{mathspec}
  \else
    \usepackage{fontspec}
  \fi
  \defaultfontfeatures{Ligatures=TeX,Scale=MatchLowercase}
\fi
% use upquote if available, for straight quotes in verbatim environments
\IfFileExists{upquote.sty}{\usepackage{upquote}}{}
% use microtype if available
\IfFileExists{microtype.sty}{%
\usepackage[]{microtype}
\UseMicrotypeSet[protrusion]{basicmath} % disable protrusion for tt fonts
}{}
\PassOptionsToPackage{hyphens}{url} % url is loaded by hyperref
\usepackage[unicode=true]{hyperref}
\hypersetup{
            pdfborder={0 0 0},
            breaklinks=true}
\urlstyle{same}  % don't use monospace font for urls
\usepackage{graphicx,grffile}
\makeatletter
\def\maxwidth{\ifdim\Gin@nat@width>\linewidth\linewidth\else\Gin@nat@width\fi}
\def\maxheight{\ifdim\Gin@nat@height>\textheight\textheight\else\Gin@nat@height\fi}
\makeatother
% Scale images if necessary, so that they will not overflow the page
% margins by default, and it is still possible to overwrite the defaults
% using explicit options in \includegraphics[width, height, ...]{}
\setkeys{Gin}{width=\maxwidth,height=\maxheight,keepaspectratio}
\IfFileExists{parskip.sty}{%
\usepackage{parskip}
}{% else
\setlength{\parindent}{0pt}
\setlength{\parskip}{6pt plus 2pt minus 1pt}
}
\setlength{\emergencystretch}{3em}  % prevent overfull lines
\providecommand{\tightlist}{%
  \setlength{\itemsep}{0pt}\setlength{\parskip}{0pt}}
\setcounter{secnumdepth}{0}
% Redefines (sub)paragraphs to behave more like sections
\ifx\paragraph\undefined\else
\let\oldparagraph\paragraph
\renewcommand{\paragraph}[1]{\oldparagraph{#1}\mbox{}}
\fi
\ifx\subparagraph\undefined\else
\let\oldsubparagraph\subparagraph
\renewcommand{\subparagraph}[1]{\oldsubparagraph{#1}\mbox{}}
\fi

% set default figure placement to htbp
\makeatletter
\def\fps@figure{htbp}
\makeatother


\date{}

\begin{document}

Q17- Système non perturbé

\begin{enumerate}
\def\labelenumi{\alph{enumi}.}
\item
  Fonction de transfert
\end{enumerate}

La formule de Black permet d'obtenir~:

\[H_{1}\left( p \right) = r_{g} \cdot \frac{\frac{K_{m}}{\left( R + L \cdot p \right) \cdot \left( f_{v} + J\  \cdot p \right)}}{1 + \frac{K_{m} \cdot K_{e}}{\left( R + L \cdot p \right) \cdot \left( f_{v} + J\  \cdot p \right)}}\]

Sous forme canonique~:

\[H_{1}\left( p \right) = \frac{\frac{r_{g} \cdot K_{m}}{R \cdot f_{v} + K_{m} \cdot K_{e}}}{1 + \frac{\left( R \cdot J + L \cdot f_{v} \right)}{R \cdot f_{v} + K_{m} \cdot K_{e}} \cdot p + \frac{L \cdot J}{R \cdot f_{v} + K_{m} \cdot K_{e}} \cdot p^{2}}\]

\begin{enumerate}
\def\labelenumi{\alph{enumi}.}
\item
  Valeur finale
\end{enumerate}

Soit \(V_{s\infty 1}\) la valeur finale de \(V_{s}\) pour le système non
perturbé. On a~:

\[V_{s\infty 1} = \operatorname{}{\left( p \cdot \frac{U_{0}}{p} \cdot H_{1}\left( p \right) \right) = \frac{r_{g} \cdot K_{m} \cdot U_{0}}{R \cdot f_{v} + K_{m} \cdot K_{e}}}\]

\begin{enumerate}
\def\labelenumi{\alph{enumi}.}
\item
  Tension à vide
\end{enumerate}

On a donc
\(U_{sat - vide} = \frac{V_{s\infty 1} \cdot \left( R \cdot f_{v} + K_{m} \cdot K_{e} \right)}{r_{g} \cdot K_{m}} = 223,8\ V\).

Q18- Système soumis à la seule perturbation

\begin{enumerate}
\def\labelenumi{\alph{enumi}.}
\item
  Fonction de transfert
\end{enumerate}

On a~:

\[H_{2}\left( p \right) = - \lambda \cdot \frac{\frac{1}{\left( f_{v} + J\  \cdot p \right)}}{1 + \frac{K_{m} \cdot K_{e}}{\left( f_{v} + J\  \cdot p \right) \cdot \left( R + L \cdot p \right)}}\]

Soit, sous forme canonique~:

\[H_{2}\left( p \right) = \frac{- \frac{\lambda \cdot R \cdot \left( 1 + \frac{L}{R} \cdot p \right)}{\left( R \cdot f_{v} + K_{m} \cdot K_{e} \right)}}{1 + \frac{\left( R \cdot J + L \cdot f_{v} \right)}{R \cdot f_{v} + K_{m} \cdot K_{e}} \cdot p + \frac{L \cdot J}{R \cdot f_{v} + K_{m} \cdot K_{e}} \cdot p^{2}}\]

\begin{enumerate}
\def\labelenumi{\alph{enumi}.}
\item
  Valeur finale
\end{enumerate}

Soit \(V_{s\infty 2}\) la valeur finale de \(V_{s}\) pour le système
soumis à la seule perturbation. On a~:

\[V_{s\infty 2} = \operatorname{}{\left( p \cdot \frac{C_{r}}{p} \cdot H_{2}\left( p \right) \right) = - \frac{\lambda \cdot R \cdot C_{r}}{\left( R \cdot f_{v} + K_{m} \cdot K_{e} \right)}}\]

\begin{enumerate}
\def\labelenumi{\alph{enumi}.}
\item
  Application numérique
\end{enumerate}

On obtient alors \(V_{s\infty 2} = - 1,15 \cdot 10^{- 3}\ m \slash s\)

Q19- Système perturbé, principe de superposition

\begin{enumerate}
\def\labelenumi{\alph{enumi}.}
\item
  Tension sous charge
\end{enumerate}

Pour compenser la perte de vitesse \(V_{s\infty 2}\), on doit appliquer
une tension supplémentaire \(\Delta_{U_{\text{sat}}}\). La question
Q17-b permet d'écrire~:

\[\Delta_{U_{\text{sat}}} = - \frac{V_{s\infty 2} \cdot \left( R \cdot f_{v} + K_{m} \cdot K_{e} \right)}{\lambda \cdot K_{m}}\]

D'où
\(U_{sat - charge} = U_{sat - vide} + \Delta_{U_{\text{sat}}} = \frac{\left( V_{s\infty 1} - V_{s\infty 2} \right) \cdot \left( R \cdot f_{v} + K_{m} \cdot K_{e} \right)}{\lambda \cdot K_{m}}\)

\begin{enumerate}
\def\labelenumi{\alph{enumi}.}
\item
  Application numérique
\end{enumerate}

Donc \(U_{sat - charge} = 249,5\ V\).

\begin{enumerate}
\def\labelenumi{\alph{enumi}.}
\item
  Correcteur
\end{enumerate}

Un correcteur proportionnel-intégral augmente la classe du système, qui
passe à 1. L'erreur statique est donc nulle, sous réserve de stabilité.
Ce correcteur, placé en amont de la perturbation, annule également
l'effet de celle-ci.

\subsubsection{Deuxième étude~:
Couplage}\label{deuxiuxe8me-uxe9tude-couplage}

Q20- Pertinence du correcteur

L'utilisation d'un correcteur à caractère intégral va poser un problème
de stabilité.

Q21- Ecart de position entre colonnes

\begin{enumerate}
\def\labelenumi{\alph{enumi}.}
\item
  Expression
\end{enumerate}

On peut adapter le schéma-blocs pour une lecture plus directe~:

\includegraphics[width=6.53194in,height=2.68403in]{media/image1.png}

Une lecture du schéma-blocs permet alors d'écrire~:

\[\varepsilon_{12 - 2}\left( p \right) = \frac{1}{\left( f_{v} + J\  \cdot p \right)} \cdot \left( - \frac{\lambda}{p} \right) \cdot \left( - C_{r2}\left( p \right) + \frac{K_{m}}{R} \cdot K_{e} \cdot \left( \frac{\left( K + K_{v} \cdot p \right)}{K_{e}} - \left( - \frac{p}{\lambda} \right) \right) \cdot \varepsilon_{12 - 2}\left( p \right) \right)\]

On isole alors \(\varepsilon_{12 - 2}\) et \(C_{r2}\), en on obtient~:

\[\varepsilon_{12 - 2}\left( p \right) = \frac{C_{r2}\left( p \right) \cdot R \cdot \lambda}{p \cdot \left\lbrack R \cdot \left( f_{v} + J\  \cdot p \right) + K_{m} \cdot \left( K_{e} + \lambda \cdot K_{v} \right) \right\rbrack + \lambda \cdot K_{m} \cdot K}\]

\begin{enumerate}
\def\labelenumi{\alph{enumi}.}
\item
  Valeur du gain
\end{enumerate}

On a donc~:

\[\varepsilon_{12}\left( p \right) = \frac{\left( C_{r2}\left( p \right) - C_{r1}\left( p \right) \right) \cdot R \cdot \lambda}{p \cdot \left\lbrack R \cdot \left( f_{v} + J\  \cdot p \right) + K_{m} \cdot \left( K_{e} + \lambda \cdot K_{v} \right) \right\rbrack + \lambda \cdot K_{m} \cdot K}\]

La valeur finale \(\varepsilon_{12\infty}\) de l'écart de position est
donc
\(\varepsilon_{12\infty} = \Delta C_{r} \cdot \frac{R}{K_{m} \cdot K}\)
, soit~:

\[K = \Delta C_{r} \cdot \frac{R}{K_{m} \cdot \varepsilon_{12\infty}}\]

Pour avoir un écart de 5mm maximal pour \(\Delta C_{r} = 2\ N \cdot m\),
on doit donc avoir \(K > 857\ V \slash m\), sous réserve que la réponse
temporelle ne présente pas de dépassement.

On calcule, pour ce système du second ordre, avec la valeur de gain
ci-dessus, une pulsation propre non amortie
\(\omega_{0} = 1,13\ rad \slash s\) et un coefficient d'amortissement
\(\xi = 14\ \): le système ne présente pas de dépassement, mais a un
temps de réponse à 5\% de 70s environ.

\end{document}
