\documentclass[10pt,fleqn]{article} % Default font size and left-justified equations
\usepackage[%
    pdftitle={Modélisation dynamique},
    pdfauthor={Xavier Pessoles}]{hyperref}

    
\input{style/new_style}
\input{style/macros_SII}
\usepackage{multicol}
\usepackage{siunitx}
%\usepackage{picins}
\fichetrue
%\fichefalse

\proftrue
%\proffalse

\tdtrue
%\tdfalse

\courstrue
\coursfalse


\def\classe{\textsf{PSI$\star$ -- MP}}
\def\xxnumpartie{Cycle 04}
\def\xxpartie{Modéliser le comportement des systèmes mécaniques dans le but d'établir une loi de comportement ou de déterminer des actions mécaniques en utilisant le PFD}

\def\xxnumchapitre{Chapitre 1 \vspace{.2cm}}
\def\xxchapitre{\hspace{.12cm} Introduction à la dynamique du solide indéformable}

\def\discipline{Sciences \\Industrielles de \\ l'Ingénieur}
\def\xxtete{Sciences Industrielles de l'Ingénieur}




\def\xxtitreexo{TD -- Véhicule TIM}%Motorisation du moteur Haibike}
\def\xxsourceexo{\hspace{.2cm} \footnotesize{Florestan Mathurin}}


\def\xxposongletx{2}
\def\xxposonglettext{1.45}
\def\xxposonglety{20}
%\def\xxonglet{Part. 1 -- Ch. 3}
\def\xxonglet{Cycle 04}

\def\xxactivite{Activation}
\def\xxauteur{\textsl{F. Mathurin}}

\def\xxcompetences{%
\textsl{%
\textbf{Savoirs et compétences :}\\
%Les sources sont associées par un \emph{hacheur série}. La détermination des grandeurs électriques associées à ce montage permet de conclure vis à vis du cahier des charges.
%\noindent \textbf{Résoudre :} à partir des modèles retenus :
%\begin{itemize}[label=\ding{112},font=\color{ocre}] 
%\item choisir une méthode de résolution analytique, graphique, numérique;
%\item mettre en \oe{}uvre une méthode de résolution.
%\end{itemize}
%\begin{itemize}[label=\ding{112},font=\color{ocre}] 
%\item \textit{Rés -- C1.1 :} Loi entrée sortie géométrique et cinématique -- Fermeture géométrique.
%\end{itemize}
%
%\noindent \textit{Mod2 -- C4.1 :} Représentation par schéma bloc.
}}

\def\xxfigures{
\includegraphics[width=.7\linewidth]{images/fig_01}
}%figues de la page de garde


\def\xxpied{%
Cycle 04 -- Modélisation mécanique -- Dynamique\\% afin de valider leurs performances.\\
Chapitre 1 -- \xxactivite%
}

\setcounter{secnumdepth}{5}
%---------------------------------------------------------------------------

\usepackage{pgfplots}
\begin{document}
\def\pathfig{images}
%\chapterimage{png/Fond_Cin}
\input{style/new_pagegarde}
\vspace{4.5cm}
\pagestyle{fancy}
\thispagestyle{plain}

\def\columnseprulecolor{\color{ocre}}
\setlength{\columnseprule}{0.4pt} 

\def\pathfig{images}

\ifprof
\begin{multicols}{2}
\else
\begin{multicols}{2}
\fi


L’éco-marathon SHELL est une compétition relative à la consommation énergétique des moyens de propulsion automobile. Les concurrents doivent concevoir et piloter leur véhicule sur une distance fixée avec une vitesse minimale imposée.  Les candidats sont ensuite classés en fonction de la consommation de leur véhicule, exprimée en « kilomètre par litre » de carburant. L’étude sur ce sujet, issue d’un projet élaboré par l’équipe T.I.M. de l’INSA Toulouse, a pour objet de quantifier les effets résistants et dissipatifs que sont la résistance au roulement et les actions aérodynamiques sur les performances de leur véhicule. Les effets inertiels étant plutôt quantifiés numériquement au niveau de la conception assistée par ordinateur du véhicule. 

\subsection*{Détermination expérimentale du coefficient de résistance au roulement}

Le principe est présenté sur la figure 1. On place 2 roues lestées sur un dispositif inclinable. On considère ensuite que l’angle d’inclinaison minimum de la pente, où il y a début du mouvement des roues, est représentatif de la résistance au roulement.  
L’ensemble des 2 roues lestées peut être assimilé à un solide 1 représenté sur la figure 1, de masse $m$, de rayon $R$ et de centre de masse $G$. 

L'accélération de la pesanteur $\vect{g}$ tel que $\vect{g}=-g\vect{z_0}$. 


\begin{center}
\includegraphics[width=.95\linewidth]{images/fig_02}
\end{center}

L’action de contact entre l’ensemble des roues 1 et le plan 0, incliné d’un angle $\alpha$ par rapport à l’horizontale, est  modélisé comme un contact ponctuel avec frottement où Figure 1 l’on tient compte de la résistance au roulement. 

Cette action de contact peut s’écrire :$\torseurstat{T}{0}{1}=\torseurl{-T_{01}\vect{x}+N_{01}\vect{z}}{-C_r\vect{y}}{A_1}$ où $C_r$ représente le couple de résistance au roulement qui s’oppose au roulement tel que : $|C_r|=r|N_{01}|$ à la limite de l’équilibre et $|C_r|=r|N_{01}|$ à l’équilibre.

\subparagraph{}
\textit{Écrire le principe fondamental de la statique appliqué au solide 1 réduit au point $G$ en projection sur la base $\base{x}{y}{z}$.}
\ifprof
\begin{corrige}
\end{corrige}
\else
\fi


\subparagraph{}
\textit{Déterminer l’expression analytique de l’angle $\alpha_{\text{lim}}$ à la limite de l’équilibre quand il y a début du roulement du solide 1 sur le plan 0. }
\ifprof
\begin{corrige}
\end{corrige}
\else
\fi

Pour une masse du solide 1 $m = \SI{50}{kg}$ et pour un rayon $R = \SI{0,25}{m}$ le roulement se produit à partir d’un angle  $\alpha_{\text{lim}}$ tel que $\tan \alpha_{\text{lim}} = 0,008$. 

\subparagraph{}
\textit{Déterminer le coefficient de résistance au roulement $r$.}
\ifprof
\begin{corrige}
\end{corrige}
\else
\fi


\subparagraph{}
\textit{Au début du roulement, montrer qu’il ne peut pas y avoir glissement en $A_1$ si le coefficient de frottement au contact vaut $f = 0,5$. }
\ifprof
\begin{corrige}
\end{corrige}
\else
\fi
\subsection*{Modélisation du véhicule}


L’objectif est d’établir un modèle analytique du véhicule, lors d’une phase de roulement sans glissement sur une ligne droite inclinée d’un angle $\alpha$, en l’absence de vent. En adoptant des conditions particulières d’essai, il sera possible d’identifier précisément, grâce à ce modèle, les actions aérodynamiques. 

L’accélération de la pesanteur $\vect{g}$ telle que $\vect{g}=-g\vect{z_0}$. Le modèle est donné figure suivante. On considère que le véhicule se déplace sur une pente inclinée d’un angle$\alpha$ par rapport à l’horizontale. Le véhicule est constitué :
\begin{itemize}
\item d’un châssis avec son pilote : solide 1 de centre d’inertie $G$, de masse $M$ en translation par rapport au repère galiléen $R$ avec $\vect{OG}=x\vect{x}+R\vect{z}$.
\item de deux roues avant : solide 23 de centre d’inertie $O_{23}$ , de masse $2 m$, de rayon $R$, dont le moment d’inertie par rapport à l’axe $\axe{O_{23}}{\vect{y}}$ sera noté $2I$. Le solide 23 est en liaison pivot sans frottement par rapport au châssis 1 d’axe $\axe{O_{23}}{\vect{y}}$ caractérisé par le paramètre $\theta_{23}$. 
\item d’une roue arrière motrice : solide 4 de centre d’inertie $O_4$, de masse $m$, de rayon $R$, dont le moment d’inertie par rapport à l’axe $\axe{O_{4}}{\vect{y}}$ sera noté $I$. Le solide 4 est en liaison pivot sans frottement par rapport au châssis 1 d’axe $\axe{O_{4}}{\vect{y}}$ caractérisé par le paramètre $\theta_4$; 
\item un moteur d’entraînement du véhicule dont le corps est solidaire du châssis 1 exerce sur la roue 4 un couple moteur noté $C_m \vect{y}$. 
\end{itemize}
*
\ifprof
\end{multicols}
\else
\end{multicols}
\fi

\end{document}

\subparagraph{}
\textit{}
\ifprof
\begin{corrige}
\end{corrige}
\else
\fi

\begin{center}
\includegraphics[width=.95\linewidth]{images/}
\end{center}
