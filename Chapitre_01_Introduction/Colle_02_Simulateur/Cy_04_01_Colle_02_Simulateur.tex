\documentclass[10pt,fleqn]{article} % Default font size and left-justified equations
\usepackage[%
    pdftitle={Modélisation dynamique},
    pdfauthor={Xavier Pessoles}]{hyperref}

    
\input{style/new_style}
\input{style/macros_SII}
\usepackage{multicol}
\usepackage{siunitx}
%\usepackage{picins}
\fichetrue
%\fichefalse

\proftrue
\proffalse

\tdtrue
%\tdfalse

\courstrue
\coursfalse


\def\classe{\textsf{PSI$\star$ -- MP}}
\def\xxnumpartie{Cycle 04}
\def\xxpartie{Modéliser le comportement des systèmes mécaniques dans le but d'établir une loi de comportement ou de déterminer des actions mécaniques en utilisant le PFD}

\def\xxnumchapitre{Chapitre 1 \vspace{.2cm}}
\def\xxchapitre{\hspace{.12cm} Introduction à la dynamique du solide indéformable}

\def\discipline{Sciences \\Industrielles de \\ l'Ingénieur}
\def\xxtete{Sciences Industrielles de l'Ingénieur}




\def\xxtitreexo{Simulateur de conduite}%Motorisation du moteur Haibike}
\def\xxsourceexo{\hspace{.2cm} \footnotesize{CCP PSI 2014}}


\def\xxposongletx{2}
\def\xxposonglettext{1.45}
\def\xxposonglety{20}
%\def\xxonglet{Part. 1 -- Ch. 3}
\def\xxonglet{Cycle 04}

\def\xxactivite{Colle 02}
\def\xxauteur{\textsl{X. Pessoles}}

\def\xxcompetences{%
\textsl{%
\textbf{Savoirs et compétences :}\\
%Les sources sont associées par un \emph{hacheur série}. La détermination des grandeurs électriques associées à ce montage permet de conclure vis à vis du cahier des charges.
%\noindent \textbf{Résoudre :} à partir des modèles retenus :
%\begin{itemize}[label=\ding{112},font=\color{ocre}] 
%\item choisir une méthode de résolution analytique, graphique, numérique;
%\item mettre en \oe{}uvre une méthode de résolution.
%\end{itemize}
%\begin{itemize}[label=\ding{112},font=\color{ocre}] 
%\item \textit{Rés -- C1.1 :} Loi entrée sortie géométrique et cinématique -- Fermeture géométrique.
%\end{itemize}
%
%\noindent \textit{Mod2 -- C4.1 :} Représentation par schéma bloc.
}}

\def\xxfigures{
\includegraphics[width=.6\linewidth]{images/fig_01}
}%figues de la page de garde


\def\xxpied{%
Cycle 04 -- Modélisation mécanique -- Dynamique\\% afin de valider leurs performances.\\
Chapitre 1 -- \xxactivite%
}

\setcounter{secnumdepth}{5}
%---------------------------------------------------------------------------

\usepackage{pgfplots}
\begin{document}
\def\pathfig{images}
%\chapterimage{png/Fond_Cin}
\input{style/new_pagegarde}
\vspace{4.5cm}
\pagestyle{fancy}
\thispagestyle{plain}

\def\columnseprulecolor{\color{ocre}}
\setlength{\columnseprule}{0.4pt} 

\def\pathfig{images}

\ifprof
%\begin{multicols}{2}
\else
\begin{multicols}{2}
\fi
Le simulateur étudié dans ce sujet est un simulateur de course automobile à deux degrés de
liberté utilisé par des particuliers dans le domaine du loisir

La cinématique retenue pour le simulateur est basée sur une structure articulée permettant
deux degrés de liberté par l’intermédiaire de deux vérins linéaires asservis. On désigne par \textbf{(3a)}
et \textbf{(3b)} les corps des vérins en liaison sphérique avec le châssis noté \textbf{(0)}, \textbf{(2a)} et \textbf{(2b)} les tiges
des vérins en liaison sphérique avec le siège noté \textbf{(1)}, lui même en liaison avec le châssis. Les
tiges des vérins sont en liaison glissière avec les corps des vérins. La liaison entre le siège et le
châssis est réalisée par un joint de cardan \textbf{(C)} qui autorise deux rotations (selon les axes $(O,\vect{x_0})$
et $(O,\vect{y_0})$.

\begin{center}
\includegraphics[width=.95\linewidth]{images/fig_02}
\end{center}

On donne le cahier des charges partiel suivant. 
\begin{center}
\begin{tabular}{|p{.45\linewidth}|p{.45\linewidth}|}
\hline
Critères &  Niveaux \\ \hline
Débattement angulaire & $\pm \, 13\degres$ \\ \hline
Accélération extrême $a_{Tx}$ (définie dans la suite) & 
$\pm \, 13\degres$ \\ \hline
Masse du conducteur admissible & \SI{100}{kg} \\
\hline 

\end{tabular}
\end{center}
\subparagraph{}\textit{Déterminer le degré d’hyperstatisme du modèle ainsi proposé en précisant bien les mobilités
utiles et internes. Indiquer l’intérêt d’une telle modélisation vis-à-vis de la détermination
des efforts dans le système. Préciser un autre intérêt de ce degré d’hyperstatisme.}


Dans la suite on ne s’intéresse qu’au mouvement de tangage (rotation
autour de $\vect{y_0}$). Dans ces conditions, il est possible de trouver un modèle plan équivalent du
mécanisme. Le vérin est alors appelé vérin équivalent.
La figure suivante correspond à cette modélisation plane équivalente. Le paramétrage est donné
sur cette figure. $\alpha$ est l’angle de tangage du siège par rapport au châssis. Dans toute la suite du problème, on ne s’intéressera qu’à ce modèle plan.


\begin{center}
\includegraphics[width=.95\linewidth]{images/fig_03}
\end{center}

Le modèle établi va permettre de vérifier le dimensionnement des vérins vis-à-vis des critères du cahier des charges. On donne les courbes faisant le lien entre l'angle $\alpha$ et la longueur du vérin d'une part ainsi que la relation entre l'angle $\alpha$ et $\beta$. Lorsque l’assise du siège est horizontale, l’angle $\alpha$ est nul, le vérin
est alors à mi course (la longueur $\lambda$ est de \SI{0,99}{m}). La course du vérin équivalent est de \SI{0,15}{m},
 $\lambda$ peut donc varier de $\pm \SI{0,075}{m}$ autour de \SI{0,99}{m}.

\begin{center}
\includegraphics[width=.95\linewidth]{images/fig_04_a}
\end{center}

\begin{center}
\includegraphics[width=.95\linewidth]{images/fig_04_b}
\end{center}

\subparagraph{}\textit{Déterminer le débattement angulaire et comparer avec la valeur du cahier des charges.}
On approche les deux courbes par des droites au voisinage de $\alpha = 0\degres$ : $\lambda = \lambda_0 + K_{\alpha} \alpha$ et
$\beta= \beta_0 + K_{\beta} \alpha$.
\subparagraph{}\textit{En utilisant les courbes ci-dessus, 
donner les valeurs numériques de $K_{\alpha}$, $K_{\beta}$ et $\beta_0$.
%,  } 
Conserver les unités définies sur le figures.}


\subsection*{Critère de masse admissible}

D'après les données du constructeur, le vérin équivalent peut développer un effort maximal de $\pm \SI{200}{N}$ environ.  On cherche dans cette partie à vérifier si le vérin est capable de mettre en mouvement le siège sur lequel serait assis un conducteur ayant la masse définie dans le cahier des charges. 
On définit les grandeurs cinétiques et géométriques suivantes :
\begin{itemize}
\item $J=\SI{10}{kg.m^2}$, moment d'inertie de l'ensemble \{conducteur + siège\} selon l'axe $\left( O,\vect{y_0}\right)$;
\item $m=\SI{100}{kg}$, masse de l'ensemble \{conducteur + siège\};
\item $\vect{OG}=d\vect{z_1}$ avec $d=\SI{0,35}{m}$ : position du centre de gravité de l'ensemble \{conducteur + siège \} (position simplifiée pour limiter les calculs).  
\end{itemize}
On note : 
\begin{itemize}
\item $-g\vect{z_0}$ avec $g=\SI{9,81}{m.s^{-2}}$, accélération de la pesanteur;
\item $F\vect{x_3}$, l'action mécanique de la tige du vérin équivalente \textbf{(2)} sur la piège 
\textbf{(1)} se modélise par un glisseur en $A$.
\end{itemize}

\subparagraph{}\textit{En isolant le vérin équivalent \{tige (2) + corps (3)\}, justifier que
l’effort exercé par ce vérin équivalent est dirigé selon $\vect{x_3}$.  La masse du vérin et ses caractéristiques inertielles seront supposées négligeables.}

\subparagraph{}\textit{Déterminer une équation reliant les quantités définies ci-dessus et l’angle $\alpha$ ainsi que ses dérivées sous la forme : $A_S\dfrac{\text{d}^2\alpha(t)}{\text{d}t^2}=B_S \times F\cos \left( \beta-\alpha\right)+C_S \sin \alpha$ où l'on donnera l'expression de $A_S$, $B_S$ et $C_S$ en fonction des paramètres constants. Pour cela préciser le système isolé, faire le bilan des actions mécaniques et indiquer l’équation du principe fondamental de la dynamique utilisée (résultante/moment, direction, point).}

Le cahier des charges définit un angle maximal de $13\degres$. L'accélération $a_{Tx}$ ressentie par le conducteur indiquée dans ce cahier des charges de $\pm\SI{2,2}{m.s^{-2}}$ est égale à $a_{Tx}=h\ddot{alpha}-g\sin\alpha$. On prend les valeurs numériques suivantes : $A_S = \SI{10}{kg.m^2}$, $B_S=\SI{0,7}{m}$ et $C_s=\SI{350}{Nm}$.

\subparagraph{}\textit{En déduire l'expression de la force du vérin $F$ en fonction de $a_{Tx}$, $\alpha$, $\beta$, $g$, $A_S$, $B_S$ et $C_S$. Effectuer l'application numérique dans les conditions les plus défavorables ($\alpha=13\degres$, accélération $a_{Tx}=\SI{-2,2}{m.s^{-2}}$), la valeur de $\beta$ sera lue sur la courbe précédente. Conclure quant au choix de ce vérin.}

\ifprof
%\end{multicols}
\else
\end{multicols}
\fi

\newpage

\begin{center}
\includegraphics[width=.8\linewidth]{images/cor_01}
\end{center}

\begin{center}
\includegraphics[width=.8\linewidth]{images/cor_02}
\end{center}


\begin{center}
\includegraphics[width=.8\linewidth]{images/cor_03}
\end{center}


\end{document}

\subparagraph{}\textit{}

\begin{center}
\includegraphics[width=.95\linewidth]{images/}
\end{center}
