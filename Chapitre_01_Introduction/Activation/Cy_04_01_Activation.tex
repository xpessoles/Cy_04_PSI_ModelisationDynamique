\documentclass[10pt,fleqn]{article} % Default font size and left-justified equations
\usepackage[%
    pdftitle={Modélisation dynamique},
    pdfauthor={Xavier Pessoles}]{hyperref}

    
\input{style/new_style}
\input{style/macros_SII}
\usepackage{multicol}
\usepackage{siunitx}
%\usepackage{picins}
\fichetrue
%\fichefalse

\proftrue
%\proffalse

\tdtrue
%\tdfalse

\courstrue
\coursfalse


\def\classe{\textsf{PSI$\star$ -- MP}}
\def\xxnumpartie{Cycle 04}
\def\xxpartie{Modéliser le comportement des systèmes mécaniques dans le but d'établir une loi de comportement ou de déterminer des actions mécaniques en utilisant le PFD}

\def\xxnumchapitre{Chapitre 1 \vspace{.2cm}}
\def\xxchapitre{\hspace{.12cm} Introduction à la dynamique du solide indéformable}

\def\discipline{Sciences \\Industrielles de \\ l'Ingénieur}
\def\xxtete{Sciences Industrielles de l'Ingénieur}




\def\xxtitreexo{Activation}%Motorisation du moteur Haibike}
\def\xxsourceexo{\hspace{.2cm} \footnotesize{Emilien Durif}}


\def\xxposongletx{2}
\def\xxposonglettext{1.45}
\def\xxposonglety{20}
%\def\xxonglet{Part. 1 -- Ch. 3}
\def\xxonglet{Cycle 02}

\def\xxactivite{Activation}
\def\xxauteur{\textsl{E. Durif}}

\def\xxcompetences{%
\textsl{%
\textbf{Savoirs et compétences :}\\
%Les sources sont associées par un \emph{hacheur série}. La détermination des grandeurs électriques associées à ce montage permet de conclure vis à vis du cahier des charges.
%\noindent \textbf{Résoudre :} à partir des modèles retenus :
%\begin{itemize}[label=\ding{112},font=\color{ocre}] 
%\item choisir une méthode de résolution analytique, graphique, numérique;
%\item mettre en \oe{}uvre une méthode de résolution.
%\end{itemize}
%\begin{itemize}[label=\ding{112},font=\color{ocre}] 
%\item \textit{Rés -- C1.1 :} Loi entrée sortie géométrique et cinématique -- Fermeture géométrique.
%\end{itemize}
%
%\noindent \textit{Mod2 -- C4.1 :} Représentation par schéma bloc.
}}

\def\xxfigures{
%\includegraphics[width=.9\linewidth]{images/c-evolution}
}%figues de la page de garde


\def\xxpied{%
Cycle 04 -- Modéliser le comportement des systèmes mécaniques -- Dynamique\\% afin de valider leurs performances.\\
Chapitre 1 -- \xxactivite%
}

\setcounter{secnumdepth}{5}
%---------------------------------------------------------------------------

\usepackage{pgfplots}
\begin{document}
\def\pathfig{images}
%\chapterimage{png/Fond_Cin}
\input{style/new_pagegarde}
\vspace{4cm}
\pagestyle{fancy}
\thispagestyle{plain}

\def\columnseprulecolor{\color{ocre}}
\setlength{\columnseprule}{0.4pt} 

\def\pathfig{images}

\ifprof
\else
\begin{multicols}{2}
\fi


\subsection{Présentation du support du cours du cours}



\begin{exemple}[Système de dépose de composants électroniques]
Le système étudié permet de déposer automatiquement des composants électroniques sur un circuit.
On s'intéresse ici à la modélisation d'un seul axe (selon la direction notée $\overrightarrow{y}_0$). actionné par un moteur électrique et utilisant un mécanisme de transformation de mouvement.

\textbf{Hypothèses :}
\begin{itemize}
\item le référentiel associé au repère $R_0=\quadruplet{O_0}{\overrightarrow{x}_0}{\overrightarrow{y}_0}{\overrightarrow{z}_0}$ est supposé galiléen ;
\item les solides seront supposés indéformables ; 
\item on notera $J_1$ le moment d'inertie du solide 1 selon l'axe $\couple{O_0}{\overrightarrow{y}_0}$ : $J_1=I_{\couple{O_0}{\overrightarrow{y}_0}}(S_1)$ ;
\item on note $M_3$ et $G_3$ respectivement la masse et le centre d'inertie du solide $S_3$ ;
\item la position de $G_3$ est définie par $\overrightarrow{O_0G_3}=x\cdot \overrightarrow{x}_0+y\cdot \overrightarrow{y}_0+z\cdot \overrightarrow{z}_0$
\item les liaisons sont supposées parfaites (sans jeu ni frottement).
\end{itemize}

\begin{center}
\begin{tabular}{cc}
\includegraphics[width=0.45\textwidth]{images/axe_y_photo.png}
&
\includegraphics[width=0.45\textwidth]{images/schema_cine_depose_composant.pdf}
\end{tabular}
\includegraphics[width=0.7\textwidth]{images/req_systeme_depose.pdf}
\end{center}
\begin{itemize}
\item $S_0$ : poutre transversale considérée comme fixe par rapport au bâti ;
\item $S_1$ : vis à billes (hélice à droite) ;
\item $S_2$ : écrou de la vis à billes ;
\item $S_3$ : chariot supportant la tête de dépose (masse $M_3$) ;
\end{itemize}
\end{exemple}

\begin{obj}
L'objectif de cette étude est de relier les grandeurs liées à l'actionneur du système (moteur) :
\begin{itemize}
\item couple transmis à $S_1$ : $\overrightarrow{C}_{Moteur\to S_1}$ ;
\item vitesse de rotation de $S_1$ : $\overrightarrow{\Omega}(S_1/R_0)\cdot \overrightarrow{y}_0=\dot{\theta}$.
\end{itemize} 
à celles liée à l'effecteur (tête de dépose $S_3$) : 
\begin{itemize}
\item masse : $M_3$;
\item cinématique de $S_3$ : $\overrightarrow{a}(G_3\in S_3/R_0)\cdot \overrightarrow{y}_0=\ddot{y}$.
\end{itemize}
\end{obj}



\begin{exemple}[Machine de dépose de composants électroniques : déplacement dynamique de $S_3$]
\begin{minipage}{0.35\textwidth}
On connaît la masse $M_3$ de la tête de dépose et on cherche l'effort ($\overrightarrow{R}_{poussée\to S_3}$) de poussée que doit fournir l'actionneur pour obtenir l'accélération souhaitée.
\end{minipage}
\begin{minipage}{0.55\textwidth}
\begin{center}
\includegraphics[width=1.0\textwidth]{images/schema_cine_depose_translation.pdf}
\end{center}
\end{minipage}

On utilise le théorème de la résultante dynamique en projection sur $\overrightarrow{y}_0$. On obtient : 
%\begin{texteCache}
\begin{align*}
M_3\frac{d^2\overrightarrow{a}(G_3/R_0)}{dt^2}\cdot \overrightarrow{y}_0=\sum \overrightarrow{R}_{ext\to S_3}\cdot \overrightarrow{y}_0.\\
\\
M_3\cdot \ddot{y}=R_{poussee\to S_3}
\end{align*} 
%\end{texteCache}

\textbf{Application numérique : }
Détermination de $R_{poussee\to S_3}$ pour obtenir une accélération de $4m/s^2$ :

%\begin{texteCache}
\begin{align*}
R_{poussee\to S_3}=20\times 4=80N
\end{align*}
%\end{texteCache}
\end{exemple}




\begin{exemple}[Machine de dépose de composants électroniques : déplacement dynamique de $S_3$]
\begin{minipage}{0.5\textwidth}
On donne les caractéristiques du moteur entraînant l'axe et la vis $S_1$ :
\begin{itemize}
\item couple maximal, $C_{max} = 21,2 N\cdot m$ ;
\item fréquence de rotation maximale, $N_m = 6000 tr/min$
\item moment d'inertie du moteur suivant l'axe $\overrightarrow{y}_0$ : $I_m = 1,6 \times 10^{-4} kg.m^2$ ;
\item moment d'inertie de la vis à billes suivant l'axe $\overrightarrow{y}_0$ : $I_v = 2,15 x\times 10^{-4} kg\cdot m^2$.
\end{itemize}
De plus on notera : $\overrightarrow{\Omega}(S_1/R_0)=\dot{\theta}(t)\cdot \overrightarrow{y}_0$
\end{minipage}
\begin{minipage}{0.45\textwidth}
\begin{center}
\includegraphics[width=1.0\textwidth]{images/schema_cine_depose_rotation.pdf}
\end{center}
\end{minipage}

\textbf{Détermination des caractéristiques maximales : }

On se place de la cas le plus limite (Couple maximal, accélération angulaire constante pour atteindre la fréquence de rotation maximale en $t_a=0,2s$)
Déterminer le couple résistant maximal que le moteur peut équilibrer dynamiquement ($C_{S_2\to S_1}$):

%\begin{texteCache}
On appliquant un théorème du moment dynamique à $S_1$ selon $\couple{O_0}{\overrightarrow{y}_0}$
\begin{align*}
(I_m+I_v)\cdot \ddot{\theta}=C_{max}+C_{S_2\to S_1}
\end{align*}
On obtient alors : 

\begin{align*}
C_{S_2\to S_1}=(I_m+I_v)\cdot \ddot{\theta}_{max}-C_{max}=(I_m+I_v)\cdot \frac{N_m\times 2\cdot \pi}{60\cdot t_a}-C_{max}=-20N\cdot m
\end{align*}

%\end{texteCache}
\end{exemple}


%\begin{bilan}
\begin{itemize}
\item Les deux cas présentés ci-dessus sont traités de manière indépendante.
\item Le lien entre ces deux parties (actionneur et effecteur) repose sur le mécanisme de transformation de mouvement (ici vis-écrou).
\item Il faudra donc procéder à une démarche de résolution globale pour relier le couple moteur $\overrightarrow{C}_{moteur\to S_1}$ à l'accélération ($\overrightarrow{a}(G_3/R_0)$) et la masse de l'effecteur ($M_3$). Ce sera l'objet des parties suivantes.
\end{itemize}





\ifprof
\else
\end{multicols}
\fi


\end{document}