\documentclass[10pt,fleqn]{article} % Default font size and left-justified equations
\usepackage[%
    pdftitle={Energétique},
    pdfauthor={Xavier Pessoles}]{hyperref}

    
\input{style/new_style}
\input{style/macros_SII}
\usepackage{multicol}
\usepackage{siunitx}
%\usepackage{picins}
\fichetrue
%\fichefalse

\proftrue

\proffalse

\tdtrue
%\tdfalse

\courstrue
\coursfalse


\def\classe{\textsf{PTSI}}
\def\xxnumpartie{}%Cycle --}
\def\xxpartie{ }

\def\xxnumchapitre{}%Chapitre -- \vspace{.2cm}}
\def\xxchapitre{\hspace{.12cm} }

\def\discipline{Sciences \\Industrielles de \\ l'Ingénieur}
\def\xxtete{Sciences Industrielles de l'Ingénieur}


  
\def\xxposongletx{2}
\def\xxposonglettext{1.45}
\def\xxposonglety{20}
%\def\xxonglet{Part. 1 -- Ch. 3}
\def\xxonglet{\textsf{}}%Cycle 05}}

\def\xxactivite{Colle 01}
\def\xxauteur{\textsl{Xavier Pessoles}}


\def\xxtitreexo{}%Siège motorisé}
\def\xxsourceexo{\hspace{.2cm} \footnotesize{}}%BTS CPI 2018}}


\def\xxcompetences{%
\vspace{-.5cm}
\footnotesize{
\textsl{%
\textbf{Savoirs et compétences :}\\
\vspace{-.2cm}
%\begin{itemize}[label=\ding{112},font=\color{ocre}] 
%\item Mod2.C18.SF1 : Déterminer l’énergie cinétique d’un solide, ou d’un ensemble de solides, dans son mouvement par rapport à un autre solide.
%\item Res1.C1.SF1 : Proposer une démarche permettant la détermination de la loi de mouvement.
%\item Mod1.C5.SF2 : Déterminer la puissance des actions mécaniques extérieures à un solide ou à un ensemble de solides, dans son mouvement rapport à un autre solide.
%\item Mod1.C5.SF3 : Déterminer la puissance des actions mécaniques intérieures à un ensemble de solides.
%\end{itemize}
}}}

\def\xxfigures{
\includegraphics[width=.5\textwidth]{images/fig_01}
}%figues de la page de garde


\def\xxpied{%
%Cycle 05 -- Modélisation mécanique -- Énergétique\\% afin de valider leurs performances.\\
%Chapitre 1 -- \xxactivite%
}

\setcounter{secnumdepth}{5}
%---------------------------------------------------------------------------


\begin{document}
%\chapterimage{png/Fond_Cin}
\input{style/new_pagegarde}
\vspace{5.5cm}
\pagestyle{fancy}
\thispagestyle{plain}


\def\columnseprulecolor{\color{ocre}}
\setlength{\columnseprule}{0.4pt} 

%\ifprof
%\else
\begin{multicols}{2}
%\fi
%\section*{Exercice 1 -- Lois de Kirchoff}
%\ifprof
%\else
%\fi
%
%\subparagraph*{}
%\textit{Sur le circuit suivant, déterminer les courants dans chacune des branches et la tension aux bornes de tous les dipôles en fonction de $E$ et des différentes résistances $R_i$.}
%\begin{center}
%\includegraphics[width=\linewidth]{images/fig_01}
%\end{center}
%
%
%\section*{Exercice 2 -- Résistance équivalente}
%\textit{Déterminer la résistance équivalente du montage suivant.}
%\begin{center}
%\includegraphics[width=\linewidth]{images/fig_05}
%\end{center}
%


\section*{Exercice 1}

Pour aller rechercher des produits dans leurs rayons, Amazon utilise des axes linéaires afin de déplacer un préhenseur.
\begin{center}
\includegraphics[width=\linewidth]{images/fig_11}
\end{center}

Les performances dynamique de l'axe demandées sont les suivantes : 
\begin{itemize}
\item vitesse linéaire maximale : $50 \; \text{m}\,\text{min}^{-1}$;
\item accélération linéaire maximale : $9,8 \; \text{m}\, \text{s}^{-2}$.
\end{itemize}

\begin{obj}
L'objectif de ce travail est de déterminer les caractéristiques du moteur (vitesse et couple) permettant d'atteindre ces performances.
\end{obj}

\subparagraph{}
\textit{Quelle est la vitesse maximale que l'axe peut atteindre en  $\text{m}\, \text{s}^{-1}$.}
\ifprof
\begin{corrige}
$V = 0,83 \, \text{ms}^{-1}$
\end{corrige}
\else
\fi

\subparagraph{}
\textit{Combien de temps l'axe met-il pour atteindre la vitesse maximale ?}
\ifprof
\begin{corrige}
$T_a =0,83/9,8 = 0,08 s$
\end{corrige}
\else
\fi

\subparagraph{}
\textit{Quelle distance l'axe parcourt-il pour atteindre la vitesse maximale ?}
\ifprof
\begin{corrige}
\end{corrige}
\else
\fi


\subparagraph{}
\textit{Quelle est la longueur minimale à commander pour que l'axe puisse atteindre la vitesse maximale ?}
\ifprof
\begin{corrige}
\end{corrige}
\else
\fi

\subparagraph{}
\textit{Proposer une longueur minimale de l'axe pour pouvoir profiter de ses performances dynamiques.}
\ifprof
\begin{corrige}
\end{corrige}
\else
\fi


\subparagraph{}
\textit{Tracer le profil de la position, de la vitesse et de l'accélération pour parcourir une distance de 50 cm. On cherchera à atteindre les performances maximales de l'axe. }
\ifprof
\begin{corrige}
\end{corrige}
\else
\fi


Un motoréducteur permet d'entraîner un système poulie -- courroie permettant de déplacer la charge. On considère :
\begin{itemize}
\item une charge de masse $1\; \text{kg}$;
\item un poulie de rayon $5\; \text{cm}$;
\item un réducteur de rapport de transmission $1:20$.
\end{itemize}

\begin{center}
\includegraphics[width=.9\linewidth]{images/fig_12}
\end{center}

\subparagraph{}
\textit{Déterminer le couple à fournir par la poulie pour déplacer la charge lorsque l'accélération est au maximum. }
\ifprof
\begin{corrige}
\end{corrige}
\else
\fi


\subparagraph{}
\textit{Déterminer la vitesse et le couple à fournir par le moteur en considérant que l'inertie du motoréducteur est négligeable. }
\ifprof
\begin{corrige}
\end{corrige}
\else
\fi


\subparagraph{}
\textit{Donner la méthode permettant de prendre en compte l'inertie $J$ du motoréducteur ? Quel serait l'impact de la prise en compte de cette hypothèse ? }
\ifprof
\begin{corrige}
\end{corrige}
\else
\fi

%
%\section*{Exercice 4}
%
%\setcounter{subparagraph}{0}
%On donne la structure suivante : 
%\begin{center}
%\includegraphics[width=.8\linewidth]{images/fig_21}
%\end{center}
%
%
%\subparagraph{}
%\textit{Déterminer $\vect{\mathcal{M}\left(A,\vect{F} \right)}$.}
%
%
%
%On donne la structure suivante : 
%\begin{center}
%\includegraphics[width=.8\linewidth]{images/fig_22}
%\end{center}
%
%
%\subparagraph{}
%\textit{Déterminer $\vect{\mathcal{M}\left(O,\vect{F} \right)}$.}

\end{multicols}

\end{document}

\subparagraph{}\textit{}
\ifprof
\begin{corrige}~\\
\end{corrige}
\else
\fi




\subparagraph{}\textit{}
\ifprof
\begin{corrige}~\\
\end{corrige}
\else
\fi

\subparagraph{}\textit{}
\ifprof
\begin{corrige}~\\
\end{corrige}
\else
\fi

\subparagraph{}\textit{}
\ifprof
\begin{corrige}~\\
\end{corrige}
\else
\fi

\subparagraph{}\textit{}
\ifprof
\begin{corrige}~\\
\end{corrige}
\else
\fi

\subparagraph{}\textit{}
\ifprof
\begin{corrige}~\\
\end{corrige}
\else
\fi

\subparagraph{}\textit{}
\ifprof
\begin{corrige}~\\
\end{corrige}
\else
\fi
\begin{center}
%\includegraphics[width=\linewidth]{images/fig_05}
\end{center}
