\documentclass[10pt,fleqn]{article} % Default font size and left-justified equations
\usepackage[%
    pdftitle={Modélisation dynamique},
    pdfauthor={Xavier Pessoles}]{hyperref}

\input{style/new_style}
\input{style/macros_SII}

\fichetrue
\fichefalse

\proftrue
%\proffalse

%\tdtrue
\tdfalse

\courstrue
%\coursfalse



% -------------------------------------
% Déclaration des titres
% -------------------------------------

\def\discipline{Sciences \\Industrielles de \\ l'Ingénieur}
\def\xxtete{Sciences Industrielles de l'Ingénieur}

\def\classe{\textsf{PSI$\star$ -- MP}}
\def\xxnumpartie{Cycle 04}
\def\xxpartie{Modéliser le comportement des systèmes mécaniques dans le but d'établir une loi de comportement ou de déterminer des actions mécaniques en utilisant le PFD}

\def\xxnumchapitre{Chapitre 1 \vspace{.2cm}}
\def\xxchapitre{\hspace{.12cm} Introduction à la dynamique du solide indéformable}

\def\xxposongletx{2}
\def\xxposonglettext{1.45}
\def\xxposonglety{19}%16

\def\xxonglet{Cycle 04}

\def\xxactivite{Cours}
\def\xxauteur{\textsl{Xavier Pessoles}}

\def\xxcompetences{%
\textsl{%
\textbf{Savoirs et compétences :}\\
\begin{itemize}[label=\ding{112},font=\color{ocre}] 
\item \textit{Res1.C2} : principe fondamental de la dynamique;
\item \textit{Res1.C1.SF1} : proposer une démarche permettant la détermination de la loi de mouvement.
\end{itemize}
}}
		
\def\xxfigures{
%\includegraphics[width=1.4\textwidth]{images/matlab}%images/prot_01
%\\
%\textit{Modèle du pilote hydraulique avec pilotage interactif.}
}%figues de la page de garde

\def\xxpied{%
Cycle 04 -- Modéliser le comportement des systèmes mécaniques\\% afin de valider leurs performances.\\
Chapitre 1 -- \xxactivite%
}

\setcounter{secnumdepth}{5}
%---------------------------------------------------------------------------


\begin{document}
\chapterimage{png/Fond_CIN}
\input{style/new_pagegarde}
\setlength{\columnseprule}{.1pt}

\vspace{2cm}
\pagestyle{fancy}
\thispagestyle{plain}


\section{Introduction et hypothèses}

\subsection{Buts et motivations}
Le \textbf{principe fondamental de la dynamique} (PFD) est une théorie clé de la mécanique classique. Ce principe fut énoncé pour la première fois par Isaac Newton au $XVII$\up{e} siècle. Ce principe fait le lien entre l'étude des mouvements des solides (\textbf{cinématique}) associés à leurs inerties (\textbf{cinétique}) et leurs causes (\textbf{actions mécaniques}). Grâce à cet outil, on peut alors prévoir avec une excellente précision les phénomènes mécaniques classiques. Dans le cadre de l'étude de systèmes, on utilisera les \textbf{théorèmes généraux de la dynamique} pour obtenir des \textbf{équations de mouvement}.


\subsection{Référentiel galiléen}

\begin{definition}[Référentiel galiléen]
Un \textbf{référentiel galiléen} se définit à partir d'une repère spatial (orthonormée direct $\quadruplet{O_g}{\overrightarrow{x_g}}{\overrightarrow{y_g}}{\overrightarrow{z_g}}$ et d'une base de temps ($t$) et est animé d'un mouvement de \textbf{translation rectiligne uniforme} (à vitesse constante) par rapport à un référentiel absolu fixe ou à un autre référentiel galiléen $\quadruplet{O}{\overrightarrow{x_0}}{\overrightarrow{y_0}}{\overrightarrow{z_0}}$. 

On peut également le définir comme un référentiel ``\textit{dans lequel le principe fondamental de la dynamique s'applique}''.
\end{definition}

\begin{rem}[Référentiels supposés galiléens]
Dans la pratique, on fera toujours la \textbf{supposition qu'un repère est galiléen}. Cela dépendra effectivement des mouvements mis en jeu et des\textbf{ échelles temporelles et spatiales} considérées. 
Par exemple :
\begin{itemize}
\item pour étudier des mouvements de l'ordre de quelques minutes à l'échelle humaine, le \textbf{référentiel terrestre} (origine liée au centre de la terre et les trois axes liés au globe terrestre) est approprié;
\item pour étudier les effets météorologiques (ouragans, courants marins), ou les mouvements des satellites, il convient alors de tenir compte de l'inertie de la terre et on pourra choisir le \textbf{référentiel géocentrique} (origine liée au centre de la terre et les trois axes dirigés vers trois étoiles très éloignées) comme référentiel galiléen.
\item pour étudier le mouvement des planètes, il convient mieux d'utiliser le \textbf{référentiel héliocentrique} (origine liée au centre du soleil et les trois axes dirigés vers trois étoiles très éloignées).
\end{itemize}

Une chronologie galiléenne est obtenue par une horloge précise (Quartz, atomique, ou mouvement des astres).
En mécanique classique (ou Newtonienne), les deux repères d'\textbf{espace et de temps} sont supposés \textbf{indépendants} ce qui n'est pas le cas de la mécanique relativiste. 
\end{rem}

\begin{figure}[ht!]
\begin{center}
\includegraphics[width=0.6\textwidth]{images/base_galileenne.pdf}
\end{center}
\caption{Définition d'un référentiel galiléen \label{fig:ref_galileen}}
\end{figure}

\subsection{Présentation du support du cours du cours}



\begin{exemple}[Système de dépose de composants électroniques]
Le système étudié permet de déposer automatiquement des composants électroniques sur un circuit.
On s'intéresse ici à la modélisation d'un seul axe (selon la direction notée $\overrightarrow{y}_0$). actionné par un moteur électrique et utilisant un mécanisme de transformation de mouvement.

\textbf{Hypothèses :}
\begin{itemize}
\item le référentiel associé au repère $R_0=\quadruplet{O_0}{\overrightarrow{x}_0}{\overrightarrow{y}_0}{\overrightarrow{z}_0}$ est supposé galiléen ;
\item les solides seront supposés indéformables ; 
\item on notera $J_1$ le moment d'inertie du solide 1 selon l'axe $\couple{O_0}{\overrightarrow{y}_0}$ : $J_1=I_{\couple{O_0}{\overrightarrow{y}_0}}(S_1)$ ;
\item on note $M_3$ et $G_3$ respectivement la masse et le centre d'inertie du solide $S_3$ ;
\item la position de $G_3$ est définie par $\overrightarrow{O_0G_3}=x\cdot \overrightarrow{x}_0+y\cdot \overrightarrow{y}_0+z\cdot \overrightarrow{z}_0$
\item les liaisons sont supposées parfaites (sans jeu ni frottement).
\end{itemize}

\begin{center}
\begin{tabular}{cc}
\includegraphics[width=0.45\textwidth]{images/axe_y_photo.png}
&
\includegraphics[width=0.45\textwidth]{images/schema_cine_depose_composant.pdf}
\end{tabular}
\includegraphics[width=0.7\textwidth]{images/req_systeme_depose.pdf}
\end{center}
\begin{itemize}
\item $S_0$ : poutre transversale considérée comme fixe par rapport au bâti ;
\item $S_1$ : vis à billes (hélice à droite) ;
\item $S_2$ : écrou de la vis à billes ;
\item $S_3$ : chariot supportant la tête de dépose (masse $M_3$) ;
\end{itemize}
\end{exemple}

\begin{obj}
L'objectif de cette étude est de relier les grandeurs liées à l'actionneur du système (moteur) :
\begin{itemize}
\item couple transmis à $S_1$ : $\overrightarrow{C}_{Moteur\to S_1}$ ;
\item vitesse de rotation de $S_1$ : $\overrightarrow{\Omega}(S_1/R_0)\cdot \overrightarrow{y}_0=\dot{\theta}$.
\end{itemize} 
à celles liée à l'effecteur (tête de dépose $S_3$) : 
\begin{itemize}
\item masse : $M_3$;
\item cinématique de $S_3$ : $\overrightarrow{a}(G_3\in S_3/R_0)\cdot \overrightarrow{y}_0=\ddot{y}$.
\end{itemize}
\end{obj}



\section{Principe Fondamental de la Dynamique : applications simplifiées}

\subsection{Cas particulier d'un solide en translation}

\begin{definition}[Solide en translation par rapport à un référentiel galiléen]
Si un ensemble matériel $E$ (de centre d'inertie $G$) est en mouvement de translation dans un référentiel galiléen ($R_g$) alors : 

\begin{itemize}
\item \textbf{Théorème de la résultante dynamique : } la résultante des efforts extérieurs est égale au produit de la masse par l'accélération de G par rapport à $R_g$ :

\begin{align}
\boxed{
m\;\overrightarrow{a}(G/R_g)=\vectf{\bar E}{E}%\resultante[\bar E]{E}
}
\end{align}

\item \textbf{Théorème du moment dynamique : } le moment des actions mécaniques extérieures s'appliquant sur $E$ est égal au vecteur nul en tout point :

\begin{align}
\boxed{
%\moment[A]{\bar E}{E}=\overrightarrow{0}\forall A
\vectm{A}{\bar E}{E}=\overrightarrow{0}\forall A
}
\end{align}
\end{itemize}

\end{definition}

\begin{exemple}[Machine de dépose de composants électroniques : déplacement dynamique de $S_3$]
\begin{minipage}{0.35\textwidth}
On connaît la masse $M_3$ de la tête de dépose et on cherche l'effort ($\overrightarrow{R}_{poussée\to S_3}$) de poussée que doit fournir l'actionneur pour obtenir l'accélération souhaitée.
\end{minipage}
\begin{minipage}{0.55\textwidth}
\begin{center}
\includegraphics[width=1.0\textwidth]{images/schema_cine_depose_translation.pdf}
\end{center}
\end{minipage}

On utilise le théorème de la résultante dynamique en projection sur $\overrightarrow{y}_0$. On obtient : 
%\begin{texteCache}
\begin{align*}
M_3\frac{d^2\overrightarrow{a}(G_3/R_0)}{dt^2}\cdot \overrightarrow{y}_0=\sum \overrightarrow{R}_{ext\to S_3}\cdot \overrightarrow{y}_0.\\
\\
M_3\cdot \ddot{y}=R_{poussee\to S_3}
\end{align*} 
%\end{texteCache}

\textbf{Application numérique : }
Détermination de $R_{poussee\to S_3}$ pour obtenir une accélération de $4m/s^2$ :

%\begin{texteCache}
\begin{align*}
R_{poussee\to S_3}=20\times 4=80N
\end{align*}
%\end{texteCache}
\end{exemple}

\subsection{Cas d'un solide en mouvement de rotation autour d'un axe fixe}


\begin{definition}[Solide en rotation autour d'un axe fixe par rapport à un référentiel galiléen]
Si un ensemble matériel $E$ (de centre d'inertie $G$) est en mouvement de rotation autour d'un axe $\Delta$ (dirigé par $\overrightarrow{u}$ unitaire) fixe dans un référentiel galiléen ($R_g$) alors : 

\begin{itemize}
\item \textbf{Théorème du moment dynamique : } Le moment des actions mécaniques extérieures s'appliquant sur $E$ est égal au vecteur nul en tout point de l'axe :

\begin{align}
\boxed{
\vectm{A}{\bar E}{E}\cdot \overrightarrow{u}=J_{\Delta}(E)\cdot \ddot{\theta}\;\;\forall A\in \Delta
}
\end{align}
\item avec $J_{\Delta}(E)$ le moment d'inertie de $E$ par rapport à l'axe $\Delta$ (en $kg\cdot m^2$) ;
\item avec $\ddot{\theta}$, l'accélération angulaire de $E$ par rapport à $R_g$ suivant $\Delta$ : $\overrightarrow{\Omega}(E/R_g)\cdot \overrightarrow{u}$.
\end{itemize}
\end{definition}

\begin{exemple}[Machine de dépose de composants électroniques : déplacement dynamique de $S_3$]
\begin{minipage}{0.5\textwidth}
On donne les caractéristiques du moteur entraînant l'axe et la vis $S_1$ :
\begin{itemize}
\item couple maximal, $C_{max} = 21,2 N\cdot m$ ;
\item fréquence de rotation maximale, $N_m = 6000 tr/min$
\item moment d'inertie du moteur suivant l'axe $\overrightarrow{y}_0$ : $I_m = 1,6 \times 10^{-4} kg.m^2$ ;
\item moment d'inertie de la vis à billes suivant l'axe $\overrightarrow{y}_0$ : $I_v = 2,15 x\times 10^{-4} kg\cdot m^2$.
\end{itemize}
De plus on notera : $\overrightarrow{\Omega}(S_1/R_0)=\dot{\theta}(t)\cdot \overrightarrow{y}_0$
\end{minipage}
\begin{minipage}{0.45\textwidth}
\begin{center}
\includegraphics[width=1.0\textwidth]{images/schema_cine_depose_rotation.pdf}
\end{center}
\end{minipage}

\textbf{Détermination des caractéristiques maximales : }

On se place de la cas le plus limite (Couple maximal, accélération angulaire constante pour atteindre la fréquence de rotation maximale en $t_a=0,2s$)
Déterminer le couple résistant maximal que le moteur peut équilibrer dynamiquement ($C_{S_2\to S_1}$):

%\begin{texteCache}
On appliquant un théorème du moment dynamique à $S_1$ selon $\couple{O_0}{\overrightarrow{y}_0}$
\begin{align*}
(I_m+I_v)\cdot \ddot{\theta}=C_{max}+C_{S_2\to S_1}
\end{align*}
On obtient alors : 

\begin{align*}
C_{S_2\to S_1}=(I_m+I_v)\cdot \ddot{\theta}_{max}-C_{max}=(I_m+I_v)\cdot \frac{N_m\times 2\cdot \pi}{60\cdot t_a}-C_{max}=-20N\cdot m
\end{align*}

%\end{texteCache}
\end{exemple}

%\begin{bilan}
\begin{itemize}
\item Les deux cas présentés ci-dessus sont traités de manière indépendante.
\item Le lien entre ces deux parties (actionneur et effecteur) repose sur le mécanisme de transformation de mouvement (ici vis-écrou).
\item Il faudra donc procéder à une démarche de résolution globale pour relier le couple moteur $\overrightarrow{C}_{moteur\to S_1}$ à l'accélération ($\overrightarrow{a}(G_3/R_0)$) et la masse de l'effecteur ($M_3$). Ce sera l'objet des parties suivantes.
\end{itemize}

%\end{bilan}

\section{Principe Fondamental de la Dynamique : cas général}

\subsection{Principe Fondamental de la Dynamique}

\begin{definition}[Énoncé du Principe Fondamental de la Dynamique]
Dans le cas général, soit un ensemble matériel $E$ en mouvement par rapport à un référentiel galiléen ($R_0$), alors la somme des actions mécaniques extérieures (\textbf{Torseur des actions mécaniques extérieures} s'appliquant sur $E$ est égale au \textbf{torseur dynamique} du mouvement de $E$ par rapport à $R_0$ :

\begin{align}\label{eqn:pfd}
\boxed{
\torseurstat{\overline{E}}{E}=\torseurcin{D}{E}{R_0}
}
\end{align}

De plus le \textbf{Principe Fondamental de la Dynamique} postule que pour tout mouvement, il existe au moins un référentiel dans lequel la relation \ref{eqn:pfd} est vérifiée. Ce sera donc un \textbf{référentiel galiléen}.

\end{definition}

\begin{rem}
\begin{itemize}
\item Les démarches pour le calcul du torseur dynamique seront vues dans le cours C3-2.
\item La démarche de calcul du torseur des actions mécaniques extérieures appliquées sur E est la  même que celle vu lors de l'utilisation du PFS (ce sont les mêmes torseurs). 
\end{itemize}
\end{rem}

\subsection{Équations de mouvement}

\begin{definition}[Équations de mouvement]
Une \textbf{équation de mouvement} est une équation différentielle du second ordre traduisant les théorèmes généraux, dans laquelle ne figure \textbf{aucune composante inconnue d'action mécanique}. 

Il est parfois nécessaire d'écrire plusieurs équations pour trouver par substitution une équation de mouvement. 

On nomme \textbf{« Intégrale première du mouvement »} une équation différentielle du premier ordre avec un second membre constant, obtenue par  intégration d'une équation de mouvement. 
\end{definition}

\subsection{Théorèmes généraux}

Du Principe Fondamental de la dynamique découle plusieurs théorèmes généraux.
\subsubsection{Théorème de la résultante dynamique}

		\begin{theorem}[de la résultante dynamique]
			Pour tout ensemble matériel $(E)$ de masse $m$ et de centre de gravité $G$ en mouvement par rapport à un référentiel galiléen ($R_0$), la somme des résultantes des efforts extérieurs s'appliquant sur $E$ est égale à la résultante dynamique du mouvement de $E$ par rapport à $R_0$ :
			
			\begin{align}
			\boxed{
				\vectf{\bar E}{E}=\overrightarrow{R_d}(E/R_0)=m\;\overrightarrow{a}(G/R_0).
			}			
			\end{align}
		\end{theorem}


\begin{rem}
On peut alors définir un Newton comme l'effort à mettre en oeuvre pour mettre en mouvement $1kg$ avec une accélération de $1\;m\cdot s^{-2}$ en son centre de gravité $G$.
\end{rem}
\subsubsection{Théorème du moment dynamique}
\begin{theorem}[du moment dynamique]
			Pour tout ensemble matériel $(E)$ de masse $m$ en mouvement par rapport à un référentiel galiléen ($R_0$), la somme des moments des efforts extérieurs s'appliquant sur $E$ en un point quelconque $A$ est égale au moment dynamique du mouvement de $E$ par rapport à $R_0$ en $A$ :
			
			\begin{align}
			\boxed{
				\vectm{A}{\bar E}{E}=\vectmd{A}{E}{R_0}.
			}			
			\end{align}
		\end{theorem}

\subsubsection{Théorème des actions mutuelles}
\begin{theorem}[des actions mutuelles]
			Soient $(E_1)$ et $(E_2)$ deux sous-ensembles matériels de $(E)$,
			en mouvement par rapport à un référentiel galiléen, et exerçant une action mécanique l'un sur l'autre.
			Alors :
			\begin{align}
			\boxed{
				\torseurstat{T}{E_1}{E_2}=-\torseurstat{T}{E_2}{E_1}
				}
			\end{align}
		\end{theorem}
		



\section{Objectif et Méthodologie}

\subsection{Objectifs}
On distingue deux principaux types de problèmes en dynamique : 
\begin{itemize}
\item \textbf{Type 1} :
\begin{itemize}
\item on connaît : les actionneurs et les inerties ;
\item on détermine : les lois de mouvement et les actions mécaniques dans les liaisons.
\end{itemize} 
\item \textbf{Type 2} :
\begin{itemize}
\item on connaît : les lois de mouvement et inerties ;
\item on détermine : les caractéristiques des actionneurs et les actions mécaniques de liaison.
\end{itemize}
\end{itemize}

\subsection{Méthodologie}
La méthodologie de résolution d'un problème de dynamique est très similaire à celle utilisée lors de la détermination des performances statiques des systèmes.

\begin{enumerate}
\item On choisit un repère galiléen et on effectue le bilan complet des données d'entrée du problème.
\item On construit un graph de structure.
\item On isole le solide ou le système de solides considéré.
\item On effectue le Bilan des Actions Mécaniques Extérieures agissant sur le système isolé.
\item On écrit le PFD.
\item On projette les relations vectorielles sur les axes choisis.
\item On injecte les lois de comportement (ressort, lois de Coulomb, ...).
\item On effectue la résolution.
\end{enumerate}



\begin{thebibliography}{2}
   \bibitem[1]{ref1} Emilien Durif, {\it Introduction à la dynamique des solides, Lycée La Martinière Monplaisir, Lyon.}
      \bibitem[2]{ref2} Florestan Mathurin, {\it Correction des SLCI, Lycée Bellevue, Toulouse, \url{http://florestan.mathurin.free.fr/}.}



\end{thebibliography}

\end{document}



