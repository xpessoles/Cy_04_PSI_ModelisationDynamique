%%%% Paramétrage du TD %%%%
\def\xxactivite{Activation 1 \ifprof -- Corrigé \else \fi} % \normalsize \vspace{-.4cm}
\def\xxauteur{\textsl{Xavier Pessoles}}


\def\xxnumchapitre{Chapitre 1 \vspace{.2cm}}
\def\xxchapitre{\hspace{.12cm} Introduction à la dynamique du solide indéformable}

\def\xxcompetences{%
\textsl{%
\textbf{Savoirs et compétences :}\\
\begin{itemize}[label=\ding{112},font=\color{ocre}] 
\item \textit{Res1.C2} : principe fondamental de la dynamique;
\item \textit{Res1.C1.SF1} : proposer une démarche permettant la détermination de la loi de mouvement.
\end{itemize}
}}

\def\xxtitreexo{Système mobile d’imagerie interventionnelle Discovery IGS 730}%Motorisation du moteur Haibike}
\def\xxsourceexo{\hspace{.2cm} \footnotesize{CCP MP 2017}}

%\def\xxauteur{\textsl{Xavier Pessoles}}

\def\xxfigures{
\includegraphics[width=.6\linewidth]{fig_02}
}%figues de la page de garde


\iflivret
\input{../../style/new_pagegarde}
\else
\input{../../style/new_pagegarde}
\fi
\setlength{\columnseprule}{.1pt}

\pagestyle{fancy}
\thispagestyle{plain}

\ifprof
\vspace{4.9cm}
\else
\vspace{4.5cm}
\fi

\def\columnseprulecolor{\color{ocre}}
\setlength{\columnseprule}{0.4pt} 

%%%%%%%%%%%%%%%%%%%%%%%

\setcounter{exo}{0}



\ifprof
%\begin{multicols}{2}
\else
\begin{multicols}{2}
\fi

Le Discovery IGS 730 est le premier système
mobile d’imagerie interventionnelle. embarquant un ensemble de logiciels de traitement d’images
pour les applications vasculaires, l’oncologie et la cardiologie et permettant un accès complet
au patient, il guide les gestes de l’équipe médicale tout au long de l’intervention chirurgicale. Le Discovery IGS 730 dispose d’une base motorisée guidée par laser qui transporte l’arceau d’imagerie.
%
%Le système Discovery IGS 730 est constitué principalement :
%\begin{itemize}
%\item d’une base motorisée, aussi appelée AGV (pour Automated Guided Vehicle, soit véhicule à
%guidage automatique) ;
%\item d’une perche et d’un support de câbles ;
%\item du sous-système d’imagerie supporté par un bras en « C » ou arceau. Le système d’imagerie
%est lié à la base motorisée par l’intermédiaire de deux liaisons pivot. Un point caractéristique
%appelé « isocentre » (point IC) est rattaché au sous-système d’imagerie. Il est défini comme
%l’intersection de l’axe optique et de l’axe de la liaison pivot AGV/système pivot.
%\end{itemize}

%\begin{center}
%\includegraphics[width=.95\linewidth]{images/fig_02}
%\end{center}

La base motorisée AGV est constituée :
\begin{itemize}
\item d’une structure support, ou châssis, composée du bras vertical et du cadre Y;
\item de deux sous-ensembles roue motrice et motorisation associée (un motoréducteur d’orientation
et un motoréducteur de propulsion pour chaque roue) ;
\item de deux doubles roues « folles » non motorisées.
\end{itemize}

\begin{center}
\includegraphics[width=.95\linewidth]{fig_03}
\end{center}

\begin{obj}
Déterminer les valeurs de décélérations maximales en cas d’arrêt d’urgence garantissant la
sécurité du patient et du personnel lors d’une man\oe{}uvre de translation.
\end{obj}

Une procédure de freinage d’urgence est prévue pour immobiliser le système au plus tôt lorsqu’un
problème est rencontré au cours de son déplacement. Pour des raisons évidentes de sécurité, le système
doit s’arrêter le plus rapidement possible sans toutefois que la décélération n’entraîne le basculement
de l’engin ou encore du glissement au niveau des roues motrices.


\subsection*{Modèle retenu pour l’étude et paramétrage associé}

Le problème est supposé plan et se ramène à l’étude
du système composé des éléments :
\begin{itemize}
\item \textbf{(1)} : roue motrice ;
\item \textbf{(2)} : roue folle ;
\item \textbf{(3)} : cadre et arceau d’imagerie.
\end{itemize}
Le repère $\mathcal{R}$ ayant pour origine le point $O$ et muni de la
base $\base{x}{y}{z}$ est attaché à la salle d’intervention \textbf{(0)}.
Le référentiel associé est supposé galiléen.
Le système est animé d’un mouvement de translation
suivant $\vect{x}$ (sens 1) tel que $\vectv{G}{\Sigma}{\mathcal{R}}= u(t)\vect{x}$ , avec
$u(t) > 0$.
Lors de ce mouvement, les roues sont animées d’un
mouvement de rotation. L’angle de rotation d’une roue
par rapport à 
$\mathcal{R}$est définir par :
$\theta_R = \left(\vect{x},\vect{x_R}\right) = \left(\vect{z},\vect{z_R}\right)$.

\begin{center}
\includegraphics[width=.8\linewidth]{fig_04}
\end{center}



\begin{hypo}
La roue motrice \textbf{(1)} roule sans glisser sur le sol \textbf{(0)} au point $I_1$.
La résistance au roulement est négligée.
\end{hypo}

\subsection*{Caractéristiques géométriques et d’inertie des solides}
\begin{itemize}
\item Ensemble du système $\Sigma$ : centre d’inertie $G$ tel que $\vect{I_1G} = x_G \vect{x} +z_G \vect{z}$, avec $x_G=\SI{450}{mm}$ et $z_G=\SI{950}{mm}$, masse : $m_{\Sigma}=\SI{840}{kg}$.
\item Roue motrice \textbf{(1)} : rayon $r = \SI{115}{mm}$, centre d'inertie $A$,  $J$ représente le moment d’inertie par rapport à l’axe $\left(A,\vect{y}\right)$.
\item Empattement du système : $I_1I_2 =\ell  = \SI{1}{m}$.
\end{itemize}

\subsection*{Actions mécaniques}

Le système $\Sigma$ est soumis aux actions mécaniques suivantes :
\begin{itemize}
\item action du sol sur la roue motrice \textbf{(1)} : 

$\torseurstat{T}{0}{1}=\torseurl{\vectf{0}{1}=-X_{01}\vect{x}+Z_{01}\vect{z}}{\vect{0}}{I_1}$ et sur la roue folle \textbf{(2)} :

$\torseurstat{T}{0}{2}=\torseurl{\vectf{0}{2}=Z_{02}\vect{z}}{\vect{0}}{I_2}$;
\item action de la pesanteur : 

$\torseurstat{T}{\text{pes}}{\Sigma}=\torseurl{\vectf{\text{pes}}{\Sigma}=-m_{\Sigma} g \vect{z}}{\vect{0}}{G}$;
\item action de freinage sur la roue motrice \textbf{(1)}: $\torseurstat{T}{\text{frein}}{1}=\torseurl{\vect{0}}{\vectm{A}{\text{frein}}{1}=-C_f\vect{y}}{A}$.
\end{itemize}


\subsection*{Condition de non basculement}

Au cours du freinage, le système subit une décélération $\vectg{G}{\Sigma}{\mathcal{R}}=-\gamma \vect{x}=r\ddot{\theta} \vect{x}$, avec $\ddot{\theta}<0$. 
Afin de s’assurer du non-basculement du système, il est nécessaire de déterminer la valeur de décélération limite $\gamma_{\text{NB}}$ pour laquelle apparaît un décollement de la roue motrice.


\subparagraph{}\textit{Par application du théorème de la résultante dynamique à l'ensemble $\Sigma$ suivant l'axe en mouvement,
déterminer l'expression de la composante tangentielle $X_{01}$ appliquée à la roue motrice \textbf{(1)} en fonction de la décélération $\gamma$.}

\subparagraph{}\textit{Par application du théorème du moment dynamique à la roue motrice \textbf{(1)} suivant l’axe $\left(A,\vect{y}\right)$ et en utilisant la relation établie à la question précédente, déterminer l’expression du couple de
freinage $C_f$ en fonction de la décélération $\gamma$.}


Dans la suite, le moment d’inertie $J$ est négligé devant le terme $m_{\Sigma}r^2$ associé à la masse de l’ensemble.

\subparagraph{}\textit{Simplifier alors l'expression établie à la question précédente.}
\subparagraph{}\textit{Déterminer l'expression du moment dynamique de l'ensemble $\Sigma$ par rapport à $\mathcal{R}$ au point $I_2$.}
\subparagraph{}\textit{Par application du théorème du moment dynamique en $I_2$, déterminer la relation liant l'accélération $\gamma$ et la composante normale $Z_{01}$. En déduite l'expression de la décélération limite $\gamma_{\text{NB,1}}$ pour laquelle $Z_{01}=0$. Réaliser l'application numérique.}


\subsection*{Condition de non-glissement}

L'exigence de sécurité conduit également à prévenir le glissement au niveau des roues motrices. Il est pour cela nécessaire de déterminer la valeur de la décélération limite $\gamma_{\text{NG}}$ correspondant à une perte d'adhérence. Le facteur d'adhérence au niveau du contact roue motrice/sol (assimilé ici au facteur de frottement) est noté $\mu$.

\subparagraph{}\textit{En se plaçant à la limite du glissement, déterminer à partir des expressions établies aux 5 questions précédentes (avant la simplification $Z_{01}=0$) l'expression de la décélération limite $\gamma_{\text{NG},1}$. Réaliser l'application numérique pour un facteur d'adhérence $\mu=0,5$.}

\subsection*{Synthèse}
Une étude similaire à celle qui vient d'être menée conduit dans le cas d'une translation suivant $\vect{-x}$ (sens 2) aux valeurs limites $\gamma_{\text{NB},2}\SI{4,5}{m.s^{-2}}$ et $\gamma_{\text{NG},2}\SI{5,5}{m.s^{-2}}$.

\subparagraph{}\textit{En réalisant la synthèse de l'ensemble de ces résultats, conclure sur la valeur de décélération limite à retenir pour satisfaire l'exigence de sécurité lors d'un mouvement de translation du système.}

\ifprof
%\end{multicols}
\else
\end{multicols}
\fi

%\end{document}