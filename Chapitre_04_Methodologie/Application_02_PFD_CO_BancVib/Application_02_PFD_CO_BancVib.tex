\documentclass[10pt,fleqn]{article} % Default font size and left-justified equations
\usepackage[%
    pdftitle={Modélisation dynamique : cinétique},
    pdfauthor={Xavier Pessoles}]{hyperref}

    
\input{style/new_style}
\input{style/macros_SII}

\fichetrue
%\fichefalse

\proftrue
\proffalse

\tdtrue
%\tdfalse

\courstrue
\coursfalse


\def\discipline{Sciences \\Industrielles de \\ l'Ingénieur}
\def\xxtete{Sciences Industrielles de l'Ingénieur}

\def\classe{\textsf{PSI$\star$ -- MP}}
\def\xxnumpartie{Cycle 04}
\def\xxpartie{Modéliser le comportement des systèmes mécaniques dans le but d'établir une loi de comportement ou de déterminer des actions mécaniques en utilisant le PFD}

\def\xxnumchapitre{Chapitre 4 \vspace{.2cm}}
\def\xxchapitre{\hspace{.12cm} Méthodologie : détermination des équations de mouvement}




\def\xxtitreexo{Chaîne ouverte -- Banc d'essai vibrant}
\def\xxsourceexo{\hspace{.2cm} \footnotesize{Pôle Chateaubriand -- Joliot Curie}}


\def\xxposongletx{2}
\def\xxposonglettext{1.45}
\def\xxposonglety{20}
%\def\xxonglet{Part. 1 -- Ch. 3}
\def\xxonglet{Cycle 04}

\def\xxactivite{Application 02}
\def\xxauteur{\textsl{Pôle Chateaubriand -- Joliot Curie}}

\def\xxcompetences{%
\textsl{%
\textbf{Savoirs et compétences :}\\
%\begin{itemize}[label=\ding{112},font=\color{ocre}] 
%\item \textit{Mod2.C13} : centre d'inertie
%\item \textit{Mod2.C14} : opérateur d'inertie
%\item \textit{Mod2.C15} : matrice d'inertie
%\end{itemize}
}}
\def\xxfigures{
\includegraphics[width=.35\linewidth]{images/fig_00}
}%figues de la page de garde


\def\xxpied{%
Cycle 04 -- Modélisation mécanique -- Cinétique\\% afin de valider leurs performances.\\
Chapitre 4 -- \xxactivite%
}

\setcounter{secnumdepth}{5}
%---------------------------------------------------------------------------

\usepackage{pgfplots}
\begin{document}
\def\pathfig{images}
%\chapterimage{png/Fond_Cin}
\input{style/new_pagegarde}
\vspace{5cm}
\pagestyle{fancy}
\thispagestyle{plain}

\def\columnseprulecolor{\color{ocre}}
\setlength{\columnseprule}{0.4pt} 

\def\pathfig{images}

\ifprof
\else
\begin{multicols}{2}
\fi

\subsection*{Présentation}
Les vibrations se retrouvent dans tous les systèmes et nuisent à leur durée de vie. On s’intéresse à un banc d’essai permettant d’étudier les conséquences de ces vibrations sur l’usure et la fatigue des pièces mécaniques.
La figure ci-après représente un modèle cinématique du dispositif étudié. Une modélisation plane a été retenue.
Le bâti vibrant est modélisé par un solide $S_1$, de masse $m_1$ en liaison glissière parfaite avec un support $S_0$, fixe par rapport à un repère $\mathcal{R}_0$ supposé galiléen.

Le solide $S_1$ est rappelé par un ressort de longueur libre $l_0$ et de raideur $k$.
Une masse ponctuelle $m_2$ excentrée, placée en $P$, tourne sur un rayon $r$ et est entraînée à vitesse constante $\Omega$. Elle modélise le 
balourd du rotor d’un moteur $S_2$.

Un pendule simple de longueur $L$, porte à son extrémité $D$ une masse concentrée $m_3$, l’ensemble constitue le solide $S_3$, en liaison pivot parfaite d’axe $\axe{C}{z_0}$ avec $S_1$.

Les masses autres que $m_1$, $m_2$ et $m_3$ sont négligées.



\begin{center}
\includegraphics[width=\linewidth]{images/fig_01}
\end{center}

\begin{obj}
Déterminer les conditions géométriques permettant de supprimer les vibrations.
\end{obj}


\subparagraph{}
\textit{Préciser les théorèmes à utiliser permettant de déterminer deux équations différentielles liant $x$, $\theta$ et leurs dérivées et les paramètres cinétiques et cinématiques utiles. Déterminer ces deux équations.}
\ifprof
\begin{corrige}
\end{corrige}
\else
\fi

On souhaite supprimer les vibrations du bâti vibrant. On recherche alors une solution du système d’équations différentielles
déterminé précédemment autour de la position d’équilibre $\left(x_0,\theta_0\right)=(0,0)$ en supposant que $x$, $\theta$, $\dot{x}$, $\dot{\theta}$ sont des petites variations
de position ou de vitesse autour de cette position d’équilibre.


\subparagraph{}
\textit{Proposer une linéarisation, à l’ordre 1, des deux équations différentielles précédentes.}
\ifprof
\begin{corrige}
\end{corrige}
\else
\fi

On s’intéresse uniquement au régime d’oscillations forcées. On cherche donc des solutions de la forme $x(t)=A\cos\left( \Omega t \right)$ et $\theta(t)=B\cos\left( \Omega t \right)$.


\subparagraph{}
\textit{Déterminer le système d’équations permettant de calculer $A$ et $B$.}
\ifprof
\begin{corrige}
\end{corrige}
\else
\fi



\subparagraph{}
\textit{Indiquer la condition que doit vérifier la longueur $L$ afin d’assurer $x(t) = 0$ en régime forcé.}
\ifprof
\begin{corrige}
\end{corrige}
\else
\fi



\vspace{1cm}
\paragraph*{Éléments de correction}
\footnotesize
\begin{enumerate}
\item $\left( m_1 + m_2 + m_3\right) \ddot{x} + kx + m_3 L\ddot{\theta} \cos \theta - m_3 L\dot{\theta}^2 \sin \theta = m_2 r \Omega^2 \cos \left( \Omega t \right)$ et $\ddot{x}\cos\theta+L\ddot{\theta}+g\sin\theta=0$.
\item $\left( m_1 + m_2 + m_3\right) \ddot{x} + kx + m_3 L\ddot{\theta}= m_2 r \Omega^2 \cos \left( \Omega t \right)$ et $\ddot{x}+L\ddot{\theta}+g\theta=0$.
\item $A=\dfrac{m_2 r \Omega^2 \left( -L\Omega^2 + g\right)}{\left[-\left( m_1 + m_2 + m_3\right)\Omega^2 +k\right]\left( -L\Omega^2 +g\right)-m_3L\Omega^4}$ et 
$B=\dfrac{m_2 r \Omega^2 }{\left[-\left( m_1 + m_2 + m_3\right)\Omega^2 +k\right]\left( -L\Omega^2 +g\right)-m_3L\Omega^4}$.
\item $L=\dfrac{g}{\Omega^2}$.
\end{enumerate}
\normalsize
\end{multicols}

\newpage
\begin{center}
\includegraphics[width=\linewidth]{images/cor_01}
\end{center}

\begin{center}
\includegraphics[width=\linewidth]{images/cor_02}
\end{center}

\begin{center}
\includegraphics[width=\linewidth]{images/cor_03}
\end{center}

\end{document}
\begin{center}
\includegraphics[width=\linewidth]{images/}
\end{center}

\subparagraph{}
\textit{}
\ifprof
\begin{corrige}
\end{corrige}
\else
\fi
