\documentclass[10pt,fleqn]{article} % Default font size and left-justified equations
\usepackage[%
    pdftitle={Résolutions de problèmes de statique : PFS 3D},
    pdfauthor={Xavier Pessoles}]{hyperref}

\input{style/new_style}
\input{style/macros_SII}
\usepackage{multicol}
\usepackage{siunitx}
%\usepackage{picins}
\fichetrue
%\fichefalse

\proftrue
%\proffalse

\tdtrue
%\tdfalse

\courstrue
\coursfalse

\newif\ifnormal
\normaltrue
%\normalfalse

\newif\ifdifficile
\difficilefalse
%\difficiletrue

\newif\iftdifficile
\tdifficilefalse
%\tdifficiletrue

% -------------------------------------
% Déclaration des titres
% -------------------------------------

\def\classe{\textsf{PSI$\star$ -- MP}}
\def\xxnumpartie{Cycle 04}
\def\xxpartie{Modéliser le comportement des systèmes mécaniques dans le but d'établir une loi de comportement ou de déterminer des actions mécaniques en utilisant le PFD}

\def\xxnumchapitre{Chapitre 4 \vspace{.2cm}}
\def\xxchapitre{\hspace{.12cm} Méthodologie : détermination des équations de mouvement}

\def\discipline{Sciences \\Industrielles de \\ l'Ingénieur}
\def\xxtete{Sciences Industrielles de l'Ingénieur}

\def\xxposongletx{2}
\def\xxposonglettext{1.45}
\def\xxposonglety{19}%16

\def\xxonglet{\textsf{Cycle 06}}

\def\xxactivite{TD 02}
\def\xxauteur{\textsl{Frédéric Sollner}}


\def\xxtitreexo{Dynamique d'un Segway de première génération \ifnormal $\star$ \else \fi \iftdifficile $\star\star\star$ \else \fi }
\def\xxsourceexo{\hspace{.2cm} \footnotesize{Frédéric SOLLNER -- Lycée Mermoz -- Montpellier}}

\def\xxcompetences{%
\textsl{%
\textbf{Savoirs et compétences :}\\
}}

\def\xxfigures{
\includegraphics[width=.3\textwidth]{images/fig_01}
}%figues de la page de garde

\def\xxpied{%
Cycle 04 -- Modélisation mécanique -- PFD\\% afin de valider leurs performances.\\
Chapitre 4 -- \xxactivite%
}


\setcounter{secnumdepth}{5}
%---------------------------------------------------------------------------


\begin{document}
%\chapterimage{png/Fond_Cin}
\input{style/new_pagegarde}
\vspace{4.5cm}
\pagestyle{fancy}
\thispagestyle{plain}


\def\columnseprulecolor{\color{ocre}}
\setlength{\columnseprule}{0.4pt} 

\ifprof
%\begin{multicols}{2}
\else
\begin{multicols}{2}
\fi

\section*{Présentation}
\ifprof
\else

Le support de l’étude est le véhicule auto balancé Segway®. Il s’agit d’un moyen de transport motorisé qui permet de se déplacer en ville. En termes de prestations, il est moins rapide qu’une voiture ou qu’un scooter, mais plus maniable, plus écologique, moins encombrant et nettement plus moderne.

	La première génération de Segway avait un guidon fixe et une poignée de direction). Cette technologie provoquait un effet de roulis qui pouvait conduire à un renversement. Dans cet exercice, nous nous proposons d’étudier le dérapage et le renversement d’un Segway de première génération.
	
	La seconde génération de Segway a vu apparaître une technologie appelée LeanSteer avec guidon inclinable qui permet de faire tourner le Segway lorsque l'utilisateur penche son corps sur le côté (non étudié dans cet exercice).

\begin{center}
\includegraphics[width=\linewidth]{images/fig_02}
%\textit{Modèle volumique 3D}
\end{center}


On donne les caractéristiques géométriques et cinématiques suivantes :
\begin{itemize}
\item la route \textbf{(0)} est munie du repère $\rep{0}=\repere{O}{x_0}{y_0}{z_0}$. Ce référentiel associé est supposé galiléen.
\item la plate-forme \textbf{(1)} a pour centre de gravité $C$. Le conducteur \textbf{(2)} a pour centre de gravité $G$. Les roues 3 et 4,de masse et inertie négligeable, sont liées à 1 par des liaisons pivots d'axe $\axe{C}{y_1}$. L’ensemble 
$E=1\cup 2$ forme le système matériel indéformable $E$ de centre de gravité $G_E$ et de masse $m_E$. Il est animée d'un mouvement de rotation par rapport au sol dont le centre instantané de rotation est $O$. Le rayon de courbure de la trajectoire du point $G_E$ dans $\rep{0}$ est $\rep{C}$. Le repère lié à 1 est $\rep{1}$  tel que $\vect{z_1}=\vect{z_0}=\vect{z_{01}}=$ et on note $\theta=\angl{x_0}{x_1}=\angl{y_0}{y_1}$. 
\end{itemize}


On donne $\vect{OG_E}=R_C\vect{y_1}+h\vect{z}_{01}$. L'opérateur d'inertie de $E$ en $G_E$ dans $\mathcal{B}_1=\base{x_1}{y_1}{z_1}$ est : 
$\inertie{G_E}{E}=\matinertie{A}{B}{C}{-D}{-E}{-F}{\mathcal{B}_1}$.

\begin{hypo}
\begin{itemize} 
\item Les contacts entre les roues 3 et 4 et la route 0 ont lieu en $A$ et $B$ définis par $\vect{G_EA}=-l\vect{y_1}-h\vect{z_0}$ et $\vect{G_EB}=l\vect{y_1}-h\vect{z_0}$, $l$ désignant la demi voie du véhicule. Les contacts sont modélisés par des liaisons sphère-plan de centres $A$ et $B$ et de normale $\vect{z_{01}}$. Le contact dans ces liaisons se fait avec un coefficient de frottement noté $f$ (on supposera pour simplifier que les coefficients de frottement et d'adhérence sont identiques). Les actions mécaniques de la route 0 sur les roues 3 et 4 sont modélisées par des glisseurs en $A$ et $B$ de résultantes $\vectf{0}{3}=-T_A\vect{y_1}+N_A\vect{z_1}$ et $\vectf{0}{4}=-T_B\vect{y_1}+N_B\vect{z_1}$.
\item On se place dans un cas où le rayon de courbure $R_C$ de la trajectoire du point $C$, ainsi que la vitesse de rotation $\dot{\theta}$ par rapport au référentiel $\rep{0}$ sont constants.
\item L'accélération de la pesanteur est $\vect{g}=-g\vect{z_0}$. Accélération de la pesanteur, $g=\SI{10}{ms^{-2}}$.
\item On néglige la masse et les l'inertie des roues. 
%\item On considère que le couple moteur est nul.
\end{itemize}
\end{hypo}
On donne : 	
\begin{itemize}
\item coefficient d'adhérence pneu-route :  $f=1$;
\item masse de $E=1+2$:  $m_E=\SI{134}{kg}$;
\item demi largeur des voies :  $l=\SI{35}{cm}$,  $h=\SI{86}{cm}$.
\end{itemize}
\fi

\begin{obj}
L'objectif est de valider l'exigence 1 : permettre à l'utilisateur de se déplacer sur le sol.
\end{obj}

\subsection*{Étude du dérapage en virage du véhicule Segway}
\ifprof

\else
On donne ci-dessous un extrait du cahier des charges.

%\begin{obj} ~\\
\begin{center}
\begin{tabular}{|p{.7\linewidth}|p{.2\linewidth}|}
\hline 
Exigence & Niveau \\
\hline
id=<<1.1>> Glissement du véhicule pour une vitesse de $\SI{20}{km.h^{-1}}$ dans un virage de rayon de courbure \SI{10}{m} 
& Interdit \\
\hline
\end{tabular}
\end{center}
%\end{obj}

\fi
\subparagraph{}\textit{Exprimer la vitesse, notée  $\vect{V\left(G_E /\rep{0}\right)}$, du point $G_E$ dans son mouvement par rapport à $\rep{0}$ en fonction de $\dot{\theta}$ et $R_C$.}
\ifprof
\begin{corrige}
On a $\vect{V\left(G_E /\rep{0}\right)}=-R_C\dot{\theta}\vect{x_1}$.
\end{corrige}
\else
\fi

\subparagraph{}\textit{Exprimer l'accélération, notée  $\vect{\Gamma \left(G_E /\rep{0}\right)}$, du point $G_E$ dans son mouvement par rapport à $\rep{0}$ en fonction de $\dot{\theta}$ et $R_C$.}
\ifprof
\begin{corrige}
$\vect{\Gamma \left(G_E /\rep{0}\right)}=\left[\dfrac{\dd \vect{V\left(G_E /\rep{0}\right)}}{\dd t}\right]_{\rep{0}}= 
-R_C\ddot{\theta}\vect{x_1}-R_C\dot{\theta}^2\vect{y_1}=-R_C\dot{\theta}^2\vect{y_1}$ ($\dot{\theta}$ est constant).
\end{corrige}
\else
\fi


\subparagraph{}\textit{Exprimer les conditions d'adhérence liant $T_A$, $T_B$, $N_A$, $N_B$ et $f$ traduisant le non glissement du véhicule. En déduire une inéquation liant $T_A + T_B$ à $f$ et $N_A + N_B$.}
\ifprof
\begin{corrige}
La direction des efforts normaux et tangentiels est donnée. En utilisant les lois de Coulomb, on a donc, $T_A\leq fN_A$ et $T_B\leq fN_B$. En sommant les inégalités, on a donc $T_A+T_B\leq f\left(N_A+N_B\right)$.
\end{corrige}
\else
\fi

\subparagraph{}\textit{Isolez $E$ et les roues. Écrire le théorème de la résultante dynamique en projection sur $\vect{z_0}$. En déduire une inéquation liant $T_A + T_B$ à $f$, $m_E$ et $g$.}
\ifprof
\begin{corrige}
$E$ étant un ensemble indéformable, on a : $\vectrd{E}{\rep{0}}=-m_E R_C\dot{\theta}^2\vect{y_1}$ (pas de projection sur $\vect{z_0}$. 
On isole $E$ et les roues  et on réalise le BAME : 
\begin{itemize}
\item pesanteur sur $E$;
\item action du sol sur les roues.
\end{itemize}

En appliquant le TRD en projection sur $\vect{z_{01}}$, on a donc :
$N_A+N_B-m_E g = 0$. 

En conséquence, $T_A+T_B\leq fm_E g $.

\end{corrige}
\else
\fi

\subparagraph{}\textit{Isolez $E$ et les roues. Écrire le théorème de la résultante dynamique en projection sur $\vect{y_1}$ . En déduire une inéquation donnant la vitesse limite $V_L$ de passage dans un virage qui ne provoque pas le dérapage.}
\ifprof

\begin{corrige}
En appliquant le TRD en projection sur $\vect{y_{1}}$, on a : $-T_A-T_B = -m_E R_C\dot{\theta}^2$ $\Leftrightarrow T_A+T_B = m_E R_C\dot{\theta}^2$. En utilisant les résultats de la question précédente, $m_E R_C\dot{\theta}^2 \leq fm_E g $. En notant $V_L=R_C\dot{\theta}$ la vitesse limite avant dérapage, on a $ \dfrac{V_L^2}{R_C} \leq f  g $.
On a donc $V_L=\sqrt{R_Cfg}$.
\end{corrige}
\else
\fi



\subparagraph{}\textit{Faire les applications numériques nécessaires et vérifiez la conformité au cahier des charges.}
\ifprof
\begin{corrige}
La vitesse limite est donc de \SI{10}{m.s^{-1}} soient \SI{36}{km.h^{-1}} ce qui satisfait le cahier des charges. 
\end{corrige}
\else
\fi
\subsection*{Étude du renversement en virage du véhicule Segway}
\ifprof
\else
On donne ci-dessous un extrait du cahier des charges.

%\begin{obj} ~\\
\begin{center}
\begin{tabular}{|p{.7\linewidth}|p{.2\linewidth}|}
\hline 
Exigence & Niveau \\
\hline
id=<<1.2>> Renversement du véhicule pour une vitesse de \SI{20}{km.h^{-1}} dans un virage de rayon de courbure \SI{10}{m}.
& Interdit \\
\hline
\end{tabular}
\end{center}
%\end{obj}

\begin{hypo}
On suppose qu’il y a adhérence des roues en $A$ et $B$.
\end{hypo}
\fi
\subparagraph{}\textit{Calculez le torseur dynamique du système matériel $E$ en $G_E$ dans son mouvement par rapport au référentiel $\rep{0}=\repere{O}{x_0}{y_0}{z_0}$ . Exprimez ses composantes dans la base $\mathcal{B}_1=\base{x_1}{y_1}{z_1}$ .}
\ifprof
\begin{corrige}
Au centre d'inertie de $E$, on a $\vectmd{G_E}{E}{\rep{0}}=\left[\dfrac{\dd \vectmc{G_E}{E}{\rep{0}}}{\dd t}\right]_{\rep{0}}$. On a $\vecto{E}{\rep{0}}=\dot{\theta}\vect{z_0}$. On a donc, $ \vectmc{G_E}{E}{\rep{0}}=-E\dot{\theta}\vect{x_1}-D\dot{\theta}\vect{y_1}+C\dot{\theta}\vect{z_{01}}$.
On a donc $\vectmd{G_E}{E}{\rep{0}}=-E\dot{\theta}^2\vect{y_1}+D\dot{\theta}^2\vect{x_1}$.

En conséquence, $\torseurdyn{E}{\rep{0}}=\torseurl{-m_E R_C\dot{\theta}^2\vect{y_1}}{-E\dot{\theta}^2\vect{y_1}+D\dot{\theta}^2\vect{x_1}}{G_E}$.
\end{corrige}
\else
\fi

\subparagraph{}\textit{Calculez $\vectmd{B}{E}{\rep{0}}\cdot \vect{x_1}$ le moment dynamique au point $B$ de l’ensemble $(E)$ dans son mouvement par rapport au référentiel $\rep{0}=\repere{O}{x_0}{y_0}{z_0}$ en projection sur $\vect{x_1}$.}
\ifprof
\begin{corrige}
$\vectmd{B}{E}{\rep{0}}=\vectmd{G_E}{E}{\rep{0}}+\vect{BG_E}\wedge\vectrd{B}{E}$ $=-E\dot{\theta}^2\vect{y_1}+D\dot{\theta}^2\vect{x_1}+\left( h \vect{z_0}-l\vect{y_1}\right) \wedge \left( {-m_E R_C\dot{\theta}^2\vect{y_1}} \right)$
$=-E\dot{\theta}^2\vect{y_1}+D\dot{\theta}^2\vect{x_1}+ h m_E R_C\dot{\theta}^2\vect{x_1}$.
$\vectmd{B}{E}{\rep{0}}\cdot \vect{x_1}=\left(D+ h m_E R_C\right)\dot{\theta}^2$.

\end{corrige}
\else
\fi

\subparagraph{}\textit{En appliquant le théorème du moment dynamique au point $B$ à l'ensemble $E$ et les roues dans leur mouvement par rapport à $\rep{0}$, en projection sur $\vect{x_1}$, écrire l’équation scalaire qui donne $N_A$ en fonction de $\vectmd{B}{E}{\rep{0}}\cdot \vect{x_1}$ et des données du problème.}
\ifprof
\begin{corrige}
On a : 
\begin{itemize}
\item $\vect{BG_E}\wedge - m_E g \vect{z_{01}}$ 
$=\left(-l\vect{y_1}+h\vect{z_0} \right) \wedge - m_E g \vect{z_{01}}$
$=l  m_E g \vect{x_{1}}$;
\item $\vect{BA}\wedge \left(-T_A\vect{y_1}+N_A\vect{z_1} \right)$ 
$=-2l\vect{y_1}\wedge \left(-T_A\vect{y_1}+N_A\vect{z_1} \right)$
$=-2l N_A\vect{x_1} $.
\end{itemize}
En appliquent le TMD en $B$ suivant $\vect{x_1}$, on a : $l  m_E g -2l N_A=\left(D+ h m_E R_C\right)\dot{\theta}^2$. 

Au final,  $ N_A=\dfrac{ l  m_E g-\left(D+ h m_E R_C\right)\dot{\theta}^2}{2l}$. 
\end{corrige}
\else
\fi

\subparagraph{}\textit{Écrire la condition de non renversement du véhicule.}
\ifprof
\begin{corrige}
Pour qu'il y ait non renversement, $N_A$ doit rester positif ou nul. 
\end{corrige}
\else
\fi

On néglige $\inertie{G_E}{E}$ pour simplifier l’application numérique.

\subparagraph{}\textit{Faire les applications numériques nécessaires et vérifiez la conformité au cahier des charges.}
\ifprof
\begin{corrige}
$ N_A\simeq\dfrac{l  m_E g-h m_E R_C\dot{\theta}^2 }{2l} \geq 0$.%\simeq \dfrac{25 \times 134 \times 10 -86\times 34\times (20000/3600)^2 /10}{2\times 35}$ $ \simeq \SI{330}{N} $. 
Ce qui est positif (pas de basculement). 
\end{corrige}
\else
\fi
%32160-9024


%Cette quille est généralement constituée d’un voile immergé dans l’eau à l’extrémité duquel se trouve un lest profilé. L’efficacité de la quille dépend de la masse du lest et de la longueur du voile. Ces deux paramètres présentent des limitations : le lest ne peut être trop important sous peine de solliciter dangereusement le voile de quille et la longueur de quille est limitée par le tirant d’eau maximal admissible (il faut permettre l’entrée dans les ports sans toucher le fond !).


%\begin{center}
%\includegraphics[width=.8\linewidth]{images/fig_02}
%%\textit{}
%\end{center}

%
%\fi
%\begin{obj}
%L’objectif est de déterminer la puissance utile au déplacement de la quille et de la comparer à celle installée
%par le constructeur.
%\end{obj}
%
%\ifprof
%\else
%%
%%\begin{center}
%%\includegraphics[width=.95\linewidth]{images/Exigences}
%%%\textit{}
%%\end{center}
%
%%\subsection*{Travail à réaliser}
%
%
%\textbf{Hypothèses}
%
%\begin{itemize}
%\item Les liaisons sont toutes parfaites.
%\item Le bateau est à l’arrêt et son repère $R_N$ est galiléen.
%\item Lors de la commande de basculement de la quille, les vérins sont alimentés de telle sorte que : $F_{h2} > 0$ et
%$F_{h3} = 0$. Le vérin 2--4 est alors moteur et le vérin 3--5 est libre ($F_{h2}$ désigne l'action hydraulique sur la tige du vérin 2; on a donc $-F_{h2}$ qui agit sur 4).
%\item Le mouvement du fluide dans les diverses canalisations s’accompagne d’un phénomène de frottement visqueux défini. L’eau exerce sur le voile de quille une action hydrodynamique.
%%\item Seul le vérin 2--4 est moteur ($F_{h3}=0$). Le fluide (pression hydraulique) agit simultanément sur les pièces 2 et 4. L’action du fluide sur 2 est donnée par 
%%$\torseurstat{T}{\text{ph}}{2}=\torseurl{F_{h2}\vect{x}_2}{\vect{0}}{C}$.
%%\item Les actions mécaniques de frottement visqueux provenant du déplacement du fluide dans les canalisations sont toutes négligées.% ($k=0$).
%%\item Les actions hydrodynamiques sur le voile et le lest de quille sont également négligées.
%%%\item Les actions hydrodynamiques sur le voile et le lest de quille sont également négligées.
%%\item Les poids des éléments constitutifs des deux vérins sont négligés.
%%\item La variation de $\theta_2$ pour toute l’amplitude du mouvement de relevage de la quille est faible; $\theta_2$ sera pris égal à 0 : les bases $\mathcal{B}_2$, $\mathcal{B}_4$ et $\mathcal{B}_N$ sont donc confondues. Cependant l’angle $\theta_1$ est différent de zéro.
%%\item Les conditions de déplacement rendent négligeables les effets dynamiques. Les théorèmes de la statique seront donc utilisés dans la suite.
%\end{itemize}
%
%Le modèle de calcul est donné dans les figures suivantes.
%
%\begin{center}
%\includegraphics[width=.7\linewidth]{images/fig_05_a}
%
%\textit{Modèle 2D}
%\end{center}
%
%\begin{center}
%\includegraphics[width=\linewidth]{images/fig_05_b}
%
%\textit{Paramétrage}
%\end{center}
%
%\textbf{Données géométriques, massiques et inertielles}
%\begin{multicols}{2}
%\footnotesize
%
%$\vect{OA}=R\vect{y_1}$; $\vect{CA}=x_{24}(t)\vect{x_2}$; $\vect{AB}=x_{35}(t)\vect{x_3}$, 
%\begin{itemize}
%\item Solide 1, masse $M_1$, centre d'inertie $G_1$, $\vect{OG_1}=-L_1\vect{y_1}$, $\inertie{G_1}{1}=\matinertie{A_1}{B_1}{C_1}{-D_1}{0}{0}{\left(\vect{x_1},\vect{y_1},\vect{z_N}, \right)}$.
%\item Solide 2, masse $M_2$, centre d'inertie $G_2$, $\vect{AG_2}=-L_2\vect{x_2}$, $\inertie{G_2}{2}=\matinertie{A_2}{B_2}{B_2}{0}{0}{0}{\left(\vect{x_2},\vect{y_2},\vect{z_N}, \right)}$.
%\item Solide 3, masse $M_3=M_2$, centre d'inertie $G_3$, $\vect{AG_3}=L_2\vect{x_3}$, $\inertie{G_3}{3}=\matinertie{A_3}{B_3}{B_3}{0}{0}{0}{\left(\vect{x_3},\vect{y_3},\vect{z_N}, \right)}$.
%\item Solide 4, masse $M_4$, centre d'inertie $C$, $\inertie{C}{4}=\matinertie{A_4}{B_4}{C_4}{0}{0}{0}{\left(\vect{x_2},\vect{y_2},\vect{z_N}, \right)}$.
%\item Solide 5, masse $M_5$, centre d'inertie $B$, $\inertie{B}{5}=\matinertie{A_5}{B_5}{C_5}{0}{0}{0}{\left(\vect{x_3},\vect{y_3},\vect{z_N}, \right)}$.
%\end{itemize}
%\normalsize
%
%\end{multicols}
%
%\textbf{Actions mécaniques}
%
%\footnotesize
%\begin{itemize}
%\item Action de pression de l'huile sur 2 : $\torseurstat{T}{\text{ph}}{2}=\torseurl{F_{h2}\vect{x_2}}{\vect{0}}{C}$.
%\item Action de pression de l'huile sur 3 : $\torseurstat{T}{\text{ph}}{3}=\torseurl{-F_{h3}\vect{x_3}}{\vect{0}}{B}$.
%\item Action de frottement visqueux de l'huile sur 2 : $\torseurstat{T}{\text{phf}}{2}=\torseurl{-k\dfrac{\dd x_{24}(t)}{\dd t}\vect{x_2}}{\vect{0}}{A}$ avec $k>0$.
%\item Action de frottement visqueux de l'huile sur 3 : $\torseurstat{T}{\text{phf}}{3}=\torseurl{-k\dfrac{\dd x_{35}(t)}{\dd t}\vect{x_3}}{\vect{0}}{A}$ avec $k>0$.
%\item Action hydraudynamique de l'eau sur 1 : $\torseurstat{T}{\text{eau}}{1}=\torseurl{F_p\vect{z_1}+F_t\vect{x_1}}{\vect{0}}{P}$ avec $\vect{OP}=-h\vect{y_1}$.
%\end{itemize}
%
%\normalsize
%%
%%\begin{center}
%%\includegraphics[width=\linewidth]{images/fig_04}
%%
%%%\vspace{-.2cm}
%%
%%$\vect{OA}=R\vect{y_1}$, 
%%$\theta_1 =\angl{x_N}{x_1}$,
%%$\vect{OG_1}=-L_1\vect{y_1}$,
%%$\vect{AA_2}=-d\vect{z_N}$
%%$\vect{AA_3}=d\vect{z_N} $.
%%
%%%\vspace{-.8cm}
%%
%%\textit{Schéma cinématique 3D}
%%\end{center}
%
%
%
%
%\fi
%
%
%
%\ifnormal
%
%\subsection*{Vecteurs vitesse}
%
%\subparagraph{}\textit{Tracer le graphe de liaisons.}
%\ifprof
%\begin{corrige} ~\\
%
%\begin{center}
%\includegraphics[width=.45\linewidth]{images/cor_01}
%\end{center}
%\end{corrige}
%\else
%\fi
%
%
%
%\subparagraph{}\textit{
%Exprimer les vitesses suivantes :
%\begin{enumerate}
%\item $\vectv{G_1}{1}{N}$ en fonction de $\dfrac{\dd \theta_1(t)}{\dd t}$ et des paramètres géométriques utiles;
%\item $\vectv{G_2}{2}{N}$ en fonction de $\dfrac{\dd \theta_2(t)}{\dd t}$, $\dfrac{\dd x_{24}(t)}{\dd t}$, $x_{24}$ et des paramètres géométriques utiles;
%\item $\vectv{G_3}{3}{N}$ en fonction de $\dfrac{\dd \theta_3(t)}{\dd t}$, $\dfrac{\dd x_{35}(t)}{\dd t}$, $x_{35}$ et des paramètres géométriques utiles;
%\item $\vectv{A}{2}{4}$ en fonction de  $\dfrac{\dd x_{24}(t)}{\dd t}$.
%\end{enumerate}}
%\ifprof
%\begin{corrige}~\\
%\begin{enumerate}
%\item $\vectv{G_1}{1}{N}=\vectv{O}{1}{N}+\vect{G_1 O}\wedge \vecto{1}{N}=$ $L_1\vect{y_1}\wedge \dot{\theta}_1\vect{z_N}$ $=L_1 \dot{\theta}_1\vect{x_1}$.
%\item $\vectv{G_2}{2}{N}=$ $\left[  \dfrac{\dd \left( \vect{OA} + \vect{AG_2}\right)}{\dd t}\right]_{R_N}$ 
%$=\left[ \dfrac{\dd \left( R\vect{y_1}  -L_2 \vect{x_2}\right)}{\dd t}\right]_{R_N}$ $= -R\dot{\theta}_1\vect{x_1}  -L_2 \dot{\theta}_2\vect{y_2}$. 
%
%On a aussi $\vectv{G_2}{2}{N}=$ $\left[  \dfrac{\dd \left( \vect{CA} + \vect{AG_2}\right)}{\dd t}\right]_{R_N}$ 
%$=\left[ \dfrac{\dd \left(  x_{24}(t)\vect{x_2}  -L_2 \vect{x_2}\right)}{\dd t}\right]_{R_N}$ $=  \dot{x}_{24}(t)  \vect{x_2}+\dot{\theta}_2\left(  x_{24}(t)  -L_2 \right)\vect{y_2}$.
%
%\item $\vectv{G_3}{3}{N}$  $=\left[  \dfrac{\dd \left( \vect{OA} + \vect{AG_3}\right)}{\dd t}\right]_{R_N}$ 
%$=\left[ \dfrac{\dd \left( R\vect{y_1} + L_2 \vect{x_3}\right)}{\dd t}\right]_{R_N}$ $= -R\dot{\theta}_1\vect{x_1}  +L_2 \dot{\theta}_3\vect{y_3}$. 
%
%
%On a aussi $\vectv{G_3}{3}{N}=$ $\left[  \dfrac{\dd \left( \vect{BA} + \vect{AG_3}\right)}{\dd t}\right]_{R_N}$ 
%$=\left[ \dfrac{\dd \left(  -x_{35}(t)\vect{x_3}  +L_2 \vect{x_3}\right)}{\dd t}\right]_{R_N}$ $= - \dot{x}_{35}(t)  \vect{x_3}+\dot{\theta}_3\left(  -x_{35}(t)  +L_2 \right)\vect{y_3}$.
%
%
%\item $\vectv{A}{2}{4}$ $=\left[  \dfrac{\dd  \vect{CA} }{\dd t}\right]_{R_4}$  $=\left[  \dfrac{\dd \left( x_{24}(t)\vect{x_2} \right)}{\dd t}\right]_{R_4}$ $= \dot{x}_{24}(t)\vect{x_2}$.
%\end{enumerate}
%\end{corrige}
%\else
%\fi
%
%\subsection*{Energie cinétique}
%Soit $E$ l’ensemble constitué des solides 1, 2, 3, 4 et 5.
%
%On note $\ec{i}{N}$ l'énergie cinétique de $i$ dans son mouvement par rapport au référentiel galiléen $R_N$.
%
%\subparagraph{}\textit{
%Exprimer les énergies cinétiques suivantes : 
%\begin{enumerate}
%\item $\ec{1}{N}$, en fonction de
%$\dfrac{\dd \theta_1(t)}{\dd t}$  et des paramètres inertiels et géométriques utiles;
%\item $\ec{2}{N}$, en fonction de
%$\dfrac{\dd \theta_2(t)}{\dd t}$ , $\dfrac{\dd x_{24}(t)}{\dd t}$ , $x_{24}(t)$ et des paramètres inertiels et géométriques utiles.
%\item $\ec{4}{N}$, en fonction de $\dfrac{\dd \theta_2(t)}{\dd t}$ et des paramètres inertiels et géométriques utiles.
%\end{enumerate}}
%\ifprof
%\begin{corrige}~\\
%\begin{enumerate}
%\item $\ec{1}{N}=\dfrac{1}{2}\torseurcin{V}{1}{N}\otimes \torseurci{1}{N}$ 
%$=\dfrac{1}{2}\torseurl{\vecto{1}{N}}{\vectv{G_1}{1}{N}}{G_1}\otimes \torseurl{M_1\vectv{G_1}{1}{N}}{\vectmc{G_1}{1}{N}}{G_1}$
%$=\dfrac{1}{2}\torseurl{\dot{\theta}_1\vect{z_N}}{L_1 \dot{\theta}_1\vect{x_1}}{G_1}\otimes \torseurl{M_1L_1 \dot{\theta}_1\vect{x_1}}{\dot{\theta}_1\left(-D_1\vect{y_N}+C_1\vect{z_N}\right) }{G_1}$ $=\dfrac{1}{2}\left(\dot{\theta}_1^2\left(-D_1\vect{y_N}+C_1\vect{z_N}\right)\vect{z_N} + M_1L_1^2 \dot{\theta}_1^2 \right)$
%$=\dfrac{1}{2}\left(\dot{\theta}_1^2C_1 + M_1L_1^2 \dot{\theta}_1^2 \right)$
%$=\dfrac{1}{2}\dot{\theta}_1^2\left(C_1 + M_1L_1^2  \right)$.
%
%\item $\ec{2}{N}=\dfrac{1}{2}\torseurcin{V}{2}{N}\otimes \torseurci{2}{N}$ 
%$=\dfrac{1}{2}\torseurl{\vecto{2}{N}}{\vectv{G_2}{2}{N}}{G_2}\otimes \torseurl{M_2\vectv{G_2}{2}{N}}{\vectmc{G_2}{2}{N}}{G_2}$
%
%$=\dfrac{1}{2}
%\torseurl{\dot{\theta}_2\vect{z_N}}{\dot{x}_{24}(t)  \vect{x_2}+\dot{\theta}_2\left(  x_{24}(t)  -L_2 \right)\vect{y_2}}{G_2}
%\otimes 
%\torseurl{M_2\left(\dot{x}_{24}(t)  \vect{x_2}+\dot{\theta}_2\left(  x_{24}(t)  -L_2 \right)\vect{y_2}\right)}{\dot{\theta}_2 B_2 \vect{z_N} }{G_1}
%$ 
%
%$=\dfrac{1}{2}\left( B_2 \dot{\theta}_2^2 +M_2\left(\dot{x}_{24}(t)  \vect{x_2}+\dot{\theta}_2\left(  x_{24}(t)  -L_2 \right)\vect{y_2}\right)^2  \right)
%$ 
%$=\dfrac{1}{2}\left( B_2 \dot{\theta}_2^2 +M_2\left(\dot{x}_{24}(t)^2+\dot{\theta}_2^2\left(  x_{24}(t)  -L_2 \right)^2\right)  \right)
%$ 
%.
%\item $\ec{4}{N}=\dfrac{1}{2}C_4\dot{\theta}_2^2$ . 
%\end{enumerate}
%\end{corrige}
%\else
%\fi
%
%
%\subsection*{Evaluation des puissances développées par les actions mécaniques intérieures à E}
%\subparagraph{}\textit{Recenser, puis exprimer les puissances non nulles (notées $\pint{i}{j}{}$) développées par les actions mécaniques intérieures à $E$ en fonction du (ou des) paramètre(s) propre(s) à la liaison ou au mouvement concerné.}
%\ifprof
%\begin{corrige} ~\\
%
%Bilan des puissances intérieures à l'ensemble 1, 2, 3, 4, 5 :
%\begin{itemize}
%\item la puissance dissipée dans les liaisons est nulle car il n'y a pas de frottements;
%\item  $\pint{4}{2}{Ph} = \torseurstat{T}{4}{2}\otimes \torseurcin{V}{2}{4}$ 
%$=\torseurl{\vectf{4}{2}}{\vectm{A}{4}{2}}{A}\otimes \torseurl{\vecto{2}{4}}{\vectv{A}{2}{4}}{A}$ 
%
% $=\torseurl{\vectf{4}{2}}{-}{A}\otimes \torseurl{\vect{0}}{\vectv{A}{2}{4}}{A}$
%  $=\torseurl{F_{h2}\vect{x_2}}{-}{A}\otimes \torseurl{\vect{0}}{ \dot{x}_{24}(t)  \vect{x_2}}{A}$
%  $=F_{h2}\dot{x}_{24}$;
%
%  \item  $\pint{4}{2}{Phf}  $
% $=\torseurl{\vectf{4}{2}}{-}{A}\otimes \torseurl{\vect{0}}{\vectv{A}{2}{4}}{A}$
%  $=\torseurl{-k\dot{x}_{24}(t)  \vect{x_2}}{-}{A}\otimes \torseurl{\vect{0}}{ \dot{x}_{24}(t)  \vect{x_2}}{A}$  $=-k\dot{x}_{24}^2(t)$;
%\item  $\pint{3}{5}{Ph}  $
%$=\torseurl{\vectf{5}{3}}{-}{A}\otimes \torseurl{\vect{0}}{\vectv{A}{3}{5}}{A}$
%$=\torseurl{F_h  \vect{x_3}}{-}{A}\otimes \torseurl{\vect{0}}{ \dot{x}_{35}(t)  \vect{x_3}}{A}$ $=F_h\dot{x}_{35}(t)$;
%\item  $\pint{3}{5}{Phf}  $
%$=\torseurl{\vectf{5}{3}}{-}{A}\otimes \torseurl{\vect{0}}{\vectv{A}{3}{5}}{A}$
%$=\torseurl{-k\dot{x}_{35}(t)  \vect{x_3}}{-}{A}\otimes \torseurl{\vect{0}}{ \dot{x}_{35}(t)  \vect{x_3}}{A}$ $=-k\dot{x}_{35}^2(t)$.
%\end{itemize}
%
%\end{corrige}
%\else
%\fi
%
%
%\subsection*{Evaluation des puissances développées par les actions mécaniques extérieures à E}
%\subparagraph{}\textit{Recenser, puis exprimer les puissances galiliéennes non nulles (notées $\pext{i}{j}{k}$) développées par les actions mécaniques extérieures à $E$. Chaque puissance sera exprimée à l’aide du (ou des)
%paramètre(s) propre(s) à la liaison ou au mouvement concerné.}
%\ifprof
%\begin{corrige}
%Bilan des puissances intérieures à l'ensemble 1, 2, 3, 4, 5 :
%\begin{itemize}
%\item la puissance dissipée dans les liaisons est nulle car il n'y a pas de frottements;
%\item  $\pext{\text{pes}}{1}{R_N} = \torseurstat{T}{\text{pes}}{1}\otimes \torseurcin{V}{1}{R_N}$ 
%$=\torseurl{-M_1g\vect{y_N}}{0}{G_1}\otimes \torseurl{\dot{\theta}_1\vect{z_N}}{\vectv{G_1}{1}{R_N}=L_1 \dot{\theta}_1\vect{x_1}}{G_1}$ 
%
% $=-M_1gL_1 \dot{\theta}_1\vect{x_1}\vect{y_N}$ $=-M_1gL_1 \dot{\theta}_1\sin \theta_1$;
%
%\item  $\pext{\text{pes}}{2}{R_N} = \torseurstat{T}{\text{pes}}{2}\otimes \torseurcin{V}{2}{R_N}$ 
%$=\torseurl{-M_2g\vect{y_N}}{0}{G_2}\otimes \torseurl{\dot{\theta}_1\vect{z_N}}{ \dot{x}_{24}(t)  \vect{x_2}+\dot{\theta}_2\left(  x_{24}(t)  -L_2 \right)\vect{y_2}}{G_1}$ 
%
% $= -M_2g\vect{y_N}\left(\dot{x}_{24}(t)  \vect{x_2}+\dot{\theta}_2\left(  x_{24}(t)  -L_2 \right)\vect{y_2}\right)$ $= -M_2g\dot{x}_{24}(t) \sin \theta_2-M_2g\dot{\theta}_2\left(  x_{24}(t)  -L_2 \right)\cos \theta_2$;
%
%\item  $\pext{\text{pes}}{3}{R_N} $
% $= -M_3g\vect{y_N}\left(-\dot{x}_{35}(t)  \vect{x_3}+\dot{\theta}_3\left( - x_{35}(t)  +L_2 \right)\vect{y_3}\right)$
% 
%  $= -M_3g\left(-\dot{x}_{35}(t)  \sin\theta_3+\dot{\theta}_3\left( - x_{35}(t)  +L_2 \right)\cos\theta_3\right)$ ;
% 
%  
%\item  $\pext{\text{eau}}{1}{R_N} = \torseurstat{T}{\text{eau}}{1}\otimes \torseurcin{V}{1}{R_N}$ 
%$=\torseurl{F_p \vect{z_1}+F_t \vect{x_1}}{0}{P}\otimes \torseurl{\dot{\theta}_1\vect{z_N}}{h \dot{\theta}_1\vect{x_1}}{P}$ 
%
%$=F_t h \dot{\theta}_1$;
%% $=-MgL_1 \dot{\theta}_1\vect{x_1}\vect{y_N}$ $=-MgL_1 \dot{\theta}_1\sin \theta_1$;
%
%\end{itemize}
%\end{corrige}
%\else
%\fi
%
%\subparagraph{}\textit{Appliquer le théorème de l’énergie-puissance à $E$ dans son mouvement par rapport à $N$. Écrire ce
%théorème de façon globale en utilisant uniquement les notations précédentes, sans leur
%développement. Exprimer dans ces conditions la puissance motrice que fournit le vérin moteur en
%fonction du reste : équation (1).}
%\ifprof
%\begin{corrige}
%On a :
%$\pext{\bar{E}}{E}{R_N} + \sum \pint{i}{j}{}=\dfrac{\dd \ec{E}{R_N}}{\dd t}$
%\end{corrige}
%\else
%\fi
%
%\else
%\fi
%
%
%\iftdifficile
%
%\subparagraph{}\textit{ Exprimer la puissance motrice que fournit le vérin moteur en
%fonction des données du problème. La méthode sera précisément décrite. Chacun des termes seront calculés. Il n'est pas demandé d'écrire la relation finale.}
%\else
%\fi
%
%\ifprof
%\else
%
%
%On se place dans le cas où une commande en vitesse est générée à destination du vérin [2, 4]. Le vérin [3, 5]
%est libre. Cette commande <<~en trapèze de vitesse~>>  provoque le déplacement de la quille de la position $\theta_1=0$
% à la position $\theta_1 = 45\degres$ en 4 secondes, le maintien de la quille dans cette position pendant 1 seconde puis le
% retour à la position $\theta_1=0$ en 4 secondes. Les phases d’accélération et de décélération (rampes) durent 1 seconde.
%
%
%\begin{center}
%\includegraphics[width=.6\linewidth]{images/fig_06}
%%\textit{Paramétrage}
%\end{center}
%
%Un logiciel de calcul permet de tracer l’évolution temporelle des puissances mises en jeu. Ces puissances sont
%représentées sur la figure suivante. 
%
%\begin{center}
%\includegraphics[width=.8\linewidth]{images/fig_07}
%%\textit{Paramétrage}
%\end{center}
%
%\fi
%
%\subparagraph{}\textit{Dans le but de chiffrer la valeur maximale de la puissance que doit fournir l’actionneur pour réaliser
%le mouvement prévu, tracer, à l’aide de la figure précédente, l’allure de
%l’évolution temporelle de cette puissance. Pour cela, évaluer les valeurs aux instants $t=\SI{0}{s}$, $t=\SI{1}{s}$,
%$t=\SI{3}{s}$ et $t=\SI{4}{s}$.
%Sur cet intervalle $[0,\SI{4}{s}]$, évaluer, en kW, la valeur maximale de la puissance que doit fournir
%l’actionneur. Expliquer pourquoi le maximum de puissance est situé sur cet intervalle.}
%\ifprof
%\begin{corrige}
%D'après UPSTI. À \SI{1}{s}, $2200+5800+2500+4000=\SI{14500}{W}$ à \SI{3}{s}
%$0+4000+2500+16000=\SI{22500}{W}$
%Maximum à environ \SI{22,5}{kW}.
%Le maximum est bien sur cet intervalle car le poids y est résistant (le poids est moteur sur
%[\SI{5}{s} ; \SI{8}{s}]).
%
%\end{corrige}
%\else
%\fi
%\subparagraph{}\textit{
%Le constructeur indique une puissance motrice installée sur son bateau de \SI{30}{kW}.
%Dans les hypothèses utilisées pour constituer le modèle de calcul, indiquer ce qui peut expliquer la
%différence entre la valeur calculée et la valeur installée.}
%\ifprof
%\begin{corrige}
%D'après UPSTI. La différence est de \SI{7,5}{kW}.
%Elle ne peut pas provenir des hypothèses faites (liaisons parfaites et RN galiléen).
%Elle provient certainement du fait que le système est surdimensionné
%pour pallier les erreurs de modélisation des actions de l'eau, le vieillissement de la quille avec
%les algues collées qui rajoutent du poids...
%\end{corrige}
%\else
%\fi


\ifprof
%\end{multicols}
\else
\end{multicols}
\fi


\end{document}

\subparagraph{}\textit{}
\ifprof
\begin{corrige}
\end{corrige}
\else
\fi

\begin{center}
\includegraphics[width=\linewidth]{images/img_04}
%\textit{}
\end{center}

