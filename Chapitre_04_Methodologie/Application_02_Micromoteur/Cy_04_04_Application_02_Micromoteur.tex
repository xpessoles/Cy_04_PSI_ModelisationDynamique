\documentclass[10pt,fleqn]{article} % Default font size and left-justified equations
\usepackage[%
    pdftitle={Modélisation dynamique : cinétique},
    pdfauthor={Xavier Pessoles}]{hyperref}

    
\input{style/new_style}
\input{style/macros_SII}

\fichetrue
%\fichefalse

\proftrue
%\proffalse

\tdtrue
%\tdfalse

\courstrue
\coursfalse


\def\discipline{Sciences \\Industrielles de \\ l'Ingénieur}
\def\xxtete{Sciences Industrielles de l'Ingénieur}

\def\classe{\textsf{PSI$\star$ -- MP}}
\def\xxnumpartie{Cycle 04}
\def\xxpartie{Modéliser le comportement des systèmes mécaniques dans le but d'établir une loi de comportement ou de déterminer des actions mécaniques en utilisant le PFD}

\def\xxnumchapitre{Chapitre 4 \vspace{.2cm}}
\def\xxchapitre{\hspace{.12cm} Méthodologie : détermination des équations de mouvement}




\def\xxtitreexo{Chaîne fermée -- Micromoteur de modélisme}
\def\xxsourceexo{\hspace{.2cm} \footnotesize{Équipe PT La Martinière Monplaisir}}


\def\xxposongletx{2}
\def\xxposonglettext{1.45}
\def\xxposonglety{20}
%\def\xxonglet{Part. 1 -- Ch. 3}
\def\xxonglet{Cycle 04}

\def\xxactivite{Application 02}
\def\xxauteur{\textsl{Équipe PT La Martinière Monplaisir}}

\def\xxcompetences{%
\textsl{%
\textbf{Savoirs et compétences :}\\
%\begin{itemize}[label=\ding{112},font=\color{ocre}] 
%\item \textit{Mod2.C13} : centre d'inertie
%\item \textit{Mod2.C14} : opérateur d'inertie
%\item \textit{Mod2.C15} : matrice d'inertie
%\end{itemize}
}}
\def\xxfigures{
\includegraphics[width=.55\linewidth]{images/fig_00.png}
}%figues de la page de garde


\def\xxpied{%
Cycle 04 -- Modélisation mécanique -- Cinétique\\% afin de valider leurs performances.\\
Chapitre 4 -- \xxactivite%
}

\setcounter{secnumdepth}{5}
%---------------------------------------------------------------------------

\usepackage{pgfplots}
\begin{document}
\def\pathfig{images}
%\chapterimage{png/Fond_Cin}
\input{style/new_pagegarde}
\vspace{5cm}
\pagestyle{fancy}
\thispagestyle{plain}

\def\columnseprulecolor{\color{ocre}}
\setlength{\columnseprule}{0.4pt} 

\def\pathfig{images}

\ifprof
\else
\begin{multicols}{2}
\fi

\subsection*{Mise en situation}
Les figures et le schéma ci-dessous représentent un micromoteur à combustion interne de modèle réduit. Du point de vue cinématique, il est basé sur un système bielle manivelle \textbf{(2,1)}, associé à un piston \textbf{(3)}, animé d’un mouvement de translation rectiligne alternatif. 

\begin{center}
\includegraphics[width=.8\linewidth]{images/fig_02}
\end{center}
On note : 
\begin{itemize}
\item $\vect{AB}=e\vect{x_1}$, $\vect{BC}=L_2\vect{y_2}$, $\vect{AC}=\lambda_3\vect{y_0}$;
\item $\vect{HG_1}=a_1\vect{x_1}$, $\vect{BG_2}=a_2\vect{y_2}$, $\vect{CG_3}=a_3\vect{y_0}$;
\item $\angl{x_0}{x_1}=\angl{y_0}{y_1}=\theta_1$, $\angl{x_0}{x_2}=\angl{y_0}{y_1}=\theta_2$; $\omega_{10}=\dot{\theta}_1$ et $\omega_{20}=\dot{\theta}_2$.
\end{itemize}

On note $C_m$ le couple délivré par le moteur et $F_e$ la force exercée sur le piston suite à l'explosion du mélange air - carburant.

\subparagraph{}
\textit{Exprimer la relation liant la vitesse de rotation $\omega_{10}$ du vilebrequin \textbf{(1)} et la vitesse du piston \textbf{(3)}, notée $V_{3/0}$. Déterminer la vitesse et l'accélération du centre d'inertie de la bielle \textbf{(2)} par rapport à \textbf{(0)}.}

\ifprof
\begin{corrige}
On réalise une fermeture géométrique dans le triangle $ABC$ et on a : 
$\vect{AB}+\vect{BC}+\vect{CA}=\vect{0}$ $\Leftrightarrow e\vect{x_1}+L_2\vect{x_2}-\lambda_3 \vect{y_0}$ $\Leftrightarrow e\left( \cos \theta_1 \vect{x_0}+\sin \theta_1 \vect{y_0} \right)+L_2\left( \cos \theta_2 \vect{x_0}+\sin \theta_2 \vect{y_0} \right)-\lambda_3 \vect{y_0} = \vect{0}$. 
On a donc : 
$\left\{
\begin{array}{l}
e\cos \theta_1 +L_2 \cos \theta_2 = 0 \\
e\sin \theta_1 + L_2 \sin \theta_2 -\lambda_3 = 0
\end{array}
\right.$
$\Leftrightarrow \left\{
\begin{array}{l}
L_2 \cos \theta_2 = -e\cos \theta_1  \\
L_2 \sin \theta_2  = \lambda_3-e\sin \theta_1
\end{array}
\right.
$
Au final, $L_2^2 = e^2\cos^2 \theta_1 + \left(\lambda_3-e\sin \theta_1\right)^2$
$\Leftrightarrow L_2^2 - e^2\cos^2 \theta_1 = \left(\lambda_3-e\sin \theta_1\right)^2$

$\Rightarrow \sqrt{L_2^2 - e^2\cos^2 \theta_1} = \lambda_3-e\sin \theta_1$
$ \Rightarrow \lambda_3 = \sqrt{L_2^2 - e^2\cos^2 \theta_1}+e\sin \theta_1$.

\end{corrige}
\else
\fi

Dans la perspective d’une étude dynamique, on se propose d’évaluer les caractéristiques de masse et inertie des trois pièces mobiles, ainsi que leurs propriétés cinétiques.

On note $\inertie{H}{1}=\matinertie{A_1}{B_1}{C_1}{-D_1}{-E_1}{-F_1}{\repere{H}{x_1}{y_1}{z_1}}$ la matrice d'inertie en $H$ de l'ensemble \{vilebrequin, hélice\} repéré \textbf{(1)}. 

\subparagraph{}
\textit{En considérant que seul le plan $\left(H,\vect{x_1},\vect{z_1}\right)$ est le plan de symétrie, indiquer quelle(s) simplification(s) cela apporte à cette matrice d'inertie.} 

\ifprof
\begin{corrige}
On a donc une invariance suivant $\vect{y_1}$ et 
$\inertie{H}{1}=\matinertie{A_1}{B_1}{C_1}{0}{-E_1}{0}{\repere{H}{x_1}{y_1}{z_1}}$
\end{corrige}
\else
\fi


\subparagraph{}
\textit{Reprendre la question précédente en l'appliquant à la bielle \textbf{(2)} et au piston \textbf{(3)}. Définir la forme de la matrice d’inertie de chacune de ces deux pièces, en précisant en quel point et dans quelle base elle est définie.}
\ifprof
\begin{corrige}
De même $\inertie{G_2}{2}=\matinertie{A_2}{B_2}{C_2}{0}{-E_2}{0}{\repere{G_2}{x_2}{y_2}{z_2}}$ et 
$\inertie{G_3}{3}=\matinertie{A_2}{B_2}{C_2}{0}{-E_2}{0}{\repere{G_3}{x_2}{y_2}{z_2}}$.
\end{corrige}
\else
\fi


Par la suite on fait l'hypothèse que les matrices d'inertie sont diagonales.
\subparagraph{}
\textit{On note $m_1$, $m_2$ et $m_3$ les masses des trois pièces mobiles \textbf{(1)}, \textbf{(2)} et \textbf{(3)}. Exprimer, pour chacune d’elles : son torseur cinétique et son torseur dynamique.}
\ifprof
\begin{corrige}
$H$ est un point fixe : 
\begin{itemize}
\item $\torseurci{1}{0}
=\torseurl{\vectrc{1}{0}=m_1\vectv{G_1}{1}{0}}{\vectmc{H}{1}{0}=\inertie{H}{1}\vecto{1}{0}}{H}
=\torseurl{\vect{0}}{\dot{\theta}_1\vect{z_1}}{H}$
\item $\torseurdyn{1}{0}
=\torseurl{\vectrd{1}{0}=m_1\vectg{G_1}{1}{0}}{\vectmc{H}{1}{0}=\left[\dfrac{\dd \vectmc{H}{1}{0}}{\dd t}\right]_{\mathcal{R}_0}}{H}
=\torseurl{\vect{0}}{\ddot{\theta}_1\vect{z_1}}{H}$
\end{itemize}
\end{corrige}
\else
\fi

\subparagraph{}
\textit{Déterminer l'équation de mouvement par les théorèmes généraux.}
\ifprof
\begin{corrige}
\end{corrige}
\else
\fi

\ifprof
\else
\end{multicols}
\fi

\begin{center}
\includegraphics[width=.7\linewidth]{images/fig_03}
\end{center}

\end{document}
\begin{center}
\includegraphics[width=\linewidth]{images/}
\end{center}

\subparagraph{}
\textit{}
\ifprof
\begin{corrige}
\end{corrige}
\else
\fi
