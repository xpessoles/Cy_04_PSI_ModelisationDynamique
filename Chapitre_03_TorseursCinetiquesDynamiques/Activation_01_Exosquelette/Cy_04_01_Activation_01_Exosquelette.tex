\documentclass[10pt,fleqn]{article} % Default font size and left-justified equations
\usepackage[%
    pdftitle={Modélisation dynamique : cinétique},
    pdfauthor={Xavier Pessoles}]{hyperref}



\input{style/new_style}
\input{style/macros_SII}
\usepackage{multicol}
\usepackage{siunitx}

\fichetrue
%\fichefalse

\proftrue
%\proffalse

\tdtrue
%\tdfalse

\courstrue
\coursfalse


\def\discipline{Sciences \\Industrielles de \\ l'Ingénieur}
\def\xxtete{Sciences Industrielles de l'Ingénieur}

\def\classe{\textsf{PSI$\star$ -- MP}}
\def\xxnumpartie{Cycle 04}
\def\xxpartie{Modéliser le comportement des systèmes mécaniques dans le but d'établir une loi de comportement ou de déterminer des actions mécaniques en utilisant le PFD}

\def\xxnumchapitre{Chapitre 3 \vspace{.2cm}}
\def\xxchapitre{\hspace{.12cm} Application du Principe Fondamental de la Dynamique}





\def\xxactivite{Activation 01}
\def\xxauteur{\textsl{Xavier Pessoles}}


\def\xxtitreexo{Assistance pour le maniement de charges dans l’industrie}
\def\xxsourceexo{\hspace{.2cm} \footnotesize{Concours Centrale Supelec TSI 2017}}




  
\def\xxposongletx{2}
\def\xxposonglettext{1.45}
\def\xxposonglety{20}
%\def\xxonglet{Part. 1 -- Ch. 3}
\def\xxonglet{\textsf{Cycle 01}}

\def\xxactivite{Activation 01}
\def\xxauteur{\textsl{Xavier Pessoles}}

\def\xxcompetences{%
\vspace{-.5cm}
\textsl{%
\textbf{Savoirs et compétences :}
\begin{itemize}[label=\ding{112},font=\color{ocre}] 
%\item \textit{Mod2.C16} : torseur cinétique
%\item \textit{Mod2.C17} : torseur dynamique
\item \textit{Mod2.C17.SF1} : déterminer le torseur dynamique d’un solide, ou d’un ensemble de solides, par rapport à un autre solide
%\item \textit{Mod2.C15} : matrice d'inertie
\item \textit{Res1.C2} : principe fondamental de la dynamique
%\item \textit{Res1.C1.SF1} : proposer une démarche permettant la détermination de la loi de mouvement
%\item \textit{Res1.C2.SF1} : proposer une méthode permettant la détermination d’une inconnue de liaison
\end{itemize}
}}
\def\xxfigures{
\includegraphics[width=.3\linewidth]{images/fig_01}}%figues de la page de garde


\def\xxpied{%
Cycle 04 -- Modélisation mécanique -- PFD\\% afin de valider leurs performances.\\
Chapitre 3 -- \xxactivite%
}

\setcounter{secnumdepth}{5}
%---------------------------------------------------------------------------


\begin{document}
%\chapterimage{png/Fond_Cin}
\input{style/new_pagegarde}
\vspace{4.5cm}
\pagestyle{fancy}
\thispagestyle{plain}


\def\columnseprulecolor{\color{ocre}}
\setlength{\columnseprule}{0.4pt} 

\ifprof
\else
\begin{multicols}{2}
\fi

\section*{Mise en situation -- Assurer le mouvement vertical}
\ifprof
\else

\ifprof
\else
\noindent
\begin{tabular}{m{.6\linewidth}m{.3\linewidth}}
L’exosquelette est un appareil qui apporte à un être humain des capacités qu’il ne possède pas ou qu’il a perdues à cause d’un accident. Ce type d’appareil peut permettre à une personne de soulever des charges lourdes et diminuer considérablement les efforts à fournir sans la moindre fatigue. Après avoir revêtu un exosquelette adapté à sa morphologie et à sa taille, l’utilisateur peut faire ses mouvements en bénéficiant
d’une grande fluidité.
& 
\includegraphics[width=\linewidth]{images/fig_02}

\end{tabular}



\begin{center}
\includegraphics[width=\linewidth]{images/Exigences}
%\textit{}
\end{center}
\fi

%\subsection*{}
\fi
\begin{obj}
Proposer un modèle de connaissance des éléments réalisant l’exigence fonctionnelle <<~assurer le mouvement vertical~>> puis valider les performances attendues listées par le cahier des charges.
\end{obj}



%\subsection*{Élaboration du modèle géométrique direct et du modèle articulaire inverse}
%\begin{obj}
%Élaborer la commande du moteur pilotant le genou à partir d’un mouvement défini dans l’espace
%opérationnel puis converti dans l’espace articulaire.
%\end{obj}
%
%\ifprof
%\else
%L’étude se limite au passage de la position accroupie à la position relevée de l’exosquelette. Lors de ce passage,
%le point $O_2$ est en mouvement de translation verticale suivant la direction $\axe{O_0}{{z_0}}$ et sa vitesse de déplacement
%évolue selon une loi trapézoïdale. Un modèle plan de la chaîne cinématique ouverte représente la partie inférieure
%de l’exosquelette en position debout et fléchie.
%
%
%\begin{center}
%\includegraphics[width=\linewidth]{images/fig_03}
%%\textit{}
%\end{center}
%
%
%%Selon le cahier des charges, pour assurer une bonne synchronisation des axes, l’exigence de précision statique suite à une entrée de type échelon, de type rampe ou de type accélération doit être inférieure à 1\%.
%
%
%On donne le paramétrage du modèle proposé. 
%
%\begin{center}
%\includegraphics[width=\linewidth]{images/fig_04}
%%\textit{}
%\end{center}
%
%\textbf{Hypothèses : }
%\begin{itemize}
%\item le référentiel lié au repère $\mathcal{R}_0\repere{A}{x_0}{y_0}{z_0}$ est galiléen et est fixe par rapport à la terre;
%\item le point $O_2$ représentant la hanche se déplace verticalement selon la direction $\axe{O_0}{z_0}$;
%\item l’angle $\alpha$ entre la charge transportée et la verticale $\vect{z_0}$ reste constant;
%\item le point d’appui $A$ du pied sur le sol est considéré fixe par rapport à la terre;
%\item lors du mouvement étudié la jambe (1) reste perpendiculaire au pied (3).
%\end{itemize}
%
%\textbf{Données : }
%\begin{itemize}
%\item $\theta_{10}=\angl{y_0}{y_1}=\angl{z_0}{z_1}$;
%\item $\theta_{21}=\angl{y_1}{y_2}=\angl{z_1}{z_2}$;
%\item $\alpha=\text{constante}$;
%\item $L=\sqrt{\left(l_2+l_3\right)^2+l_4^2}$.
%\end{itemize}
%
%\fi
%
%\subparagraph{}\textit{Déterminer littéralement les coordonnées opérationnelles $l_4$ et $h(t)$ en fonction des coordonnées articulaires $\theta_{10}$, $\theta_{21}$ et des paramètres dimensionnels $L$ et $l_1$.}
%\ifprof
%\begin{corrige}
%On a $\vect{A O_1}+\vect{O_1 O_2}+\vect{O_2 O_0}+\vect{O_0 A}=\vect{0}$ soit 
%$L\vect{y_1}+l_1\vect{y_2}-h(t)\vect{z_0}+l_4\vect{y_0}=\vect{0}$. 
%En projetant sur $\vect{y_0}$ et $\vect{z_0}$ on a :
%$$
%\left\{
%\begin{array}{l}
%L\cos\theta_{10} +l_1\cos\left(\theta_{10}+\theta_{21}\right)+l_4={0} \\
%L\sin\theta_{10}  +l_1\sin\left(\theta_{10}+\theta_{21}\right)-h(t) ={0} \\
%\end{array}
%\right.
%$$
%
%En projetant sur $\vect{y_1}$ et $\vect{z_1}$ on a :
%$$
%\left\{
%\begin{array}{l}
%L+l_1\cos\theta_{21}-h(t)\sin\theta_{10}+l_4\cos\theta_{10}={0} \\
%l_1\sin\theta_{21}-h(t)\cos\theta_{10}-l_4\sin\theta_{10}={0}
%\end{array}
%\right.
%$$
%
%
%\end{corrige}
%\else
%\fi
%
%
%\subparagraph{}\textit{Déterminer le modèle articulaire inverse $\theta_{10}$ et $\theta_{21}$ en fonction de $l_1$, $l_4$, $L$ et $h(t)$.}
%\ifprof
%\begin{corrige}
%Pour exprimer $\theta_{10}$, on peut utiliser le premier système d'équation : 
%$$
%\left\{
%\begin{array}{l}
%L\cos\theta_{10}+l_4 =-l_1\cos\left(\theta_{10}+\theta_{21}\right) \\
%L\sin\theta_{10}-h(t)  =-l_1\sin\left(\theta_{10}+\theta_{21}\right) \\
%\end{array}
%\right.
%$$
%En élevant les expressions au carré, on a alors : 
%$
%l_1^2 = \left(L\cos\theta_{10}+l_4 \right)^2+ \left(L\sin\theta_{10}-h(t)\right)^2
%$
%$
%\Leftrightarrow 
%l_1^2 = L^2+l_4^2 +h(t)^2+2Ll_4\cos\theta_{10} -2Lh(t)\sin\theta_{10}
%$
%
%$
%\Leftrightarrow 
%\dfrac{l_1^2 - L^2-l_4^2 -h(t)^2}{2L}=l_4\cos\theta_{10} -h(t)\sin\theta_{10}
%$
%
%En utilisant l'indication, on a : 
%$$
%\dfrac{l_4}{\sqrt{l_4^2 + h(t)^2}}\cos\theta_{10} +\dfrac{-h(t)}{\sqrt{l_4^2 + h(t)^2}}\sin\theta_{10}=\dfrac{l_1^2 - L^2-l_4^2 -h(t)^2}{2L{\sqrt{l_4^2 + h(t)^2}}}
%$$
%En conséquence, on pose $\cos\varphi=\dfrac{l_4}{\sqrt{l_4^2 + h(t)^2}}$ et 
%$\sin\varphi = \dfrac{-h(t)}{\sqrt{l_4^2 + h(t)^2}}$. 
%En conséquences %$\tan\varphi = \dfrac{\dfrac{-h(t)}{\sqrt{l_4^2 + h(t)^2}}}{\dfrac{l_4}{\sqrt{l_4^2 + h(t)^2}}}=\dfrac{-h(t)}{l_4}$.
%$\tan\varphi =\dfrac{-h(t)}{l_4}$.
%
%
%Par suite, $\cos\left( \theta_{10} - \varphi \right) =\dfrac{l_1^2 - L^2-l_4^2 -h(t)^2}{2L{\sqrt{l_4^2 + h(t)^2}}}$. On a donc 
%$\theta_{10} =\arccos \left( \dfrac{l_1^2 - L^2-l_4^2 -h(t)^2}{2L{\sqrt{l_4^2 + h(t)^2}}}\right) + \varphi$.
%
%
%Au final, 
%$$\theta_{10} =\arccos \left( \dfrac{l_1^2 - L^2-l_4^2 -h(t)^2}{2L{\sqrt{l_4^2 + h(t)^2}}}\right) + \arctan\left(\dfrac{-h(t)}{l_4} \right).$$
%
%Pour exprimer  $\theta_{21}$ on réutilise le premier système d'équations :
%$
%\left\{
%\begin{array}{l}
%-l_4 =l_1\cos\left(\theta_{10}+\theta_{21}\right) + L\cos\theta_{10} \\
%h(t)  =l_1\sin\left(\theta_{10}+\theta_{21}\right) + L\sin\theta_{10}\\
%\end{array}
%\right.
%$
%
%On a alors 
%$l_4 ^2 +h(t)^2= L^2 + l_1^2 + 2l_1L\left(\cos\theta_{10}\cos\left(\theta_{10}+\theta_{21}\right) +\sin\left(\theta_{10}+\theta_{21}\right) \sin\theta_{10}\right)$.
%En conséquences, 
%$\dfrac{l_4 ^2 +h(t)^2- L^2 - l_1^2}{2l_1L} = \cos\theta_{10}\cos\left(\theta_{10}+\theta_{21}\right) +\sin\left(\theta_{10}+\theta_{21}\right) \sin\theta_{10}$
%$= \cos\left(\theta_{10}+\theta_{21}-\theta_{10}\right) $.
%D'où $\theta_{21}=\arccos\left(\dfrac{l_4 ^2 +h(t)^2- L^2 - l_1^2}{2l_1L}  \right)$.
%\end{corrige}
%
%
%\else
%\fi
%
%\ifprof
%\else
%\begin{methode}
%Lorsqu'on a une équation de la forme $A\cos\theta_{10}+B\sin\theta_{10}=C$. On peut normer cette équation en la mettant sous la forme $\dfrac{A}{\sqrt{A^2+B^2}}\cos\theta_{10}+\dfrac{B}{\sqrt{A^2+B^2}}\sin\theta_{10}=\dfrac{C}{\sqrt{A^2+B^2}}$.
%On pose alors $\cos\varphi = \dfrac{A}{\sqrt{A^2+B^2}}$ et $\sin\varphi=\dfrac{B}{\sqrt{A^2+B^2}}$. On a alors $\cos\left( \theta_{10} - \varphi \right)=\dfrac{C}{\sqrt{A^2+B^2}}$.
%\end{methode}
%\fi
%
%\subsection*{Élaboration du modèle cinématique}
%\begin{obj}
%En vue de dimensionner le moteur du genou, déterminer la vitesse articulaire en fonction de la  vitesse opérationnelle.
%\end{obj}
%
%\subparagraph{}\textit{Déterminer à partir du modèle articulaire inverse la vitesse angulaire 
%$\theta_{21}$ en fonction de $h(t)$, $\dot{h}(t)$, $l_1$, $L$ et $\sin\theta_{21}$.}
%\ifprof
%\begin{corrige}
%On a vu que $\cos\theta_{21}=\dfrac{l_4 ^2 +h(t)^2- L^2 - l_1^2}{2l_1L}  $.
%En dérivant, on a donc $- \dot{\theta}_{21} \sin\theta_{21}=\dfrac{2\dot{h}(t)h(t)}{2l_1L}$.
%Au final,  $\dot{\theta}_{21} =-\dfrac{\dot{h}(t)h(t)}{l_1L\sin\theta_{21}}$.
%\end{corrige}
%\else
%\fi
%
%\ifprof
%\else
%Un modèle multiphysique a permis de déterminer les conditions suivantes correspondant à la vitesse maximale : $t=\SI{1,5}{s}$, $h(t=1,5)=\SI{0,829}{m}$, $\dot{h}(t=1,5)=\SI{0,422}{m.s^{-1}}$ et $\theta_{21} = 55,9\degres$. Les longueurs $l_1$ et $L$ valent
%respectivement \SI{43,1}{cm} et \SI{51,8}{cm}. Le réducteur de vitesse utilisé a un rapport de réduction égal à $r=\dfrac{1}{120}$.
%\fi
%
%\subparagraph{}\textit{Déterminer la valeur maximale de la vitesse angulaire $\dot{\theta}_{21}$ et $\text{rad s}^{-1}$ puis celle de la fréquence de rotation d’un moteur de genou en $\text{tr min}^{-1}$.}
%\ifprof
%\begin{corrige}
%On a : $\dot{\theta}_{21} =-\dfrac{0,422 \times 0,829}{0,431\times 0,518\sin \left(55,9\right)}\simeq -\SI{1,89}{rad.s^{-1}}$. Soit une fréquence de rotation du moteur de \SI{2168}{tr.min^{-1}}.
%\end{corrige}
%\else
%\fi


\subsection*{Élaboration du modèle dynamique}

\begin{obj}
Dimensionner le moteur situé au niveau d’un genou permettant à l’exosquelette de soulever une masse de \SI{60}{kg} de la position accroupie à la position debout.
\end{obj}

Ces calculs visent à déterminer l’équation dynamique qui permet d’obtenir le couple moteur (minimal) en fonction des caractéristiques géométriques et massique de la charge à soulever ainsi que des conditions d’utilisation.
Le modèle d’étude est celui représenté à la figure suivante correspondant au modèle d’étude plan position fléchie.
\begin{center}
\includegraphics[width=\linewidth]{images/fig_03}
%\textit{}
\end{center}

\begin{center}
\includegraphics[width=.7\linewidth]{images/fig_14}
%\textit{}
\end{center}


\noindent\textbf{Hypothèses :}
\begin{itemize}
\item L’étude est modélisable dans le plan.
\item Toutes les liaisons sont supposées parfaites.
\item Les inerties des pièces sont négligées sauf la masse de la charge à soulever.
\item L’angle $\alpha$ entre la charge transportée et la verticale$\vect{z_0}$ reste constant.
\item $G_4$, centre de gravité de la charge transportée (4), reste en permanence à la verticale du point $A$ d’appui au sol.
\end{itemize}

\noindent\textbf{Données :}
\begin{itemize}
\item $\vect{O_1G_4}=\lambda(t)\vect{z_0}-L\cos\theta_{10}\vect{y_0}$;
\item Accélération de la pesanteur $g=\SI{9,81}{m.s^{-2}}$;
\item Longueur de la cuisse $l_1 = \SI{43,1}{cm}$.
\item Longueur de la jambe $l_2 = \SI{43,3}{cm}$.
\item Longueur de l'articulation de la cheville à la plante arrière du pied $l_3 = \SI{6,9}{cm}$.
\item Longueur de la plante arrière du pied au point d’appui sur le sol $l_4 = \SI{13}{cm}$.
\item Longueur $\vect{O_0O_1}=L\vect{y_1}$ avec $L=\SI{51,8}{cm}$.
\item Rapport de réduction : $r=\dfrac{1}{120}$.
\end{itemize}

On note E=\{cuisse(2)+charge transportée(4)\}. 

\subparagraph{} \textit{Donner qualitativement le mouvement de E par rapport à 0. Tracer le graphe de structure du système.}

\ifprof
\begin{corrige}
Étant donné que l'on souhaite que l'angle $\alpha$ reste constant pendant la levée d'une charge, le mouvement de $E$ sera donc un mouvement de translation rectiligne.  
\end{corrige}
\else
\fi


\subparagraph{} \textit{Déterminer $\vectmc{O_1}{E}{0}\cdot \vect{x_0}$ en fonction de $m_4$, $\dot{h}(t)$, $L$ et $\cos\theta_{10}$.}

\ifprof
\begin{corrige}
$E$ étant en translation, on a $\vectmc{G_4}{E}{0}=\vect{0}$. On a alors 
 $\vectmc{O_1}{E}{0}=\vectmc{G_4}{E}{0}+\vect{O_1G_4}\wedge\vectrc{E}{0}$.

Par ailleurs, $\vectrc{E}{0}=m_4\vectv{G_4}{E}{0} = m_4\dot{h}(t)\vect{z_0}$. 

On a donc :
$ \vectmc{O_1}{E}{0}\cdot \vect{x_0} 
= \left(\left( \lambda(t)\vect{z_0}-L\cos\theta_{10}\vect{y_0}\right)\wedge m_4\dot{h}(t)\vect{z_0}\right) \cdot \vect{x_0}
=  -Lm_4\cos\theta_{10}\dot{h}(t)$.
 
 
\end{corrige}
\else
\fi



\subparagraph{} \textit{Déduire $\vectmd{O_1}{E}{0}\cdot \vect{x_0}$ en fonction de $m_4$, $\ddot{h}(t)$, $L$ et $\cos\theta_{10}$.}

\ifprof
\begin{corrige}
\subsubsection*{Méthode 1 -- Calcul de $\vectmd{G_4}{E}{0}$ et déplacement}
On a $\vectmd{G_4}{E}{0}= \dfrac{\dd \vectmc{G_4}{E}{0}}{\dd t}= \vect{0}$. En conséquences, 
$ \vectmd{O_1}{E}{0}\cdot \vect{x_0} 
= \left(\left( \lambda(t)\vect{z_0}-L\cos\theta_{10}\vect{y_0}\right)\wedge m_4\ddot{h}(t)\vect{z_0}\right) \cdot \vect{x_0} =  -Lm_4\cos\theta_{10}\ddot{h}(t)$.

\subsubsection*{Méthode 2 -- Calcul de $\vectmd{O_1}{E}{0}$}
On a aussi $\vectmd{O_1}{E}{0} = \left(\dfrac{d \vectmc{O_1}{E}{0}}{\dd t}\right)+m_4\vect{V\left(O_1/0\right)}\wedge\vectv{G_4}{E}{0} $. 

Par suite on a 
$\left(\vectv{O_1}{E}{0}\wedge\vectv{G_4}{E}{0}\right)\vect{x_0}=
\left(\left( L\vect{y_1}\wedge \dot{\theta}_{10} \vect{x_0} \right) \wedge\dot{h}(t)\vect{z_0}\right)\vect{x_0} 
= \left( -L\dot{\theta}_{10} \vect{z_1} \wedge\dot{h}(t)\vect{z_0}\right)\vect{x_0} $ 
$= -L\dot{\theta}_{10} \dot{h}(t) \cos \theta_{10} $. 

Enfin, 
 $\vectmd{O_1}{E}{0}\cdot\vect{x_0}=-Lm_4\cos\theta_{10}\ddot{h}(t)+Lm_4\dot{\theta}_{10}\sin\theta_{10}\dot{h}(t)-m_4L\dot{\theta}_{10} \dot{h}(t) \cos \theta_{10}$.
\textbf{ :( Chercher l'erreur....}

 
\end{corrige}
\else
\fi

La loi d'évolution de la vitesse de la hanche est donnée à la figure suivante. 

\begin{center}
\includegraphics[width=\linewidth]{images/fig_11}
%\textit{}
\end{center}


\subparagraph{} \textit{Déterminer l’expression littérale du couple $C_r$ exercé par l’arbre de sortie du réducteur sur le genou imposé par la loi d’évolution de la hanche et calculer numériquement ce couple pour une valeur de $\theta_{10}$ égale à
54,5\degres correspondant à la valeur maximale du couple.}

\ifprof
\begin{corrige}
\end{corrige}
\else
\fi



\subparagraph{} \textit{Calculer le couple $C_m$ au niveau de l’arbre moteur du genou en prenant un facteur de perte $\eta = 0,75$ (estimé à l’aide du modèle multiphysique).}
\ifprof
\begin{corrige}
\end{corrige}
\else
\fi



\subparagraph{} \textit{Expliquer en moins de 5 lignes comment estimer un rendement à partir d'un modèle multiphysique.}
\ifprof
\begin{corrige}
\end{corrige}
\else
\fi

\subsection*{Validation du dimensionnement du moteur}
\begin{obj}
Vérifier que le moteur choisi convient pour une utilisation intensive comprenant 4 cycles par minute
de descente suivie d’une montée.
\end{obj}

Le cycle suivant obtenu à l’aide du modèle multiphysique de représente l’évolution du couple moteur,
et ce en tenant compte du moment d’inertie du rotor, sur un cycle de période $T=\SI{15}{s}$.


\noindent Quatre phases sont définies sur cette période :
\begin{itemize}
\item phase 1 pour $0 \leq t < \SI{2}{s}$, valeur efficace du couple moteur $C_1 = \SI{0,838}{Nm}$ ;
\item phase 2 pour $2 \leq t < \SI{4}{s}$, couple moteur constant $C_2 = -\SI{0,912}{Nm}$ ;
\item phase 3 pour $4 \leq t < \SI{6}{s}$, valeur efficace du couple moteur $C_3 = \SI{0,838}{Nm}$;
\item phase 4 pour $6 \leq t < \SI{15}{s}$, couple moteur nul.
\end{itemize}



\begin{center}
\includegraphics[width=\linewidth]{images/fig_12}
%\textit{}
\end{center}


\subparagraph{} \textit{Préciser à quels mouvements correspondent les 4 phases de ce cycle.}

\ifprof
\begin{corrige}
\end{corrige}
\else
\fi

Le couple efficace est également appelé couple thermiquement équivalent, il est défini par :
$C_{\text{eff}}=\sqrt{\dfrac{1}{T}\int\limits_0^Tc(t)^2 \dd t}$.



\subparagraph{} \textit{Calculer la valeur efficace du couple moteur du genou pour ce cycle périodique de \SI{15}{s}.}

Le couple moteur varie entre -\SI{1,156}{Nm} et \SI{0,596}{Nm}.
Les caractéristiques du moteur choisi sont :
\begin{itemize}
\item vitesse à vide de \SI{3120}{tr.min^{-1}} pour une alimentation nominale en amont de l’onduleur de \SI{36}{V};
\item couple permanent admissible de \SI{0,560}{Nm};
\item pente de la courbe de la vitesse en fonction du couple de \SI{423}{tr.min^{-1}N^{-1}m^{-1}}.
\end{itemize}

De plus une étude cinématique précédente a montré que le moteur permettant d'actionner le moteur doit pouvoir atteindre une vitesse de \SI{2200}{tr.min^{-1}}.
\ifprof
\begin{corrige}
\end{corrige}
\else
\fi

\subparagraph{} \textit{Conclure quant au choix de ce moteur au regard de la valeur maximale de la vitesse angulaire calculée lors d'une étude précédente et du couple efficace calculé à la question précédente.}

\ifprof
\begin{corrige}
\end{corrige}
\else
\fi

\ifprof
\else

\noindent 
\footnotesize
\begin{tabular}{|p{.95\linewidth}|}
\hline
Éléments de corrigé :
\begin{enumerate}
\item ....
\end{enumerate} \\
\hline
\end{tabular}
\normalsize
\fi

\ifprof
\else
\end{multicols}
\fi



\begin{center}
\includegraphics[width=\linewidth]{images/fig_07}
%\textit{}
\end{center}

%\begin{center}
%\includegraphics[width=.65\linewidth]{images/cor_01}
%%\textit{}
%\end{center}

\end{document}
