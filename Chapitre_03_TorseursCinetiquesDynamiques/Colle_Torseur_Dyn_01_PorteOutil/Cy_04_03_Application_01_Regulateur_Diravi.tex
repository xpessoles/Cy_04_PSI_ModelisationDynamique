\documentclass[10pt,fleqn]{article} % Default font size and left-justified equations
\usepackage[%
    pdftitle={Modélisation dynamique : cinétique},
    pdfauthor={Xavier Pessoles}]{hyperref}

    
\input{style/new_style}
\input{style/macros_SII}

\fichetrue
%\fichefalse

\proftrue
\proffalse

\tdtrue
%\tdfalse

\courstrue
\coursfalse


\def\discipline{Sciences \\Industrielles de \\ l'Ingénieur}
\def\xxtete{Sciences Industrielles de l'Ingénieur}

\def\classe{\textsf{PSI$\star$ -- MP}}
\def\xxnumpartie{Cycle 04}
\def\xxpartie{Modéliser le comportement des systèmes mécaniques dans le but d'établir une loi de comportement ou de déterminer des actions mécaniques en utilisant le PFD}

\def\xxnumchapitre{Chapitre 3 \vspace{.2cm}}
\def\xxchapitre{\hspace{.12cm} Cinétique et application du Principe Fondamental de la Dynamique}




\def\xxtitreexo{Porte-outil d’affûtage}%Motorisation du moteur Haibike}
\def\xxsourceexo{\hspace{.2cm} \footnotesize{Équipe PT -- PT$\star$ La Martinière Monplaisir}}


\def\xxposongletx{2}
\def\xxposonglettext{1.45}
\def\xxposonglety{20}
%\def\xxonglet{Part. 1 -- Ch. 3}
\def\xxonglet{Cycle 04}

\def\xxactivite{Colle 01}
\def\xxauteur{\textsl{Équipe PT -- PT$\star$ La Martinière Monplaisir}}

\def\xxcompetences{%
\textsl{%
\textbf{Savoirs et compétences :}\\
%\begin{itemize}[label=\ding{112},font=\color{ocre}] 
%\item \textit{Mod2.C13} : centre d'inertie
%\item \textit{Mod2.C14} : opérateur d'inertie
%\item \textit{Mod2.C15} : matrice d'inertie
%\end{itemize}
}}
\def\xxfigures{
%\includegraphics[width=.4\linewidth]{images/fig_00}
}%figues de la page de garde


\def\xxpied{%
Cycle 04 -- Modélisation mécanique -- Cinétique\\% afin de valider leurs performances.\\
Chapitre 3 -- \xxactivite%
}

\setcounter{secnumdepth}{5}
%---------------------------------------------------------------------------

\usepackage{pgfplots}
\begin{document}
\def\pathfig{images}
%\chapterimage{png/Fond_Cin}
\input{style/new_pagegarde}
\vspace{5cm}
\pagestyle{fancy}
\thispagestyle{plain}

\def\columnseprulecolor{\color{ocre}}
\setlength{\columnseprule}{0.4pt} 

\def\pathfig{images}

\ifprof
\else
\begin{multicols}{2}
\fi

Le dispositif porte-outil d'une machine d'affûtage est composé de trois solides 1, 2 et 3. 

Le repère R0(O, , , ), avec (O,  ) vertical ascendant, est lié au bâti 0 de la machine. Il est supposé galiléen. 
Toutes les liaisons sont supposées parfaites.
%
%Le repère R1(O,  , , ) est lié au support tournant 1 en liaison pivot d'axe (O,  ) avec le bâti 0.
%La position de 1 (de centre d’inertie O) est repérée par  = ( , ) = ( , ). 
%On note I1 le moment d'inertie de 1 par rapport à l'axe (O,  ) et H le point tel que  = h .
%Le repère R2(H,  , , ) est lié au bras pivotant 2 en liaison pivot d'axe (H,  ) avec 1 
%La position de 2 est repérée par  = ( , ) = ( , ). 
%On note m2 la masse de 2, de centre d’inertie H (on est sympa!), de matrice d’inertie J(H,2) =  
%Le repère R3(G,  , , ) est lié au porte-outil 3 (avec l’outil à affûter tenu par le mandrin) en liaison pivot glissant d'axe (H,  ) avec 2.
%La position de 3 est repérée par  = ( , ) = ( , ) et par  =  . 
%On note m3 la masse de 3, de centre d’inertie G, de matrice d’inertie J(G,3) =  
%1.	Justifier la forme de la matrice de la pièce 3.
%2.	Calculer  
%3.	Indiquer la méthode permettant de calculer le torseur dynamique en G de 3 en mouvement par rapport à R0 en projection sur  
%4.	Calculer le moment  dynamique en H appliqué à l'ensemble {2, 3}  en mouvement par rapport à R0 en projection sur   
%5.	Calculer le moment dynamique en O appliqué à l'ensemble {1, 2, 3} en mouvement par rapport à R0 en projection sur  .


\ifprof
\else
\end{multicols}
\fi



%\newpage


\end{document}
\begin{center}
\includegraphics[width=\linewidth]{images/}
\end{center}

\subparagraph{}
\textit{}
\ifprof
\begin{corrige}
\end{corrige}
\else
\fi
