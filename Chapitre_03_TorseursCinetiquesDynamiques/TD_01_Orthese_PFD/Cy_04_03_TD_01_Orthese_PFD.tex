\documentclass[10pt,fleqn]{article} % Default font size and left-justified equations
\usepackage[%
    pdftitle={Modélisation dynamique : cinétique},
    pdfauthor={Xavier Pessoles}]{hyperref}

    
\input{style/new_style}
\input{style/macros_SII}

\fichetrue
%\fichefalse

\proftrue
\proffalse

\tdtrue
%\tdfalse

\courstrue
\coursfalse


\def\discipline{Sciences \\Industrielles de \\ l'Ingénieur}
\def\xxtete{Sciences Industrielles de l'Ingénieur}

\def\classe{\textsf{PSI$\star$ -- MP}}
\def\xxnumpartie{Cycle 04}
\def\xxpartie{Modéliser le comportement des systèmes mécaniques dans le but d'établir une loi de comportement ou de déterminer des actions mécaniques en utilisant le PFD}

\def\xxnumchapitre{Chapitre 3 \vspace{.2cm}}
\def\xxchapitre{\hspace{.12cm} Cinétique et application du Principe Fondamental de la Dynamique}




\def\xxtitreexo{Orthèse d'épaule}%Motorisation du moteur Haibike}
\def\xxsourceexo{\hspace{.2cm} \footnotesize{Centrale Supélec 2010 -- PSI}}


\def\xxposongletx{2}
\def\xxposonglettext{1.45}
\def\xxposonglety{20}
%\def\xxonglet{Part. 1 -- Ch. 3}
\def\xxonglet{Cycle 04}

\def\xxactivite{TD 1}
\def\xxauteur{\textsl{Xavier Pessoles}}

\def\xxcompetences{%
\textsl{%
\textbf{Savoirs et compétences :}\\
%\begin{itemize}[label=\ding{112},font=\color{ocre}] 
%\item \textit{Mod2.C13} : centre d'inertie
%\item \textit{Mod2.C14} : opérateur d'inertie
%\item \textit{Mod2.C15} : matrice d'inertie
%\end{itemize}
}}
\def\xxfigures{
\includegraphics[width=.4\linewidth]{images/fig_01}
}%figues de la page de garde


\def\xxpied{%
Cycle 04 -- Modélisation mécanique -- Cinétique\\% afin de valider leurs performances.\\
Chapitre 3 -- \xxactivite%
}

\setcounter{secnumdepth}{5}
%---------------------------------------------------------------------------

\usepackage{pgfplots}
\begin{document}
\def\pathfig{images}
%\chapterimage{png/Fond_Cin}
\input{style/new_pagegarde}
\vspace{5cm}
\pagestyle{fancy}
\thispagestyle{plain}

\def\columnseprulecolor{\color{ocre}}
\setlength{\columnseprule}{0.4pt} 

\def\pathfig{images}

\ifprof
\else
\begin{multicols}{2}
\fi

\subsection*{Mise en situation}

Le support de cette étude est une orthèse portable, de type exosquelette, qui contribue au développement de la tonicité musculaire de l’épaule et du bras. Installée dans le dos de l’individu,
et liée à la fois au bras et à la main, elle offre une résistance aux mouvements de la main. Ainsi, le thérapeute
peut réaliser des protocoles très fins de rééducation en programmant des spectres d’efforts résistants pour chaque
mouvement du patient. Le travail du patient peut également être optimisé en le plaçant dans un environnement
de réalité virtuelle permettant de visualiser les situations de travail conçues par le thérapeute.

\begin{center}
\includegraphics[width=\linewidth]{images/fig_02}
\end{center}
\begin{obj}
L’objectif est de mettre en place une loi de commande utilisée, par exemple, pour des
situations de travail où le patient peut déplacer le bras et doit appliquer une force prédéterminée par le
physiothérapeute, dépendante des positions des articulations. Dans le cadre de cette étude, l’effort est
élastique et caractérisé par une raideur de torsion. 
\end{obj}

La synthèse de cette loi de commande sera faite en
deux étapes : dans un premier temps, la mise en équation de l’exosquelette (limité à deux axes pour des
raisons de simplicité) sera effectuée en vue d’obtenir un modèle dynamique ; dans un deuxième temps,
la loi de commande sera déterminée en utilisant le modèle dynamique établi au préalable. Il s’agira, de
plus, de valider le dimensionnement de la chaîne de motorisation.

\begin{center}
\begin{tabular}{|p{.45\linewidth}|p{.45\linewidth}|}
\hline 
Module de l'effort de manipulation maximal en régime permanent & \SI{50}{N} \\ \hline 
Compensation du couple statique (dû à la pesanteur) & Totale \\ \hline 
Raideurs $(K_1,K_2)$ de maintien (pour ce critère, seule la force $Z_F$ est
considérée). & $|\Delta Z_F/\Delta \gamma|=K_1\geq \SI{500}{N.rad^{-1}} (\pm 5\%)$ \\
& $|\Delta Z_F/\Delta \delta|=K_2\geq \SI{500}{N.rad^{-1}} (\pm 5\%)$ \\\hline 
\end{tabular}
\end{center}
L’actionneur ne peut fournir, en régime permanent, sur l’axe de l’articulation qu’un
couple de module inférieur à \SI{50}{Nm}. On suppose, de plus, qu’en régime transitoire le couple maximal peut
atteindre quatre fois la valeur maximale autorisée en régime permanent.
On s’intéresse ici à une situation de travail où les relations entre les variations des positions angulaires du
bras et de l’avant bras ($\gamma$ et $\delta$) et la variation de la force $Z_F$ (ces grandeurs seront définies par la suite dans la section III.A****) exercée par le patient sont équivalentes à des raideurs de torsion de valeurs $(K_1,K_2)$.

La structure de commande retenue est représentée par le schéma de la figure suivante où :
\begin{itemize}
\item $q$ et $\dot{q}$ sont respectivement les vecteurs des angles et des vitesses angulaires des articulations;
\item une boucle externe génère les trajectoires (positions, vitesses et accélérations) et éventuellement un contexte
de travail;
\item une boucle interne (de loi non linéaire) génère les couples souhaités sur chaque axe (articulation) à partir
des mesures des angles et des vitesses angulaires des articulations et éventuellement des données issues du
générateur de trajectoire ;
\item un ensemble d’actionneurs fournit les couples, sur les axes des articulations, identiques aux couples de
référence $C_a = C_{\text{ref}}$.
\end{itemize}

\begin{center}
\includegraphics[width=\linewidth]{images/fig_03}
\end{center}



\subsection*{Modélisation dynamique «deux axes» de l’exosquelette}
\begin{obj}
Le but de cette partie est d’établir un modèle dynamique du bras et de l’avant-bras dans un plan vertical
donné. Ces deux ensembles sont soumis aux actions de la pesanteur, des couples des deux moteurs montés
dans le bras et de la force extérieure exercée sur l’extrémité de l’avant-bras. Le cadre de l’étude se limite
aux mouvements de deux axes (les deux autres axes étant supposés fixes).
\end{obj}

Le système étudié se réduit donc à l’ensemble \{Bras + Avant-bras\} relativement au reste du dispositif supposé
fixe : on suppose que les angles d’abduction/adduction et de rotation interne/rotation externe de l’épaule sont
maintenus identiquement nuls par l’action des moteurs situés dans la partie dorsale du dispositif (non étudiée
ici). Le paramétrage se réduit donc à la situation de la figure suivante qui représente l’ensemble étudié dans un plan
$\left(\vect{x};\vect{z} \right)$ donné, où l’on choisit $\vect{z}$ vertical dans le sens descendant. Le tableau précise les différents
paramètres utiles pour le calcul de dynamique envisagé.

\begin{center}
\includegraphics[width=\linewidth]{images/fig_04}
\end{center}


\begin{center}
\includegraphics[width=\linewidth]{images/fig_05}
\end{center}


\subparagraph{}\textit{Exprimer littéralement, au point $G_2$ et dans le repère $R_1$, le torseur dynamique du mouvement du solide
\{Avant-bras\} par rapport au référentiel fixe $R_0$ supposé galiléen : $\torseurdyn{\text{Avant-bras}}{R_0}_{G2,\base{x_1}{y_1}{z_1}}$.}

Les différentes actions mécaniques agissant sur le dispositif sont les suivantes :
\begin{itemize}
\item l’action de la pesanteur sur les solides \{Bras\} et \{Avant-bras\} ;
\item l’action du bâti sur le solide \{Bras\} au travers de la liaison pivot d’axe $\axe{O}{y}$ et de torseur d’action mécanique écrit sous la forme générique suivante : 
$\torseurstat{T}{\text{Bâti}}{\text{Bras}}=\torseurl{X_1\vect{x_1}+Y_1\vect{y_1}+Z_1\vect{z_1}}{L_1\vect{x_1}+M_1\vect{y_1}+N_1\vect{z_1}}{O}$ où les paramètres $\left(X_1, Y_1, Z_1, L_1, M_1, N_1\right)$ sont inconnus;
\item l’action du premier actionneur sur le solide \{Bras\} :
$\torseurstat{T}{\text{Actionneur 1}}{\text{Bras}}=\torseurl{\vect{0}}{C_1(t) \vect{y}}{O}$ où le couple $C_1(t)$ exercé est connu au cours du temps;
\item  l’action du solide \{Bras\} sur le solide \{Avant-bras\} au travers de la liaison pivot d’axe $\axe{A}{y}$ et de torseur d’action mécanique écrit sous la forme générique suivante : 
$\torseurstat{T}{\text{Bras}}{\text{Avant-bras}}=\torseurl{X_2\vect{x_1}+Y_2\vect{y}+Z_2\vect{z_1}}{L_2\vect{x_1}+M_2\vect{y}+N_2\vect{z_1}}{A}$ où les paramètres $\left(X_2, Y_2, Z_2, L_2, M_2, N_2\right)$ sont inconnus;
\item les actions du second actionneur sur le solide \{Bras\} et le solide \{Avant-bras\}, respectivement notées :
$\torseurstat{T}{\text{Actionneur 2}}{\text{Bras}}=\torseurl{\vect{0}}{-C_2(t) \vect{y}}{A}$ et 
$\torseurstat{T}{\text{Actionneur 2}}{\text{Avant-bras}}=\torseurl{\vect{0}}{C_2(t) \vect{y}}{A}$ où le couple $C_2(t)$ exercé est connu au cours du temps;
\item l’action du patient sur l’avant-bras, modélisée par une force appliquée à l’extrémité $B$ de l’avant-bras et
définie par : 
$\torseurstat{T}{\text{Force}}{\text{Avant-bras}}=\torseurl{X_F\vect{x}+Z_F\vect{z}}{\vect{0}}{B}$. 
\end{itemize}

On veut déterminer les deux équations permettant de décrire le mouvement des deux axes de l’orthèse. On
suppose pour cela que les deux liaisons pivots sont parfaites.

Le PFD permet d'obtenir la relation suivante : 
$$
\begin{array}{ll}
C_1(t)=&
\left(B_1+B_2 +m_1 \lambda_1^2 + m_2 l_1^2 + m_2 \lambda_2^2 \right)\ddot{\gamma} +\left( B_2 + m_2\lambda_2^2\right) \ddot{\delta}\\
& +m_2 l_1 \left( \lambda_2 \left(2\ddot{\gamma}+\ddot{\delta} \right)\cos \delta + \lambda_2 \left( \dot{\gamma}^2-\left( \dot{\gamma} + \dot{\delta}\right)^2\right) \sin\delta\right) \\
& + m_1g\lambda_1\sin\gamma + m_2 g \left(l_1 \sin \gamma+\lambda_2 \sin \left(\gamma+ \delta\right) \right)\\
& -X_F \left( l_1 \cos \gamma +l_2 \cos \left( \gamma+\delta\right) \right) + Z_F \left( l_1 \sin \gamma + l_2 \left( \gamma + \delta \right)\right)
\end{array}
$$
\end{multicols}

\subparagraph{}
\textit{Détailler la démarche qui a permis d’obtenir cette équation, on précisera en particulier l’isolement,
le bilan des Actions Mécaniques Extérieures et le choix des équations utilisées.}
\ifprof
\begin{corrige}
\end{corrige}
\else
\fi

\subparagraph{}
\textit{Appliquer la démarche pour retrouver l'équation donnée.}
\ifprof
\begin{corrige}
\end{corrige}
\else
\fi

\subparagraph{}
\textit{Écrire une deuxième relation issue du Principe Fondamental de la Dynamique, indépendante de la
précédente, faisant intervenir le couple $C_2(t)$, et qui permette de ne pas faire apparaître les composantes $L_1$, $M_1$, $N_1$, $L_2$, $M_2$, $N_2$ des torseurs des actions de liaison. On détaillera la démarche de la même façon que ci-dessus.}

\ifprof
\begin{corrige}
\end{corrige}
\else
\fi


\subparagraph{}
\textit{}
\ifprof
\begin{corrige}
\end{corrige}
\else
\fi


\subparagraph{}
\textit{}
\ifprof
\begin{corrige}
\end{corrige}
\else
\fi

\subparagraph{}
\textit{}
\ifprof
\begin{corrige}
\end{corrige}
\else
\fi


\subparagraph{}
\textit{}
\ifprof
\begin{corrige}
\end{corrige}
\else
\fi


\subparagraph{}
\textit{}
\ifprof
\begin{corrige}
\end{corrige}
\else
\fi

\end{document}
\begin{center}
\includegraphics[width=\linewidth]{images/}
\end{center}

\subparagraph{}
\textit{}
\ifprof
\begin{corrige}
\end{corrige}
\else
\fi
