\documentclass[10pt,fleqn]{article} % Default font size and left-justified equations
\usepackage[%
    pdftitle={Equilibrage des solides en rotation},
    pdfauthor={Xavier Pessoles}]{hyperref}

\input{style/new_style}
\input{style/macros_SII}
\usepackage{bm}
\fichetrue
\fichefalse

\proftrue
%\proffalse

%\tdtrue
\tdfalse

\courstrue
%\coursfalse



% -------------------------------------
% Déclaration des titres
% -------------------------------------

\def\discipline{Sciences \\Industrielles de \\ l'Ingénieur}
\def\xxtete{Sciences Industrielles de l'Ingénieur}

\def\classe{\textsf{PSI$\star$ -- MP}}
\def\xxnumpartie{Cycle 05}
\def\xxpartie{Modéliser le comportement des systèmes mécaniques dans le but d'établir une loi de comportement en utilisant les méthodes énergétiques.}

\def\xxnumchapitre{Chapitre 1 \vspace{.2cm}}
\def\xxchapitre{\hspace{.12cm} Approche énergétique}

\def\xxposongletx{2}
\def\xxposonglettext{1.45}
\def\xxposonglety{19}%16

\def\xxonglet{\textsf{Cycle 05}}

\def\xxactivite{Cours}
\def\xxauteur{\textsl{Xavier Pessoles}}

\def\xxcompetences{%
\footnotesize\textsl{%
\textbf{Savoirs et compétences :}\\
\begin{itemize}[label=\ding{112},font=\color{ocre}] 
\item Mod2.C18.SF1 : Déterminer l’énergie cinétique d’un solide, ou d’un ensemble de solides, dans son mouvement par rapport à un autre solide.
\item Res1.C1.SF1 : Proposer une démarche permettant la détermination de la loi de mouvement.
\item Res1.C3.SF1 : Choisir une méthode pour déterminer la valeur des paramètres conduisant à des positions d'équilibre.
\item Mod1.C4.SF1 : Associer les grandeurs physiques aux échanges d’énergie et à la transmission de puissance.
\item Mod1.C5.SF1 : Identifier les pertes d’énergie .
\item Mod1.C6.SF1 : Évaluer le rendement d’une chaîne d’énergie en régime permanent.
\item Mod1.C5.SF2 : Déterminer la puissance des actions mécaniques extérieures à un solide ou à un ensemble de solides, dans son mouvement rapport à un autre solide.
\item Mod1.C5.SF3 : Déterminer la puissance des actions mécaniques intérieures à un ensemble de solides.
\end{itemize}
}
\normalsize}
		
\def\xxfigures{
%\includegraphics[width=4cm]{images/fig_01}\\
%\textit{Toupie}

%\includegraphics[width=4cm]{images/fig_02}\\
%\textit{Volants d'inertie d'un vilebrequin}

}%figues de la page de garde

\def\xxpied{%
Cycles 4 et 5 -- Modéliser les systèmes mécaniques \\% afin de valider leurs performances.\\
Cours% deChapitre 1 -- \xxactivite%
}

\setcounter{secnumdepth}{5}
%---------------------------------------------------------------------------


\begin{document}
\chapterimage{png/Fond_CIN}
\input{style/new_pagegarde}
\setlength{\columnseprule}{.1pt}

\vspace{2cm}
\pagestyle{fancy}
\thispagestyle{plain}
%%%%%%%%%%%%%%%%%%%%%%%%%%%%%%%%%%%ù




\section{Caractéristiques d'inertie des solides}
L'inertie d'un solide peut se << caractériser >> par la résistance ressentie lorsqu'on souhaite mettre un solide en mouvement. Pour un mouvement de translation, la connaissance de la masse permet de déterminer l'effort nécessaire à la mettre en mouvement. 
Pour un mouvement de rotation, il est nécessaire de connaître la répartition de la masse autour de l'axe de rotation.

\begin{exemple}~\\

\begin{itemize}
\item Couple pour faire tourner une hélice bipale, tripale, quadripale.
\item Couple pour faire tourner une bille et effort pour faire translater une bille.
\end{itemize}
\end{exemple}

\subsection{Détermination de la masse d'un solide}
\subsubsection{Définition}
\begin{defi}~\\
%
On peut définir la masse $M$ d'un système matériel (solide) $S$ par : 
\begin{multicols}{2}
\vspace{-1cm}
$$ M = \int\limits_S \dd m =\int\limits_{P\in V} \mu(P) \dd v $$ 
avec :
\begin{itemize}
\item $\mu(P)$ la masse volumique au point $P$;
\item $\dd v$ un élément volumique de $S$.
\end{itemize}
\end{multicols}
\vspace{.01cm}
\end{defi}
\subsubsection{Principe de conservation de la masse}

\subsection{Centre d'inertie d'un solide}
\subsubsection{Définition}

\begin{defi}[Centre d'inertie d'un solide]
La position du centre d'inertie $G$ d'un solide $S$ est définie par $\int\limits_{P\in S} \vect{GP} \dd m = \vect{0}$.
\end{defi}

Pour déterminer la position du centre d'inertie d'un solide $S$, on passe généralement par l'origine du repère associé à $S$. On a alors 
$\int\limits_{P\in S} \vect{GP} \, \dd m=\int\limits_{P\in S} \left(\vect{GO}+\vect{OP}\right) \dd m = \vect{0} 
\Leftrightarrow \int\limits_{P\in S} \vect{OG} \,\dd m =\int\limits_{P\in S} \vect{OP} \,\dd m
\Leftrightarrow  M\vect{OG} =\int\limits_{P\in S} \vect{OP} \,\dd m$.

\begin{methode}
Pour déterminer les coordonnées $\left(x_G,y_G,z_G\right)$ du centre d'inertie $G$ du solide $S$ dans la base $\repere{O}{x}{y}{z}$, on a donc :
\begin{multicols}{2}
$$
\left\{
\begin{array}{l}
M x_G =\mu \int\limits_{P\in S} x_P \,\dd V \\
M y_G =\mu \int\limits_{P\in S} y_P \,\dd V \\
M z_G =\mu \int\limits_{P\in S} z_P \,\dd V \\
\end{array}
\right. 
$$
avec : 
\begin{itemize}
  \item $\dd V$ : un élément volumique de $S$;
  \item $\mu$ : la masse volumique supposée constante.
\end{itemize}
Pour simplifier les calculs, on peut noter que le centre d'inertie appartient au(x) éventuel(s) plan(s) de symétrie du solide.

\end{multicols}



\end{methode}

\subsubsection{Centre d'inertie d'un solide constitué de plusieurs solides}
Soit un solide composé de $n$ solides élémentaires dont la position des centres d'inertie $G_i$ et les masses $M_i$ sont connues. On note $M=\sum\limits_{i=1}^{n}M_i$.  La position du centre d'inertie $G$ de l'ensemble $S$ est donné par :
$$\vect{OG}=\dfrac{1}{M}\sum\limits_{i=1}^{n}M_i \vect{OG_i} .$$


\subsubsection{Théorème de Guldin}
\paragraph{Centre d'inertie d'une courbe plane}
\paragraph{Centre d'inertie d'une surface plane}

\subsection{Grandeurs inertielles d'un solide}
\subsubsection{Matrice d'inertie}
\subsubsection{Moment d'inertie}
\subsubsection{Propriétés des matrices d'inertie}
\subsubsection{Théorème de Huygens}
\subsubsection{Rotation de la matrice d'inertie}

\section{Cinétique et dynamique du solide indéformable}
\subsection{Le torseur cinétique}
\subsubsection{Définition}
\subsubsection{Cas particuliers}

\subsection{Le torseur dynamique}
\subsubsection{Définition}
\subsubsection{Cas particuliers}

\subsection{Énergie cinétique}
\subsubsection{Définition}
\subsubsection{Cas du solide indéformable}
\subsubsection{Cas d'un système de solide}
\subsubsection{Inertie équivalente}

\section{Principe fondamental de la dynamique}

\section{Théorème de l'énergie puissance}

\section{Méthodologie}





\begin{thebibliography}{2}
   \bibitem[1]{ref1} Émilien Durif, {\it Approche énergétique des systèmes, Lycée La Martinière Monplaisir, Lyon.}
\end{thebibliography}


\end{document}




