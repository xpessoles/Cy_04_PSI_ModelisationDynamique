\documentclass[10pt,fleqn]{article} % Default font size and left-justified equations
\usepackage[%
    pdftitle={Modélisation dynamique : cinétique},
    pdfauthor={Xavier Pessoles}]{hyperref}

    
\input{style/new_style}
\input{style/macros_SII}

\fichetrue
%\fichefalse

\proftrue
\proffalse

\tdtrue
%\tdfalse

\courstrue
\coursfalse


\def\discipline{Sciences \\Industrielles de \\ l'Ingénieur}
\def\xxtete{Sciences Industrielles de l'Ingénieur}

\def\classe{\textsf{PSI$\star$ -- MP}}
\def\xxnumpartie{Cycle 04}
\def\xxpartie{Modéliser le comportement des systèmes mécaniques dans le but d'établir une loi de comportement ou de déterminer des actions mécaniques en utilisant le PFD}

\def\xxnumchapitre{Chapitre 3 \vspace{.2cm}}
\def\xxchapitre{\hspace{.12cm} Cinétique et application du Principe Fondamental de la Dynamique}




\def\xxtitreexo{Eolienne}%Motorisation du moteur Haibike}
\def\xxsourceexo{\hspace{.2cm} \footnotesize{Équipe PT -- PT$\star$ La Martinière Monplaisir}}


\def\xxposongletx{2}
\def\xxposonglettext{1.45}
\def\xxposonglety{20}
%\def\xxonglet{Part. 1 -- Ch. 3}
\def\xxonglet{Cycle 04}

\def\xxactivite{Colle 03}
\def\xxauteur{\textsl{Équipe PT -- PT$\star$ La Martinière Monplaisir}}

\def\xxcompetences{%
\textsl{%
\textbf{Savoirs et compétences :}\\
%\begin{itemize}[label=\ding{112},font=\color{ocre}] 
%\item \textit{Mod2.C13} : centre d'inertie
%\item \textit{Mod2.C14} : opérateur d'inertie
%\item \textit{Mod2.C15} : matrice d'inertie
%\end{itemize}
}}
\def\xxfigures{
%\includegraphics[width=.4\linewidth]{images/fig_00}
}%figues de la page de garde


\def\xxpied{%
Cycle 04 -- Modélisation mécanique -- Cinétique\\% afin de valider leurs performances.\\
Chapitre 3 -- \xxactivite%
}

\setcounter{secnumdepth}{5}
%---------------------------------------------------------------------------

\usepackage{pgfplots}
\begin{document}
\def\pathfig{images}
%\chapterimage{png/Fond_Cin}
\input{style/new_pagegarde}
\vspace{5cm}
\pagestyle{fancy}
\thispagestyle{plain}

\def\columnseprulecolor{\color{ocre}}
\setlength{\columnseprule}{0.4pt} 

\def\pathfig{images}

\ifprof
\else
\begin{multicols}{2}
\fi

Soit $\mathcal{R}_0=\repere{O}{x}{y}{z}$ un repère lié au support \textbf{(0)} d’une éolienne.
La girouette \textbf{(1)} est en liaison pivot d’axe $\axe{O}{z}$ avec le support \textbf{(0)}. Soit $\mathcal{R}_1=\repere{O}{x_1}{y_1}{z}$ un repère lié à la girouette \textbf{(1)}. On pose $\alpha=\angl{x}{x_1}=\angl{y}{y_1}$.

L’hélice \textbf{(2)}, de centre d’inertie $G$, de masse $M$, est en liaison pivot d’axe $\axe{G}{x_1}$ avec la girouette \textbf{(1)} avec $\vect{OG} = a \vect{x_1}$. Soit $\mathcal{R}_2=\repere{G}{x_1}{y_2}{z_2}$ un repère lié à l’hélice \textbf{(2)}. 
On donne  $\vect{GP}=b\vect{z_2}$, $\beta=\angl{y_1}{y_2}=\angl{z}{z_2}$.
et $\inertie{G}{2}=\matinertie{A}{B}{C}{0}{0}{0}{\mathcal{R}_2}$ ainsi que 
$\inertie{G}{2}=\matinertie{A_1}{B_1}{C_1}{-D_1}{-E_1}{-F_1}{\mathcal{R}_1}$.
Un balourd \textbf{(3)} (modélisant un déséquilibre de l’hélice en rotation), fixe par rapport à \textbf{(2)}, est représenté par une masse ponctuelle $m$ en $P$.


\begin{center}
\includegraphics[width=\linewidth]{images/fig_01}
\end{center}


\subparagraph{}\textit{Déterminer la projection sur l’axe $\vect{z}$ du moment cinétique en $O$ de la girouette \textbf{(1)} dans son mouvement par rapport au repère $\mathcal{R}$.}

\subparagraph{}\textit{Justifier l’allure de la matrice d’inertie de \textbf{(2)}.}

\subparagraph{}\textit{Déterminer les torseurs cinétiques en $O$ de l’hélice \textbf{(1)} et du balourd \textbf{(3)} dans leur mouvement par rapport au repère $\mathcal{R}$.}

\subparagraph{}\textit{Déterminer la projection sur l’axe $\vect{z}$ du moment dynamique en $O$ de l’hélice \textbf{(2)} dans son mouvement par rapport au repère $\mathcal{R}$.}
% Donner le résultat sous la forme  .}

\subparagraph{}\textit{Déterminer la projection sur l’axe $\vect{x_1}$ du moment dynamique en O du balourd \textbf{(3)} dans son mouvement par rapport au repère $\mathcal{R}$.}

%\subparagraph{}\textit{Déterminer l’énergie cinétique de l’éolienne E = {1, 2, 3} dans son mouvement par rapport au repère R.}

\ifprof
\else
\end{multicols}
\fi


\newpage

\begin{center}
\includegraphics[width=\linewidth]{images/cor_01}
\end{center}
\begin{center}
\includegraphics[width=\linewidth]{images/cor_02}
\end{center}


%\newpage


\end{document}
\begin{center}
\includegraphics[width=\linewidth]{images/}
\end{center}

\subparagraph{}
\textit{}
\ifprof
\begin{corrige}
\end{corrige}
\else
\fi
