\documentclass[10pt,fleqn]{article} % Default font size and left-justified equations
\usepackage[%
    pdftitle={Modélisation dynamique : cinétique},
    pdfauthor={Xavier Pessoles}]{hyperref}

    
\input{style/new_style}
\input{style/macros_SII}

\fichetrue
%\fichefalse

\proftrue
\proffalse

\tdtrue
%\tdfalse

\courstrue
\coursfalse


\def\discipline{Sciences \\Industrielles de \\ l'Ingénieur}
\def\xxtete{Sciences Industrielles de l'Ingénieur}

\def\classe{\textsf{PSI$\star$ -- MP}}
\def\xxnumpartie{Cycle 04}
\def\xxpartie{Modéliser le comportement des systèmes mécaniques dans le but d'établir une loi de comportement ou de déterminer des actions mécaniques en utilisant le PFD}

\def\xxnumchapitre{Chapitre 3 \vspace{.2cm}}
\def\xxchapitre{\hspace{.12cm} Cinétique et application du Principe Fondamental de la Dynamique}




\def\xxtitreexo{Disque non équilibré}%Motorisation du moteur Haibike}
\def\xxsourceexo{\hspace{.2cm} \footnotesize{Équipe PT -- PT$\star$ La Martinière Monplaisir}}


\def\xxposongletx{2}
\def\xxposonglettext{1.45}
\def\xxposonglety{20}
%\def\xxonglet{Part. 1 -- Ch. 3}
\def\xxonglet{Cycle 04}

\def\xxactivite{Colle 02}
\def\xxauteur{\textsl{Équipe PT -- PT$\star$ La Martinière Monplaisir}}

\def\xxcompetences{%
\textsl{%
\textbf{Savoirs et compétences :}\\
%\begin{itemize}[label=\ding{112},font=\color{ocre}] 
%\item \textit{Mod2.C13} : centre d'inertie
%\item \textit{Mod2.C14} : opérateur d'inertie
%\item \textit{Mod2.C15} : matrice d'inertie
%\end{itemize}
}}
\def\xxfigures{
%\includegraphics[width=.4\linewidth]{images/fig_00}
}%figues de la page de garde


\def\xxpied{%
Cycle 04 -- Modélisation mécanique -- Cinétique\\% afin de valider leurs performances.\\
Chapitre 3 -- \xxactivite%
}

\setcounter{secnumdepth}{5}
%---------------------------------------------------------------------------

\usepackage{pgfplots}
\begin{document}
\def\pathfig{images}
%\chapterimage{png/Fond_Cin}
\input{style/new_pagegarde}
\vspace{5cm}
\pagestyle{fancy}
\thispagestyle{plain}

\def\columnseprulecolor{\color{ocre}}
\setlength{\columnseprule}{0.4pt} 

\def\pathfig{images}

\ifprof
\else
\begin{multicols}{2}
\fi

Soit le rotor \textbf{(1)} défini ci-dessous. Il est constitué d'un arbre de masse négligeable en liaison pivot par rapport à un bâti \textbf{(0)}. Sur cet arbre est monté, en liaison complète, un disque de masse $M$, de rayon $R$ et d'épaisseur $H$. 
Le repère $\mathcal{R}'_1=\repere{G}{x_1'}{y_1'}{z_1'}$ est attaché à ce solide.

La base $\mathcal{B}'_1=\base{x_1'}{y_1'}{z_1'}$ se déduit de $\mathcal{B}_1=\base{x_1}{y_1}{z_1}$  par une rotation d'angle $\alpha$ autour de $\vect{z_1}= \vect{z_1'}$. 

La base $\mathcal{B}_1=\base{x_1}{y_1}{z_1}$ se déduit de $\mathcal{B}_0=\base{x_0}{y_0}{z_0}$  par une rotation d'angle $\theta$ autour de $\vect{x_1}= \vect{x_0}$. 

Enfin, le rotor \textbf{1} est entrainé par un moteur (non représenté) fournissant un couple noté $C_m\vect{x_0}$. 
Le montage de ce disque présente deux défauts :
\begin{itemize}
\item un défaut de perpendicularité caractérisé par l'angle $\alpha$ ;
\item un défaut d'excentricité représenté par la cote $e$.
\end{itemize}


\begin{center}
\includegraphics[width=\linewidth]{images/fig_01}
\end{center}
\begin{center}
\includegraphics[width=\linewidth]{images/fig_02}
\end{center}

\subparagraph{}\textit{Déterminer la forme de la matrice d'inertie dy cylindre en C dans la base $\mathcal{B}_1'$.}

\subparagraph{}\textit{Déterminer les éléments de réduction en $A$ du torseur dynamique de \textbf{(1)} dans son mouvement par rapport à $\mathcal{R}_0$.}

\subparagraph{}\textit{Appliquer le PFD pour déterminer les inconnues de liaison.}

\ifprof
\else
\end{multicols}
\fi


\newpage

\begin{center}
\includegraphics[width=\linewidth]{images/cor_01}
\end{center}
\begin{center}
\includegraphics[width=\linewidth]{images/cor_02}
\end{center}
\begin{center}
\includegraphics[width=\linewidth]{images/cor_03}
\end{center}
\begin{center}
\includegraphics[width=\linewidth]{images/cor_04}
\end{center}
\begin{center}
\includegraphics[width=\linewidth]{images/cor_05}
\end{center}


%\newpage


\end{document}
\begin{center}
\includegraphics[width=\linewidth]{images/}
\end{center}

\subparagraph{}
\textit{}
\ifprof
\begin{corrige}
\end{corrige}
\else
\fi
