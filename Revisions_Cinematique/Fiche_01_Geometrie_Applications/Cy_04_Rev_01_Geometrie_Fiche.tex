\documentclass[10pt,fleqn]{article} % Default font size and left-justified equations
\usepackage[%
    pdftitle={Modélisation cinématique : Géométrie},
    pdfauthor={Xavier Pessoles}]{hyperref}

\input{style/new_style}
\input{style/macros_SII}
\usepackage{multicol}
\usepackage{siunitx}
%\usepackage{picins}
\fichetrue
%\fichefalse

\proftrue
%\proffalse

\tdtrue
%\tdfalse

\courstrue
\coursfalse

% -------------------------------------
% Déclaration des titres
% -------------------------------------

\def\discipline{Sciences \\Industrielles de \\ l'Ingénieur}
\def\xxtete{Sciences Industrielles de l'Ingénieur}


\def\classe{\textsf{PSI$\star$ -- MP}}
\def\xxnumpartie{Rév -- Cin}
\def\xxpartie{Modéliser le comportement géométrique des systèmes mécaniques}

\def\xxnumchapitre{Révision 1 \vspace{.2cm}}
\def\xxchapitre{\hspace{.12cm} Modélisation géométrique -- Lois entrées-sorties}

\def\xxposongletx{2}
\def\xxposonglettext{1.45}
\def\xxposonglety{19}%16

\def\xxonglet{Rév -- Cin}

\def\xxactivite{Application 01}
\def\xxauteur{\textsl{Xavier Pessoles}}


\def\xxtitreexo{Micromoteur d'un avion de modélisme}
\def\xxsourceexo{\hspace{.2cm} \footnotesize{Xavier Pessoles}}

\def\xxcompetences{%
\textsl{%
\textbf{Savoirs et compétences :}\\
}}

\def\xxfigures{
\includegraphics[width=.8\textwidth]{images/fig_01}
}%figues de la page de garde

\def\xxpied{%
Révision cinématique -- Modélisation géométrique\\
Fiche 1 -- \xxactivite%
}

\setcounter{secnumdepth}{5}
%---------------------------------------------------------------------------


\begin{document}
%\chapterimage{png/Fond_Cin}
\input{style/new_pagegarde}
\vspace{4cm}
\pagestyle{fancy}
\thispagestyle{plain}
\begin{multicols}{2}

\section*{Mise en situation}
La mise en mouvement d'une certaine catégorie d'avions de modélisme est assurée par un moteur thermique.   La figure ci-dessous propose un éclaté d'un modèle 3D ainsi que le schéma cinématique associé. 

\begin{center}
\includegraphics[width=.45\linewidth]{images/fig_03}
\includegraphics[width=.45\linewidth]{images/fig_04}
\end{center}

On appelle : 
\begin{itemize}
\item \textbf{(1)} le vilebrequin, solidaire de l'hélice de l'avion;
\item \textbf{(2)} la bielle;
\item \textbf{(3)} le piston.
\end{itemize}

\begin{obj}

\end{obj}


\section*{Modélisation}
\begin{center}
\includegraphics[width=\linewidth]{images/fig_06}
\end{center}


\end{multicols}

%\begin{center}
%\includegraphics[width=\linewidth]{images/fig_04}
%%\textit{}
%\end{center}


\end{document}

\subparagraph{}\textit{}


\begin{center}
\includegraphics[width=\linewidth]{images/fig_06}
%\textit{}
\end{center}
\begin{center}
\includegraphics[width=\linewidth]{images/img_04}
%\textit{}
\end{center}

