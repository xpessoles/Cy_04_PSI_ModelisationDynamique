\documentclass[10pt,fleqn]{article} % Default font size and left-justified equations
\usepackage[%
    pdftitle={Modélisation SLCI : Rapidité des systèmes},
    pdfauthor={Xavier Pessoles}]{hyperref}
    
\input{style/new_style}
\input{style/macros_SII}
\usepackage{multicol}
\usepackage{siunitx}
%\usepackage{picins}
\fichetrue
%\fichefalse

\proftrue
%\proffalse

\tdtrue
%\tdfalse

\courstrue
\coursfalse

\def\discipline{Sciences \\Industrielles de \\ l'Ingénieur}
\def\xxtete{Sciences Industrielles de l'Ingénieur}

\def\classe{PSI$\star$ -- MP}
\def\xxnumpartie{Cycle 02}
\def\xxpartie{Modéliser les systèmes asservis dans le but de prévoir leur comportement}


\def\xxnumchapitre{Chapitre 2 \vspace{.2cm}}
\def\xxchapitre{\hspace{.12cm} Rapidité des systèmes}


\def\xxtitreexo{Radar d'avion}
\def\xxsourceexo{\hspace{.2cm} \footnotesize{F. Mathurin}}


\def\xxposongletx{2}
\def\xxposonglettext{1.45}
\def\xxposonglety{20}
%\def\xxonglet{Part. 1 -- Ch. 3}
\def\xxonglet{Cycle 02}

\def\xxactivite{TD 01}
\def\xxauteur{\textsl{F. Mathurin.}}

\def\xxcompetences{%
\textsl{%
\textbf{Savoirs et compétences :}\\
%Les sources sont associées par un \emph{hacheur série}. La détermination des grandeurs électriques associées à ce montage permet de conclure vis à vis du cahier des charges.
%\noindent \textbf{Résoudre :} à partir des modèles retenus :
%\begin{itemize}[label=\ding{112},font=\color{ocre}] 
%\item choisir une méthode de résolution analytique, graphique, numérique;
%\item mettre en \oe{}uvre une méthode de résolution.
%\end{itemize}
%\begin{itemize}[label=\ding{112},font=\color{ocre}] 
%\item \textit{Rés -- C1.1 :} Loi entrée sortie géométrique et cinématique -- Fermeture géométrique.
%\end{itemize}
%
%\noindent \textit{Mod2 -- C4.1 :} Représentation par schéma bloc.
}}

\def\xxfigures{
\includegraphics[width=.6\linewidth]{images/fig_01}
}%figues de la page de garde


\def\xxpied{%
Cycle 01 -- Modéliser le comportement des systèmes multiphysiques\\
Chapitre 2 -- \xxactivite%
}

\setcounter{secnumdepth}{5}
%---------------------------------------------------------------------------

\usepackage{pgfplots}
\begin{document}

%\chapterimage{png/Fond_Cin}
\input{style/new_pagegarde}
\vspace{5cm}
\pagestyle{fancy}
\thispagestyle{plain}

\def\columnseprulecolor{\color{ocre}}
\setlength{\columnseprule}{0.4pt} 

\def\pathfig{images}

\begin{multicols}{2}
Le support d'étude est un radar d'avion. Il permet au pilote de connaître la position des engins extérieurs (avions, hélicoptères, bateaux, ...). L'objectif de cette étude est de vérifier les performances décrites dans l’extrait de cahier des charges de ce système. 

\begin{center}
\includegraphics[width=\linewidth]{images/fig_02}
%\textit{}
\end{center}

On réalise un asservissement de position angulaire du radar d’avion : l'angle souhaité est $\theta_c (t)$, l'angle réel du radar est $\theta_r (t)$. La différence des deux angles est transformée en une tension $u_m (t)$, selon la loi $u_m (t)= A(\theta_c - \theta_r (t))$. La tension $u_m (t)$ engendre, via un moteur de fonction de transfert $H_m (t)$, une vitesse angulaire $\omega_m (t)$. Cette vitesse angulaire est réduite grâce à un réducteur de vitesse, selon la relation $\omega_r (t) = B \omega_m (t)$ ($B<1$), $\omega_r (t)$ étant la vitesse angulaire du radar. 

\subparagraph{}\textit{Réaliser le schéma-bloc du système. }

Les équations du moteur à courant continu, qui est utilisé dans la motorisation, sont les suivantes : 
$$
u_m(t)=e(t)+Ri(t) 
\quad e(t)=k_e\omega_m(t) 
$$
$$J \dfrac{\text{d}\omega_m(t)}{\text{d}t} = c_m(t)
\quad c_m(t) = k_m i(t)
$$


Avec : 
\begin{itemize}
\item $u(t)$ : tension aux bornes du moteur (en V) (entrée du moteur);
\item $e(t)$ : force contre-électromotrice (en V);
\item $i(t)$ : intensité (en A);
\item $\omega_m (t)$ : vitesse de rotation du moteur (en rad/s);
\item $C_m (t)$ : couple moteur (en N.m) (un couple est une action mécanique qui tend à faire tourner); 
\item $J$ : inertie équivalente en rotation de l’arbre moteur (en $\text{kg.m}^2$): 
\item $R$ : résistance électrique du moteur;
\item $k_e$ : constante de force contre-électromotrice;
\item $k_m$ : constante de couple.
\end{itemize}


\subparagraph{}\textit{Déterminer la fonction de transfert $H_m(p)=\dfrac{\Omega_m(p)}{U_m(p)}$.}

\subparagraph{}\textit{Montrer que $H_m(p)$ peut se mettre sous la forme canonique $H_m(p)=\dfrac{K_m}{1+T_m p}$ et déterminer les valeurs littérales de $K_m$ et $T_m$.}

\subparagraph{}\textit{Préciser la valeur de $\omega_m (t)$ à l'origine, la pente de la tangente à l'origine de $\omega_m (t)$ et la valeur finale atteinte par $\omega_m (t)$ quand t tend vers l’infini. }

\subparagraph{}\textit{Déterminer la fonction de transfert $H(p)=\dfrac{\theta_r(p)}{\theta_c(p)}$. Montrer que cette fonction peut se mettre sous la forme d'un système du second ordre dont on précisera les caractéristiques.}

  La réponse indicielle de $H(p)$ à un échelon unitaire est donnée sur la figure suivante : 
  
\begin{center}
\includegraphics[width=\linewidth]{images/fig_03}
%\textit{}
\end{center}


\subparagraph{}\textit{Déterminer, en expliquant la démarche utilisée, les valeurs numériques de $K$, $z$ et $\omega_0$.}

   Sans préjuger du résultat trouvé dans la question précédente, on prendra, pour la suite : $K = 1$, $z = 0,5$ et $\omega_0 = \SI{15}{rad/s}$.


\subparagraph{}\textit{Déterminer, en expliquant la démarche utilisée, le temps de réponse à 5\%. Conclure quant la capacité du radar à vérifier le critère de rapidité du cahier des charges. }

On améliore la performance du radar en ajoutant un composant électronique (un correcteur) entre l'amplificateur et le moteur. La nouvelle fonction de transfert est : 
$
H(p)=\dfrac{1}{\left(1+0,05p \right)\left(1+0,0005p \right)\left(1+0,002p \right)}
$.
 

\subparagraph{}\textit{Tracer le diagramme de Bode asymptotique (en gain et en phase) de cette fonction de transfert.}

\subparagraph{}\textit{Déterminer $G$ et $\varphi$ pour $\omega = \SI{10}{rad/s}$.}
\subparagraph{}\textit{Déterminer, en régime permanent, $\theta_r (t)$ pour une entrée $\theta_c (t) = 0,2 \sin(10t)$.}

Pour $\omega < \SI{20}{rad/s}$, on a $H(p)\simeq \dfrac{1}{1+0,05p}$.

\subparagraph{}\textit{Déterminer, sur cette approximation, la pulsation de coupure à $-\SI{3}{dB}$. Conclure quant à la capacité du radar à satisfaire le critère de bande passante du cahier des charges.  }

\subparagraph{}\textit{Déterminer, sur cette approximation, le temps de réponse à 5\% du système. Conclure quant à la capacité du radar à satisfaire le critère de rapidité du cahier des charges. }

\end{multicols}

%\begin{center}
%\includegraphics[width=\linewidth]{images/fig_04}
%%\textit{}
%\end{center}


\end{document}

\subparagraph{}\textit{}


\begin{center}
\includegraphics[width=\linewidth]{images/fig_06}
%\textit{}
\end{center}
\begin{center}
\includegraphics[width=\linewidth]{images/img_04}
%\textit{}
\end{center}

