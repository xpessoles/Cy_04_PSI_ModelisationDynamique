%%%% Paramétrage du cours %%%%
\def\xxactivite{Cours}
\def\xxauteur{\textsl{Xavier Pessoles}}

\fichetrue
\proftrue
\tdfalse
\coursfalse

\def\xxnumpartie{Rév -- Cin}
\def\xxpartie{Modélisation des actions mécaniques dans les systèmes}
\def\xxnumchapitre{ Révisions 4\vspace{.2cm}}
\def\xxchapitre{\hspace{.12cm} Transmetteurs de puissance}

\def\xxpied{%
Révisions Cinématique -- \xxchapitre \\
Fiche 4 -- \xxactivite%
}


\def\xxcompetences{%
\textsl{%
\textbf{Savoirs et compétences :}\\
%\begin{itemize}[label=\ding{112},font=\color{ocre}] 
%\item \textit{Res1.C2} : principe fondamental de la dynamique;
%\item \textit{Res1.C1.SF1} : proposer une démarche permettant la détermination de la loi de mouvement.
%\end{itemize}
}}


\iflivret
\input{../../style/new_pagegarde}
\else
\input{../../style/new_pagegarde}
\fi
\setlength{\columnseprule}{.1pt}

\vspace{2cm}
\pagestyle{fancy}
\thispagestyle{plain}

\section*{Transmission par engrenages}
\begin{defi}{Engrenage}
Un engrenage est constitué de deux roues dentées en contact. Une roue dentée est caractérisée (entre autre) par son nombre de dents $Z$, son diamètre primitif $D$ en \si{mm} et son module en \si{mm}. On a $D=mZ$.
Pour que deux dents engrènent elles doivent avoir le même module.
\end{defi}

\subsection*{Engrenage -- Contact extérieur}
\begin{minipage}[c]{.4\linewidth}
\begin{resultat}
$$
\dfrac{\omega(2/0)}{\omega(1/0)}= (-1)^n \dfrac{Z_1}{Z_2}=-\dfrac{Z_1}{Z_2}
$$
$n$ caractérise le nombre de contacts extérieurs, ici $n=1$.
\end{resultat}
\end{minipage}\hfill
\begin{minipage}[c]{.6\linewidth}
\begin{center}
\includegraphics[height=3cm]{images/fig_01.png}
\end{center}
\end{minipage}

\subsection*{Engrenage -- Contact intérieur}
\begin{minipage}[c]{.4\linewidth}
\begin{resultat}
$$
\dfrac{\omega(2/0)}{\omega(1/0)}= (-1)^n \dfrac{Z_1}{Z_2}=+\dfrac{Z_1}{Z_2}
$$
$n$ caractérise le nombre de contacts extérieurs, ici $n=0$.
\end{resultat}
\end{minipage}\hfill
\begin{minipage}[c]{.6\linewidth}
\begin{center}
\includegraphics[height=3cm]{images/fig_02.png}
\end{center}
\end{minipage}

\subsection*{Train d'engrenages à axes fixes}
\begin{minipage}[c]{.45\linewidth}
\begin{resultat}
$$
\dfrac{\omega(4/0)}{\omega(1/0)}= (-1)^n \dfrac{\Pi Z_{\text{menantes}}}{\Pi Z_{\text{menées}}}=-\dfrac{Z_1Z_{22}}{Z_{21}Z_4}
$$
$n$ caractérise le nombre de contacts extérieurs, ici $n=1$.
\end{resultat}
\end{minipage}\hfill
\begin{minipage}[c]{.5\linewidth}
\begin{center}
\includegraphics[height=4.5cm]{images/fig_06.png}
\end{center}
\end{minipage}


\subsection*{Train d'engrenages épicycloïdal}
\begin{minipage}[c]{.65\linewidth}
\begin{methode}
\begin{enumerate}
\item On identifie le porte-satellite, ici 3.
\item On bloque le porte-satellite. On peut alors se ramener au cas du train simple (voir ci-dessus). 
\item On écrit le rapport de vitesse \textbf{par rapport au porte-satelltite 3} :$
\dfrac{\omega(4/3)}{\omega(1/3)}=-\dfrac{Z_1Z_{22}}{Z_{21}Z_4} = K
$ (raison du train épicycloïdal).
\item En fonction de la roue bloquée, on réalise une décomposition des vitesses. Par exemple, Si 4 est bloquée, on   peut chercher à établir $\dfrac{\omega(3/0)}{\omega(1/0)}$. 
\item On repart du point 3 et on a : $\dfrac{\omega(4/3)}{\omega(1/3)}= K$
$\Leftrightarrow \dfrac{\omega(4/0)+\omega(0/3)}{\omega(1/0)+\omega(0/3)}= K$
$\Leftrightarrow \dfrac{-\omega(3/0)}{\omega(1/0)-\omega(3/0)}= K$
%$\Leftrightarrow -\omega(3/0) = K\omega(1/0)-K\omega(3/0)$
%$\Leftrightarrow \left(K-1\right)\omega(3/0) = K\omega(1/0)$
$\Leftrightarrow \dfrac{\omega(3/0)}{\omega(1/0)} = \dfrac{K}{K-1}$.
\end{enumerate}
\end{methode}
\end{minipage}\hfill
\begin{minipage}[c]{.35\linewidth}
\begin{center}
\includegraphics[height=4.5cm]{images/fig_05.png}
\end{center}
\end{minipage}

\subsection*{Système pignon -- crémaillère}
\begin{minipage}[c]{.4\linewidth}
\begin{resultat}
Soit $R$ le rayon primitif du pignon. On a $V(2/0) = \pm R \omega(1/0)$.
\end{resultat}
\end{minipage}\hfill
\begin{minipage}[c]{.6\linewidth}
\begin{center}
\includegraphics[height=3cm]{images/fig_03.png}
\end{center}
\end{minipage}

\subsection*{Transmission par poulie chaine et par poulie courroie}
\begin{minipage}[c]{.4\linewidth}
\begin{resultat}

$\dfrac{\omega(2/0)}{\omega(1/0)} = \dfrac{D_1}{D_2}$.
\end{resultat}
\end{minipage}\hfill
\begin{minipage}[c]{.6\linewidth}
\begin{center}
\includegraphics[height=3cm]{images/fig_07.png}
\end{center}
\end{minipage}

\subsection*{Roue et vis sans fin}
\begin{minipage}[c]{.4\linewidth}
\begin{resultat}
Soit $Z$ le nombre de dents de la roue et $n$ le nombre de filets de la vis, on a 
$\dfrac{\omega(2/0)}{\omega(1/0)} = \pm \dfrac{n}{Z}$.
\end{resultat}
\end{minipage}\hfill
\begin{minipage}[c]{.6\linewidth}
\begin{center}
\includegraphics[height=3cm]{images/fig_04.png}
\end{center}
\end{minipage}


\subsection*{Système vis-écrou}
\begin{resultat}
En notant $v$ la vis et $e$ l'écrou, soit $p$ le pas de la vis (ici à droite) on a $v(v/e)=\omega(v/e) \dfrac{\text{pas}}{2 \pi}$.
\end{resultat}

