%%%% Paramétrage du cours %%%%
\def\xxactivite{Cours}
\def\xxauteur{\textsl{Xavier Pessoles}}

\fichetrue
\proftrue
\tdfalse
\coursfalse

\def\xxnumpartie{Rév -- Cin}
\def\xxpartie{Modélisation des actions mécaniques dans les systèmes}
\def\xxnumchapitre{ Révisions 4\vspace{.2cm}}
\def\xxchapitre{\hspace{.12cm} Transmetteurs de puissance}

\def\xxpied{%
Révisions Cinématique -- \xxchapitre \\
Fiche 4 -- \xxactivite%
}


\def\xxcompetences{%
\textsl{%
\textbf{Savoirs et compétences :}\\
%\begin{itemize}[label=\ding{112},font=\color{ocre}] 
%\item \textit{Res1.C2} : principe fondamental de la dynamique;
%\item \textit{Res1.C1.SF1} : proposer une démarche permettant la détermination de la loi de mouvement.
%\end{itemize}
}}


\iflivret
\input{../../style/new_pagegarde}
\else
\input{../../style/new_pagegarde}
\fi
\setlength{\columnseprule}{.1pt}

\vspace{2cm}
\pagestyle{fancy}
\thispagestyle{plain}

\section*{Transmission par engrenages}
\begin{defi}{Engrenage}
Un engrenage est constitué de deux roues dentées en contact. Une roue dentée est caractérisée (entre autre) par son nombre de dents $Z$, son diamètre primitif $D$ en \si{mm} et son module en \si{mm}. On a $D=mZ$.
Pour que deux dents engrènent elles doivent avoir le même module.
\end{defi}

\subsection*{Engrenage -- Contact extérieur}

\subsection*{Engrenage -- Contact intérieur}

\subsection*{Train d'engrenages à axes fixes}

\subsection*{Train d'engrenages épicycloïdal}

\section*{Transmission par poulie chaine et par pourle courroie}

\section*{Système vis-écrou}

