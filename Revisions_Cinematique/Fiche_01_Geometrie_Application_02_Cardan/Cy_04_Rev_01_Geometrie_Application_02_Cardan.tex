\documentclass[10pt,fleqn]{article} % Default font size and left-justified equations
\usepackage[%
    pdftitle={Modélisation cinématique : Géométrie},
    pdfauthor={Xavier Pessoles}]{hyperref}

\input{style/new_style}
\input{style/macros_SII}
\usepackage{multicol}
\usepackage{siunitx}
%\usepackage{picins}
\fichetrue
%\fichefalse

\proftrue
\proffalse

\tdtrue
%\tdfalse

\courstrue
\coursfalse

% -------------------------------------
% Déclaration des titres
% -------------------------------------

\def\discipline{Sciences \\Industrielles de \\ l'Ingénieur}
\def\xxtete{Sciences Industrielles de l'Ingénieur}


\def\classe{\textsf{PSI$\star$ -- MP}}
\def\xxnumpartie{Rév -- Cin}
\def\xxpartie{Modéliser le comportement géométrique des systèmes mécaniques}

\def\xxnumchapitre{Révision 1 \vspace{.2cm}}
\def\xxchapitre{\hspace{.12cm} Modélisation géométrique -- Lois entrées-sorties}

\def\xxposongletx{2}
\def\xxposonglettext{1.45}
\def\xxposonglety{19}%16

\def\xxonglet{\textsf{Rév -- Cin}}

\def\xxactivite{Application 02}
\def\xxauteur{\textsl{Xavier Pessoles}}


\def\xxtitreexo{Joint de cardan}
\def\xxsourceexo{\hspace{.2cm} \footnotesize{Xavier Pessoles}}

\def\xxcompetences{%
\textsl{%
\textbf{Savoirs et compétences :}\\
}}

\def\xxfigures{
\includegraphics[width=.6\textwidth]{images/fig3_1}
}%figues de la page de garde

\def\xxpied{%
Révision cinématique -- Modélisation géométrique\\
Fiche 1 -- \xxactivite%
}

\setcounter{secnumdepth}{5}
%---------------------------------------------------------------------------


\begin{document}
%\chapterimage{images/Fond_Cin}
\input{style/new_pagegarde}
\vspace{4cm}
\pagestyle{fancy}
\thispagestyle{plain}


\def\columnseprulecolor{\color{ocre}}
\setlength{\columnseprule}{0.4pt} 

\ifprof
\else
\begin{multicols}{2}
\fi
\subsection*{Joint de Cardan}
\setcounter{subparagraph}{0}


Un joint de Cardan est un accouplement qui permet de transmettre un mouvement de rotation entre deux arbres concourants mais non alignés. L'angle maximum pratiquement utilisé entre les arbres est de $45\textdegree$. Une application courante est la transmission entre boite de vitesses  et roues-avant d’une voiture. 

Les vues ci-dessous donnent des images d’un joint de cardan.

\begin{center}

%\includegraphics[width=.5\linewidth]{images/fig3_1}  

\includegraphics[width=0.5\linewidth]{images/fig3_2}  

\includegraphics[width=.5\linewidth]{images/fig3_3} 

\end{center}

La modélisation suivante est proposée.
\begin{center}
\includegraphics[width=\linewidth]{images/fig3_4} 
\end{center}

On appelle : 
\begin{itemize}
\item $\mathcal{R}$ le repère lié au solide $R$ considéré comme fixe. $\mathcal{R}=\left(O,\vect{x},\vect{y},\vect{z} \right)$;
\item $\mathcal{R}'$ le repère lié au solide R considéré comme fixe. $\mathcal{R}'=\left(O,\vect{u},\vect{v},\vect{z} \right)$. On pose $\alpha = \left(\vect{y},\vect{v} \right)$ (constant);
\item $\alpha$ l'"angle de brisure";
\item $\mathcal{R}_1$ le repère lié au solide 1. $\mathcal{R}_1 = \left(O,\vect{x_1},\vect{y},\vect{z_1} \right)$. On pose  $\theta_1 = \left(\vect{x},\vect{x_1} \right)$;
\item $\mathcal{R}_3$ le repère lié au solide 3. $\mathcal{R}_3 = \left(O,\vect{x_3},\vect{v},\vect{z_3} \right)$. On pose $\theta_3 = \left(\vect{u},\vect{x_3} \right)$.
\end{itemize}

\subparagraph{}
\textit{Tracer en vue orthogonale, les trois dessins (figures de changement de base) permettant le passage de $\mathcal{R}$ à $\mathcal{R}_1$ , de $\mathcal{R}$ à $\mathcal{R}'$ et de $\mathcal{R}'$ à $\mathcal{R}_3$.}
\ifthenelse{\boolean{prof}}{
\begin{corrige}
\end{corrige}
}{}

\subparagraph{}
\textit{Exprimer la condition géométrique sur 2 permettant de lier $\mathcal{R}_1$ à $\mathcal{R}_3$.}
\ifthenelse{\boolean{prof}}{
\begin{corrige}
\end{corrige}
}{}

\subparagraph{}
\textit{Développer cette relation et trouver la loi entrée sortie : $\theta_3 = f(\theta_1 , \alpha)$. Tracer, pour $\alpha=45\textdegree$, la courbe représentant l’évolution de la sortie $\theta_3$ en fonction de l’entrée $\theta_1$ avec $\theta_1$ variant de $-\pi$ à $+\pi$.}
\ifthenelse{\boolean{prof}}{
\begin{corrige}
\end{corrige}
}{}

\subparagraph{}
\textit{Dériver cette relation par rapport au temps pour trouver la vitesse de sortie $\dot{\theta_3}$ en fonction de la vitesse d’entrée $\dot{\theta_1}$, de $\theta_1$ et de $\alpha$.}

\ifthenelse{\boolean{prof}}{
\begin{corrige}
\end{corrige}
}{}

\subparagraph{}
\textit{Tracer l’évolution de la vitesse de sortie $\dot{\theta_3}$ en fonction notamment de l’évolution de l’angle d’entrée $\theta_1$. On prendra un angle de brisure de $45\textdegree$ et une vitesse d’entée constante $\dot{\theta_1}$ de 1 rad/s.}
\ifthenelse{\boolean{prof}}{
\begin{corrige}
\end{corrige}
}{}

\subparagraph{}
\textit{Conclure sur une des propriétés de ce mécanisme.}
\ifthenelse{\boolean{prof}}{
\begin{corrige}
\end{corrige}
}{}


\end{multicols}
\end{document}
\ifprof
\else
\end{multicols}
\fi

%\begin{center}
%\includegraphics[width=\linewidth]{images/fig_04}
%%\textit{}
%\end{center}

\end{document}

\subparagraph{}\textit{}


\begin{center}
\includegraphics[width=\linewidth]{images/fig_06}
%\textit{}
\end{center}
\begin{center}
\includegraphics[width=\linewidth]{images/img_04}
%\textit{}
\end{center}

